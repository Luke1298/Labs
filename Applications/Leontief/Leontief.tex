\lab{Applications}{Leontief Input-Output Models}{Leontieff Input-Output Models}
\label{Leontief}

\objective{Explain the basics of input-output models, and apply them to analyze a state economy.}

One of the primary applications of linear algebra is modeling interactions between a large number of variables in a concise way.
We can use it to simplify the analysis of very complicated interactions in a variety of settings.

For example, in economics, one might be interested in how one the inputs of one sector of an economy are reliant upon the outputs of another economy.
To build a car, you need refined steel, plastic, and glass (among other materials).
These are all outputs of other sectors of the economy.
If there is an increase in car production there will necessarily be increases in production in other sectors of the economy like steel, plastic, and glass.
This type of interaction can be described using what is called a \emph{Leontief Input-Output model}, or just an Input-Output model.

Consider a simple economy divided up into sectors representing agriculture, textiles, and construction.
Given sufficient data, we can estimate how much of the output from one economy is used as an input to another.
An array representing the required inputs in this simple economy is shown in \ref{IOCoefTable}.
This is called a consumption matrix.
It can be especially useful when determining the short term impacts of a given change in the economy.
Directly, its columns represent the inputs required to produce one additional unit of output.
For example, to produce one extra unit of agriculture output we need to input about .4 units of agriculture, .04 units of textiles and .08 units of construction.

\begin{table}
\begin{center}
\begin{tabular}{|c|c|c|c|}
\hline
& Agriculture & Textiles & Construction \\ \hline
Agriculture & .4167 & .5357 & .25 \\ \hline
Textiles & .0417 & .0893 & .1667 \\ \hline
Construction & .0833 & .0714 & .0417 \\ \hline
\end{tabular}
\caption{Input Output Coefficients.
The columns represent the amount of each product needed to produce one additional unit of output.}
\label{IOCoefTable}
\end{center}
\end{table}

This sort of matrix can easily be computed from a table of inputs and outputs like the one shown in Table \ref{IORawTable}.
A matrix of input and output values like the one shown in Table \ref{IORawTable} is called an exchange matrix.
If we divide each column by the output of the corresponding industry we can obtain the coefficient matrix (consumption matrix).
We can do this in Python as follows, and we obtain the coefficients in Table \ref{IOCoefTable}.

\begin{lstlisting}
>>> import numpy as np
>>> IO = np.array([[250., 150., 30., 600.], [25., 25., 20., 280.], [50., 20., 5., 120.]])
>>> IOCoeff = IO[:,:3] / IO[:,3]
\end{lstlisting}

\begin{table}
\begin{center}
\begin{tabular}{|c|c|c|c|c|}
\hline
& Agriculture & Textiles & Construction & Total Output \\ \hline
Agriculture & 250 & 150 & 30 & 600 \\ \hline
Textiles & 25 & 25 & 20 & 280 \\ \hline
Construction & 50 & 20 & 5 & 120 \\ \hline
Total Input & 325 & 195 & 55 & \\ \hline
\end{tabular}
\caption{Raw Input Output Data.
Columns represent inputs to an industry, and rows represent outputs.
Notice that not all the total output is used as input for the economy itself.}
\label{IORawTable}
\end{center}
\end{table}

We will now consider the sorts of things these matrices tell us.
By way of example, we will consider the total output needed to produce one extra unit of agriculture.
This certainly costs one unit of agriculture without accounting for the inputs required, but in order to produce one unit of agriculture, we also need inputs of .4167 units of agriculture, .0417 units of textiles, and .0833 units of construction.
We also need the inputs necessary to produce those inputs, and the inputs necessary to produce the inputs to the inputs for the inputs, etc.
Letting $d$ be the $3\times 1$ array of needed goods and $C$ be the coefficient matrix, the total cost of producing the outputs in $d$ will be

\[ d + C d + C^2 d + C^3 d + \dots = \left( \sum_{n=0}^{\infty} C^n \right) d = \left( I - C \right)^{-1} d\]

This is just a geometric series in $C$ (recall that $C^0$ is just the identity matrix).
The last formula for evaluating the geometric series can be derived in the same way as the formula for a geometric series of complex numbers.
It can be shown that such a geometric matrix series converges if and only if the absolute value of each of the eigenvalues of the matrix is strictly less than one.
This is a constraint that will be built into any real system, and the non-convergence of such a series would indicate a problem with the data.

Since the overall effect of the increase in demand from inputs to inputs to inputs, etc. may not happen immediately, it may often be helpful to just evaluate the first two terms in this summation.

\begin{problem}
What is the cost of producing an additional 50\% more units of construction?
Calculate the answer in two different ways.
For the first way, you will approximate the answer by using the geometric series.
Write a function named \li{geomSeries} that takes an $n \times n$ array $C$ and an $n$-dimensional
vector $d$ as inputs, as well as an integer $m$, and returns $(\sum_{i=0}^m C^i)d$.
You can then use this function to approximate the solution for $m=5$.
Also calculate the exact answer by evaluating $(I-C)^{-1}d$.
Notice how a distortion in the market for construction can drastically affect the need for other goods as well.
\end{problem}

If $X$ represents the total output of the economy, we can represent the ``net'' output to consumers $D$ (demand) by the following equation:
\[ D = X - C X \]
In other words, the amount of output used for consumption is the total output minus the portion that is reused as input for the economy itself.

This agrees with our geometric series since the total output of the economy must be the total output required to produce everything that isn't reused, i.e.
\[ X = \left( \sum_{n=0}^{\infty} C^n \right) D = \left( I - C \right)^{-1} D \]
Which is equivalent to $ D = X - C X $.

\begin{problem}
What are the demand levels for the three product economy?
\end{problem}

\begin{problem}
A city is trying to determine how to allocate funds for the celebration of the centennial anniversary of its founding.
It is expecting to take in \$100000 in space (lodging, etc.) \$100000 in consumable goods (food, and other retail items), and \$40000 in services.
Given that the coefficient matrix for space, consumable goods, and services is
\[ \begin{bmatrix}
.2 & .3 & .3 \\
.1 & .2 & .3 \\
.2 & .2 & .2 \\
\end{bmatrix} \]
Estimate the amount the city should be prepared to produce of each good.
\end{problem}

\section*{Importing CSV Data with Missing Values}
Before we get to the next problem, we review some important skills concerning reading and writing data with CSV files.
Recall that a CSV file, or a `Comma Separated Values' file, contains tabular data with the rows on separate lines
of the file, and the columns separated by some delimiter character, such as a comma or a white space. This is a
ubiquitous format for storing and transferring data, and Python has good support for it. We will use the \li{csv}
module to examine the contents of the file \li{missing_data.txt}.

\begin{lstlisting}
>>> import csv
>>> with open("missing_data.txt", "r") as csvfile:
>>>     myReader = csv.reader(csvfile, delimiter=';')
>>>     for row in myReader:
>>>         print row
['1.0', '2.2', '-3.0', '???']
['0.0', '2.1', '???', '5.1']
['0.4', '2.3', '-2.8', '4.8']
['0.3', '???', '-3.1', '5.0']
\end{lstlisting}

Notice the \li{with ... as ...} block. This is the preferred way to open and read from files, as it takes care of
closing the file once you exit the code block. We next created a \li{csv.reader} object around
the file, and specified the delimiter character to be a semi-colon. Then we iterated through the rows of our reader
and printed the contents.

Notice the entries that consist of three question marks. This indicates missing data. Such situations arise frequently
in real life, and we need to know ways to cope with imperfect or incomplete datasets. NumPy provides a very convenient
way to import homogeneous data from CSV files, even with missing data. We will use the \li{np.genfromtxt} function to
extract the data and fill in the missing values. In this example, we will set the missing value in the second column to
2.2, the missing value in the third column to -3.0, and the missing value in the fourth column to 4.9.

\begin{lstlisting}
>>> data = np.genfromtxt("missing_data.txt", delimiter=';',
                         missing_values = ['???'],
                         filling_values = {1:2.2, 2:-3.0, 3:4.9})
>>> data
array([[ 1. ,  2.2, -3. ,  4.9],
       [ 0. ,  2.1, -3. ,  5.1],
       [ 0.4,  2.3, -2.8,  4.8],
       [ 0.3,  2.2, -3.1,  5. ]])
\end{lstlisting}

We now have a NumPy array containing our filled-in data, and we can continue with whatever computations we desire. Notice
that we still needed to specify the delimiter character. The \li{missing_values} argument is a list of strings that
designate missing values, and the \li{filling_values} argument is a dictionary that maps the index of a column to the
value used to replace all the missing values in that column.
% The value of a \li{None} key in this dictionary will be used as a default fill value, except where you specify.
% Except the above statement doesn't seem to be true!!!
If you want to replace all missing values with the same
filling value, the \li{filling_values} argument can be set to this single filling value (if the desired filling value
happens to be 0, you must enclose it in quotes, i.e. \li{filling_values = '0'}).

If for some reason we want to omit certain rows from the beginning or end of the CSV file, we can use the \li{skip_header} or \li{skip_footer}
arguments, which are integers representing the number of lines at the beginning or end of the file to skip, respectively.
If we want to extract only certain columns from the CSV file, we can use the \li{usecols} argument, which is a list of
the column indices that we want to extract. In this next example, we omit the first row and the first column.

\begin{lstlisting}
>>> data = np.genfromtxt("missing_data.txt", delimiter=';',
                         missing_values = ['???'],
                         filling_values = {1:2.2, 2:-3.0, 3:4.9},
                         skip_header = 1,
                         usecols = (1,2,3))
>>> data
array([[ 2.1, -3. ,  5.1],
       [ 2.3, -2.8,  4.8],
       [ 2.2, -3.1,  5. ]])

\end{lstlisting}

\begin{problem}
The file \li{io2002table.txt} contains the raw input output data for the state of Washington in the year 2002.
The last column is the output vector. This is a CSV file with a tab delimiter, so you will set to set your
delimiter to \li{'\\t'}. The top row contains the names of each of the columns, and thus should be omitted when
you import the data. Similarly, the first column contains labels for the rows, and should be omitted. Finally,
there are some missing values in this file, and we assume that they are missing because there is not interaction
between the two corresponding sectors of the economy. Thus, set all missing values to 0. Import the data
to a NumPy array, and calculate the IO coefficient matrix and the demand vector.

Next, find the output vector if the demand for construction increases by ten percent.
Note that column 8 of the original CSV file corresponds to construction.
This will require increasing the correct entry of the initial
demand vector by ten percent and using \li{linalg.solve()} to find the new output vector.
\end{problem}

This method can be used to model economies of almost any scale.
One of the primary limitations is that this model assumes that production varies linearly in its inputs, which may not be realistic.
