\lab{QR Decomposition}{QR decomposition}
\label{lab:QRdecomp}
\objective{Use the Gram-Schmidt algorithm and orthonormal transformations to perform the QR decomposition.}

The QR decomposition of a matrix $A$ is a factorization $A=QR$, where $Q$ has orthonormal columns and $R$ is upper triangular.
This decomposition is useful for computing least squares and finding eigenvalues.
As stated in the following theorem, the QR decomposition of $A$ always exists when the rank of $A$ equals the number of columns of $A$.
\begin{theorem}
Let $A$ be an $m\times n$ matrix of rank $n$.  Then $A$ can be
factored into a product $Q R$, where $Q$ is an $m\times m$ matrix
with orthonormal columns and $R$ is a $m \times n$ upper
triangular matrix of rank $n$.
\end{theorem}

In this lab we will only discuss real matrices. 
All of these results can be extended to the complex numbers by replacing ``orthogonal matrix'' with ``unitary matrix,'' ``transpose'' with ``hermitian conjugate,'' and ``symmetric matrix'' with ``hermitian matrix.''

\section*{Modified Gram-Schmidt}
Let $A$ be an $m \times n$ matrix of rank $n$.
There are many methods for computing the QR decomposition.
First we will discuss the algorithm that computes $Q$ by applying Gram-Schmidt to the columns of $A$.

Let $\{\x_i\}_{i=1}^n$ be the columns of $A$ (the rank hypothesis implies that these are linearly independent vectors).
Then the Gram-Schmidt algorithm computes an orthonormal basis $\{\q_i\}_{i=1}^n$ for the span of the $\x_i$. 
The Gram-Schmidt algorithm defines  \[ \q_1 = \frac{\x_1}{\norm{\x_1}}.\]
It recursively defines $\q_k$ for $k>1$ by
\[
\q_{k} = \frac{\x_k - \p_{k-1}}{\|\x_k - \p_{k-1}\|}, \,\,\,\, k=2,\ldots,n,
\]
where
\[
\p_{k-1} = \sum_{i=1}^{k-1} \langle \q_i, \x_k\rangle \q_i, \,\,\,\, k=2, \ldots, n
\]
and $\p_0 = 0$. 


Let $Q$ be the matrix with columns $\{\q_i\}_{i=1}^n$. 
Let $R$ be the upper triangular matrix with entries $r_{jk}$ where $r_{kk} = \|\x_k-\p_{k-1}\|$ and $r_{j k} = \langle \q_j, \x_k\rangle$ when $j < k$. 
Then $QR=A$ and $R$ is nonsingular (see [ref:textbook] for a proof).
Thus, if $A$ is square, the matrices $Q$ and $R$ are its QR decomposition.


When implemented with a computer, the Gram-Schmidt algorithm may produce a matrix $Q$ with columns that are not even close to orthonormal due to rounding errors. 
We now introduce the modified Gram-Schmidt algorithm, which consistently produces matrices $Q$ whose columns are ``very close'' to orthonormal.

In the modified Gram-Schmidt algorithm, $\q_1$ is the normalization of $\x_1$ as before. 
We then make each of the vectors $\x_2, \ldots, \x_n$ orthogonal to $\q_1$ by defining
\[
\x_k := \x_k - \langle \q_1,\x_{k} \rangle \q_1,\quad k=2,\ldots,n.
\]
(Compare this to the usual Gram-Schmidt algorithm, where we only made $\x_2$ orthogonal to $\q_1$.) 
Next we define $\q_2 = \frac{\x_2}{\|\x_2\|}.$ Once again we make $\x_3, \ldots, \x_n$ orthogonal to $\q_2$ by defining
\[
\x_k := \x_k - \langle \q_2,\x_{k} \rangle \q_2,\quad k=3,\ldots,n.
\]
(Each of these new vectors is a linear combination of vectors orthogonal to $\q_1$, and hence is orthogonal to $\q_1$ as well.) 
We continue this process until we have an orthonormal set $\q_1, \ldots, \q_k$. 
The entire modified Gram-Schmidt algorithm is described in Algorithm \ref{Alg:gram_schmidt}.

%TODO: proofread this, explain notation, and make the LU decomposition alg. like this
\begin{algorithm}
\begin{algorithmic}[1]
\Procedure{Modified Gram-Schmidt}{$A$}
\State $m, n \gets \text{shape} \left( A \right)$
\State $Q \gets \text{copy} \left( A \right)$
\State $R \gets \text{zeros}((n,n))$
\For{$0 \leq i < n$}
    \State $R_{i,i} \gets \norm{Q_{:,i}}$
    \State $Q_{:,i} \gets Q_{:,i}/R_{i,i}$
    \For{$i+1 \leq j < n$}
        \State $R_{i,j} \gets Q_{:,j}^\mathsf{T}Q_{:,i}$
        \State $Q_{:,j} \gets Q_{:,j}-R_{i,j}Q_{:,i}$
	\EndFor
\EndFor
\State \pseudoli{return} $Q, R$
\EndProcedure
\end{algorithmic}
\caption{The modified Gram-Schmidt. This algorithm returns orthogonal $Q$ and upper triangular $R$ such that $A = QR$.}
\label{Alg:gram_schmidt}
\end{algorithm}


%TODO: put this section after the next? What algorithm does SciPy use?
\section*{QR decomposition in SciPy}
The linear algebra library in SciPy calculates the QR decomposition using a software package called LAPACK (Linear Algebra PACKage), which is incredibly efficient.
Here is an example of using SciPy to compute the QR decomposition of a matrix.

\begin{lstlisting}
>>> import numpy as np
>>> from scipy import linalg as la

>>> A = np.random.rand(4,3)
>>> Q, R = la.qr(A)
>>> Q.dot(R) == A                      
array([[ True, False, False],
       [ True, False, False],
       [ True,  True, False],
       [ True,  True, False]], dtype=bool)
\end{lstlisting}
 Note that \li{Q.dot(R)} does not equal \li{A} exactly because of rounding errors. 
 However, we can check that the entries of \li{Q.dot(R)} are ``close'' to the entries of \li{A} with the NumPy method \li{np.allclose()}. 
 This method checks that the elements of two arrays differ by less than a given tolerance, a tolerance specified by two keyword arguments \li{rtol} and \li{atol} that default to $10^{-5}$ and $10^{-8}$ respectively. 
 You can read the documentation to learn how the tolerance is computed from these numbers.
\begin{lstlisting}
>>> np.allclose(Q.dot(R), A) 
True
\end{lstlisting}
We can use the same method to check that \li{Q} is ``very close'' to an orthogonal matrix.
\begin{lstlisting}
>>> np.allclose(Q.T.dot(Q), np.eye(4)) 
True
\end{lstlisting}


\begin{problem}
\label{prob:QR}
Write a function that accepts as input a $m \times n$ matrix of rank $n$ and computes its QR decomposition, returning the matrices $Q$ and $R$. 
Your function should use Algorithm \ref{Alg:gram_schmidt}. 
Hint: Read about the function \li{np.inner()}.
Another hint: Check that your function works by using \li{np.allclose()}.
\end{problem}

\begin{problem}
Write a function that accepts a square matrix $A$ of full rank and returns $\abs{\det(A)}$. 
Use the QR decomposition of $A$ to perform this calculation.
Hint: What is the determinant of an orthonormal matrix?
\end{problem}

\section*{Householder triangularization}
Another way to compute the QR decomposition of a matrix is with a series of orthonormal transformations. 
Like the Modified Gram-Schmidt algorithm, orthonormal transformations are numerically stable, meaning that they are less susceptible to rounding errors.

\subsection*{Householder transformations}
The \emph{hyperplane} in $\mathbb{R}^n$ with normal vector $\mathbf{v}$ is the set $\{ \x \in \mathbb{R}^n \mid \langle \x, \mathbf{v} \rangle = 0 \}$. 
Equivalently, the hyperplane defined by $\mathbf{v}$ is just the orthogonal complement $\mathbf{v}^{\perp}$. 
See Figure \ref{fig:Householder_reflector}.

A \emph{Householder transformation} of $\mathbb{R}^n$ is a linear transformation that reflects about a hyperplane. 
If a hyperplane $H$ has normal vector $\mathbf{v}$, let $\mathbf{u} = \mathbf{v}/\|\mathbf{v}\|$. 
Then the Householder transformation that reflects about $H$ corresponds to the matrix $H_{\mathbf{u}} = I - 2 \mathbf{u}\mathbf{u}^T$. 
You can check that $(I - 2 \mathbf{u}\mathbf{u}^T)^T(I - 2 \mathbf{u}\mathbf{u}^T)=I$, so Householder transformations are orthogonal.

%\newcommand{\ipt}[2]{\ensuremath{\left\langle #1,#2 \right\rangle}}
\begin{figure}
\begin{center}
\begin{tikzpicture}
%\draw[-, thick](0,-1.5)--(0,3); %x-axis
%\draw[-,thick](-1,0)--(4,0); %y-axis
%\node[draw, thick, minimum height=4.5cm, minimum width=5cm]()at(1.5,.75){};
\draw[-,dashed, gray](-2,-1.333)--(3, 2); % hyperplane
\draw[->, gray, >=stealth,ultra thick](0,0)--(.8,-1.2); % v
\draw[->, >=stealth, thick](0,0)--(2.815, .777); % H x
\draw[->, >=stealth, thick](0,0)--(1.8,2.3); % x
\node[draw=none](v)at(.65,-.6){$\v$};
\node[draw=none](x)at(.75,1.5){$\x$};
\node[draw=none](Hx)at(3, .5){$H_{\v}\x$};
\node[draw=none](H)at(-1.5,-.6){$H$};
%\node[draw=none](bullet)at(2.31, 1.53){\textbullet};
%\node[draw=none](vandx)at(4.0,1.5){$\x -
%\left \langle \dfrac{\v}{\|\v\|}, \x \right \rangle \dfrac{\v}{\|\v\|}$};
\end{tikzpicture}
\end{center}
\caption{This is a picture of a Householder transformation. 
The normal vector $\mathbf{v}$ defines the hyperplane $H$. 
The Householder transformation $H_{\mathbf{v}}$ of $\mathbf{x}$ is just the reflection of $\mathbf{x}$ across $H$.}
\label{fig:Householder_reflector}
\end{figure}

\subsection*{Householder triangularization}
The QR decomposition of an $m \times n$ matrix $A$ can also be computed with Householder transformations via the Householder triangularization.
Whereas Gram-Schmidt makes $A$ \emph{orthonormal} using a series of transformations stored in an \emph{upper triangular} matrix, Householder triangularization makes $A$ \emph{triangular} by a series of \emph{orthonormal} transformations.
More precisely, the Householder triangularization finds an $m \times m$ orthogonal matrix $Q$ and and $m \times n$ upper triangular matrix $R$ such that $QA = R$. 
Since $Q$ is orthogonal, $Q^{-1}=Q^T$ so $A = Q^TR$, and we have discovered the QR decomposition.

Let's demonstrate the idea behind Householder triangularization on a $4 \times 3$ matrix $A$.
Let $e_1, \ldots, e_4$ be the standard basis of $\mathbb{R}^4$.
First we find an orthonormal transformation $Q_1$ that maps the first column of A into the span of $e_1$. 
This is diagrammed below, where $*$ represents an arbitrary entry.

\def\mc#1{\multicolumn{1}{c|}{#1}}
\def\lc#1{\multicolumn{1}{|c}{#1}}
\begin{equation*}
\begin{pmatrix}
* & * & * \\
* & * & * \\
* & * & * \\
* & * & *
\end{pmatrix}
\underrightarrow{Q_1}
\begin{pmatrix}

* & * & * & \\ \cline{2-3}
\mc{0} & * & \mc{*}& \\
\mc{0} & * & \mc{*} & \\
\mc{0}& * & \mc{*} & \\ \cline{2-3}
\end{pmatrix}
\end{equation*}
Let $A_2 = Q_1A$ be the matrix on the right above.
Now find an orthonormal transformation $Q_2$ that fixes $e_1$ and maps the second column of $A_2$ into the span of $e_1$ and $e_2$. 
Notice that since $Q_2$ fixes $e_1$, the top row of $A_2$ will be fixed by $Q_2$, and only entries in the boxed submatrix will change.

%\def\mc#1{\multicolumn{1}{c|}{#1}}
%\begin{equation*}
%\begin{pmatrix}
%* & * & * & \\ \cline{2-3}
%\mc{0} & * & \mc{*}& \\
%\mc{0} & * & \mc{*} & \\
%\mc{0}& * & \mc{*} & \\ \cline{2-3}
%\end{pmatrix}
%
%\underrightarrow{Q_1}
%
%\begin{pmatrix}
%* & * & * & \\ \cline{2-3}
%\mc{0} & 0 & \mc{*}& \\
%\mc{0} & 0 & \mc{*} & \\
%\mc{0}& 0 & \mc{*} & \\ \cline{2-3}
%\end{pmatrix}
%\end{equation*}

Let $A_3 = Q_2Q_1A$ be the matrix on the right above. 
Finally, find an orthonormal transformation $Q_3$ that fixes $e_1$ and $e_2$ and maps the third column of $A_3$ into the span of $e_1$, $e_2$, and $e_3$. 
The diagram below summarizes this process, where boxed entries indicate those affected by the operation just performed.

\begin{equation*}
Q_3 Q_2 Q_1
\begin{pmatrix}
* & * & * \\
* & * & * \\
* & * & * \\
* & * & *
\end{pmatrix}
= Q_3 Q_2
\begin{pmatrix}  \cline{2-4}
&\lc{*} & * & \mc{*} & \\
&\lc{0} & * & \mc{*}& \\
&\lc{0} & * & \mc{*} & \\
&\lc{0}& * & \mc{*} & \\ \cline{2-4}
\end{pmatrix}
= Q_3
\begin{pmatrix}
* & * & * \\ \cline{2-3}
\mc{0} & * & \mc{*} \\
\mc{0} & 0 & \mc{*} \\
\mc{0} & 0 & \mc{*} & \\ \cline{2-3}
\end{pmatrix}
=
\begin{pmatrix}
* & * & * \\
0 & * & * \\ \cline{3-3}
0 & \mc{0} & \mc{*} \\
0 & \mc{0} & \mc{0} & \\ \cline{3-3}
\end{pmatrix}.
\end{equation*}

We've accomplished our goal, which was to triangularize $A$ using orthonormal transformations.
But how do we construct the matrices $Q_k$?

It turns out that we can choose each $Q_k$ to be a Householder transformation. 
Suppose we have found Householder transformations $Q_1, \ldots, Q_{k-1}$ such that $Q_{k-1}\ldots Q_2Q_1 = A_k$ where 
\[
A_k = \begin{pmatrix}
T & X' \\
0 & X''
\end{pmatrix}.
\]
Here, $T$ is a $(k-1) \times (k-1)$ upper triangular matrix. 
Let $\x$ be the $k^{th}$ column of $A_k$. 
Write $\x = \x' + \x''$ where $\x'$ is in the span of $e_1, \ldots, e_{k-1}$. 
So $\x''$ looks like $k-1$ zeros followed by the first column of $X''$. 
The idea is to reflect $\x''$ about a hyperplane into the span of $e_k$. 
It turns out that there are two choices that will work (see Figure \ref{fig:two reflectors}). 
These hyperplanes have normal vectors $\x'' + \| \x'' \|e_k$ and $\x'' - \| \x'' \|e_k$.
In fact, the more numerically stable transformation is to reflect about the hyperplane with normal vector $\mathbf{v}_k = \x'' +\sign(x_k) \| \x'' \|e_k$, where $x_k$ is the $k^{th}$ entry of $\x''$ (or the top left entry of $X''$). 
(You can check that $\mathbf{v}_k$ is orthonormal to $e_1, \ldots, e_{k-1}$, so the plane orthonormal to $\mathbf{v}_k$ contains $e_1, \ldots, e_{k-1}$, and reflecting about it fixes $e_1, \ldots, e_{k-1}$.)
Thus, $Q_k$ is the Householder transformation $H_{\mathbf{v}_k}$.

The final question is to find an efficient algorithm for computing $Q = Q_nQ_{n-1} \ldots Q_1$ and $R = Q_nQ_{n-1} \ldots Q_1A$. 
The idea is to start with $R=A$ and $Q = I$. Then we compute $Q_1$ and modify $R$ to be $Q_1A$ and $Q$ to be $Q_1$. 
Next we compute $Q_2$ and modify $R$ to be $Q_2Q_1A$ and $Q$ to be $Q_2Q_1$, and so forth. 
As we have already discussed, $Q_k$ fixes the first $k-1$ rows and columns of any matrix it acts on. 
In fact, if $\x'' = (0, \ldots, 0, x_k, x_{k+1}, \ldots, x_n)$ as above, then $\mathbf{v}_k = (0, \ldots, 0, v_{k_0}, x_{k+1}, \ldots, x_n)$ where $v_{k_0} = x_k + \sign(x_k) \| \x'' \|$. 
If $\mathbf{u}_k$ is the normalization of $(v_{k_0}, x_{k+1}, \ldots, x_n) \in \mathbb{R}^{n-(k-1)}$, then
\[
Q_k = I-\frac{2\x''(\x'')^T}{\|\x''\|^2} =  \begin{pmatrix}
I & 0 \\
0 & I-2\mathbf{u}_k\mathbf{u}_k^T
\end{pmatrix}.
\]
This means that, using block multiplication,
\[
Q_kA_k =  \begin{pmatrix}
I & 0 \\
0 & I-2\mathbf{u}_k\mathbf{u}_k^T
\end{pmatrix}\begin{pmatrix}
T & X' \\
0 & X''
\end{pmatrix} = \begin{pmatrix}
T & X' \\
0 & ( I-2\mathbf{u}_k\mathbf{u}_k^T)X''
\end{pmatrix}.
\]
So at each stage of the algorithm, we only need to update the entries in the bottom right submatrix of $A_k$, and these change via matrix multiplication by $ I-2\mathbf{u}_k\mathbf{u}_k^T$. Similarly,
\[
Q_kQ_{k-1}\ldots Q_1 = Q_k \begin{pmatrix}
A\\
B
\end{pmatrix} = \begin{pmatrix}
I & 0 \\
0 & I-2\mathbf{u}_k\mathbf{u}_k^T
\end{pmatrix}\begin{pmatrix}
A\\
B
\end{pmatrix} = \begin{pmatrix}
A\\
(I-2\mathbf{u}_k\mathbf{u}_k^T)B
\end{pmatrix},
\]
so to update $\prod Q_i$, we need only modify the bottom rows. 
These also change via matrix multiplication by $I-2\mathbf{u}_k\mathbf{u}_k^T$.

These arguments produce Algorithm \ref{Alg:Householder}.

%To find $Q_1$, we first identify an appropriate hyperplane to reflect $x$ into the span of $e_1$.
%It turns out there are two hyperplanes that will work, as shown in figure \ref{fig:two reflectors}.
%(In the complex case, there are infinitely many such hyperplanes.)
%Between the two, the one that reflects $x$ further will be more numerically stable.
%This is the hyperplane perpendicular to $v = sign(x_1)\norm{x}_2 e_1 + x$.
%
%To see how this works, let $x$ be the first column of the submatrix that we want to project onto the span of $e_1$.
%In order for this to be a unitary operation, this will need to preserve the norm of $x$.
%This means that $\left( I - 2 v v^\mathsf{H} \right) x = \pm \norm{x} e_1$, or, in other words,
%
%\[ 2 v v^\mathsf{H} x =
%\begin{pmatrix}
%x_1 \pm \norm{x} \\
%x_2 \\
%x_3 \\
%\vdots \\
%x_n
%\end{pmatrix}\]
%
%Let $u$ be the vector on the right hand side of this expression.
%It can be shown that the vector  $\frac{u}{\norm{u}}$ is the proper choice for $v$.
%We will show that the vector $\frac{u}{\norm{u}}$ is the proper choice for $v$.
%Notice that:
%
%\[\norm{u}^2 = \norm{x}^2 \pm 2 \norm{x} x_1 + x_1^2 + x_2 + \dots + x_n^2 = 2 \norm{x}^2 \pm 2 \norm{x} x_1 \]
%
%and that
%
%\[\norm{x}^2 \pm \norm{x} x_1 = u^\mathsf{H} x \]
%
%So we have
%
%\begin{align*}
%2 v v^\mathsf{H} x &= 2 u \frac{\norm{x}^2 \pm x_1 \norm{x}}{\norm{u}^2} \\
%		&= 2 u \frac{u^\mathsf{H} x}{\norm{u}^2} \\
%		&= 2 \frac{u}{\norm{u}} \left( \frac{u}{\norm{u}} \right)^\mathsf{H} x
%\end{align*}
%
%So $\frac{u}{\norm{u}}$ is a proper choice of $v$ that will project $x$ into the span of $e_1$.

\begin{figure}
\begin{tikzpicture}

\draw[-, dashed, gray](-3,-1)--(3,1);
\draw[-, dashed, gray](-.8,2.4)--(.6,-1.8);
\draw[-, gray, thick](-4,0)--(4,0);
\draw[->, thick, >=stealth'](0,0)--(2,1.6);
\draw[<->, thick, >=stealth'](-2.6,0)--(2.6,0);
\draw[->, gray,  ultra thick, >=stealth](0,0)--(1.5,.5);
\draw[->, gray, ultra thick, >=stealth](0,0)--(-.5,1.5);

\node[draw=none, node distance=3.5cm]
	(dummy)at(2.5,.2){};
\node[draw=none, node distance=.5cm](Hvx)
	[below of=dummy]{$H_{\v_1}\x$};
\node[draw=none, node distance=2cm](x)
	[above left of=Hvx]{$\x$};
\node[draw=none](v1)[left of=x]{$\v_1$};
\node[draw=none, node distance=.55cm](v2)[below of=x]{$\v_2$};
\node[draw=none, node distance=4cm](hvx2)[left of=dummy]{$H_{\v_2}\x$};

\end{tikzpicture}
\caption{If we want to reflect $\x$ about a hyperplane into the span of $e_1$, there are two hyperplanes that will work. 
The two choices are defined by the normal vectors $\v_1$ and $\v_2$. 
Reflecting about the hyperplane defined by $\v_i$ produces  $H_{\v_i}\x$.}
\label{fig:two reflectors}
\end{figure}

%TODO: explain notation, fix algorithm
\begin{algorithm}
\caption{Householder triangularization. 
This algorithm returns orthonormal $Q$ and upper triangular $R$ satisfying $A = QR$.}
\label{Alg:Householder}
\begin{algorithmic}[1]
\Procedure{Householder}{$A$}
\State $m, n \gets \text{shape} \left( A \right)$
\State $R \gets \text{copy} \left( A \right)$
\State $Q \gets I_m$
\For{$0 \leq k < n-1$}
    \State $u_k \gets \text{copy} \left( R_{k:,k} \right)$
    \State $u_{k_0} \gets u_{k_0} + \text{sign} \left( u_{k_0} \right) \norm{u_k}$
    \State $u_k \gets u_k / \norm{u_k}$
    \State $R_{k:,k:} \gets R_{k:,k:} - 2 u_k \left( u_k^\mathsf{T} R_{k:,k:} \right)$
    \State $Q_{k:} \gets Q_{k:} - 2 u_k \left( u_k^\mathsf{T} Q_{k:} \right)$
\EndFor
\State \pseudoli{return} $Q^\mathsf{H}, R$
\EndProcedure
\end{algorithmic}
\end{algorithm}


\begin{problem}
\label{prob:HouseholderQR}
Write a function that accepts as input a $m \times n$ matrix of rank $n$ and computes its QR decomposition, returning the matrices $Q$ and $R$. 
Your function should use Algorithm \ref{Alg:Householder}. 
Hint: Read about the function \li{np.outer()}.
\end{problem}

\begin{comment}
\subsection*{Stability of the Householder QR algorithm}
We will now examine the stability of the Householder QR algorithm.
We will use SciPy's built in QR factorization which uses Householder reflections internally.

Try the following.

\begin{lstlisting}
>>> Q, X = la.qr(np.random.rand(500,500)) # create a random orthonormal matrix:
>>> R = np.triu(np.random.rand(500,500)) # create a random upper triangular matrix
>>> A = np.dot(Q,R) # Q and R are the exact QR decomposition of A
>>> Q1, R1 = la.qr(A) # compute QR decomposition of A
\end{lstlisting}

Observe:

\begin{lstlisting}
>>> la.norm(Q1-Q)/la.norm(Q) # check error in Q
0.282842955725
>>> la.norm(R1-R)/la.norm(R) # check error in R
0.0428922016647
\end{lstlisting}

This is terrible!
This algorithm works in $16$ decimal points of precision, but $Q_1$ and $R_1$ are only accurate to $0$ and $1$ decimal points, respectively.
We've lost $16$ decimal points of precision!

Don't lose hope.
Check how close the product $Q_1 R_1$ is to $A$.
\begin{lstlisting}
>>> A1 = Q1.dot(R1)
>>> np.absolute(A1 - A).max()
3.9968028886505635e-15
\end{lstlisting}
We've now recovered $15$ digits of accuracy.
Considering the error relative to the norm of $A$ (using the 2-norm for matrices), we see that this relative error is even smaller.
\begin{lstlisting}
>>> la.norm(A1 - A, ord=2) / la.norm(A, ord=2)
8.8655568331889288e-16
\end{lstlisting}
The errors in $Q_1$ and $R_1$ were somehow ``correlated," so that they canceled out in the product.
The errors in $Q_1$ and $R_1$ are called \emph{forward errors}.
The error in $A_1$ is the \emph{backward error}.

In fact, the large errors in \li{Q1} and \li{R1} were not because the algorithm was bad, it was because $A$ was poorly conditioned.
The condition number for randomly generated upper triangular matrices is generally very high, and this was the case here.
This has, in turn, made the condition number of $A$ extremely large.

Try the following to compute the condition number of $A$.
In this case the condition number of $A$ and $R$ are computed to be different, though, in theory, they should be exactly the same.
\begin{lstlisting}
>>> from numpy.linalg import cond
>>> cond(A)
4.1426075832870472e+18
>>> cond(R)
3.1767577244363792e+19
\end{lstlisting}

Householder QR factorization is more numerically stable than Gram-Schmidt or even Modified Gram-Schmidt (MGS).
However, MGS is still useful for some types of iterative methods because it finds the orthonormal basis one vector at a time instead of all at once (for an example see Lab \ref{lab:EigSolve}).
\end{comment}

\section*{Upper Hessenberg form}
An upper Hessenberg matrix is a square matrix with zeros below the first subdiagonal.
Every  $n \times n$ matrix $A$ can be written $A = Q^THQ$ where $Q$ is orthonormal and $H$ is an upper Hessenberg matrix, called the Hessenberg form of $A$.

A fast algorithm for computing the QR decomposition of a Hessenberg matrix will be taught in Lab \ref{lab:givens}. This algorithm in turn leads to a fast algorithm for finding eigenvalues of a matrix, which will be discussed in Lab \ref{lab:EigSolve}.

For now, we will outline an algorithm for computing the upper Hessenberg form of any matrix. 
Like Householder triangularization, this algorithm uses Householder transformations.
To find orthogonal $Q$ and upper Hessenberg $H$ such that $A = Q^THQ$, it suffices to find such matrices that satisfy $Q^TAQ=H$. 
Thus, our strategy is to multiply $A$ on the right and left by a series of orthonormal matrices until it is in Hessenberg form.
If we use the same $Q_1$ as in the first step of the Householder algorithm, then with $Q_1 A$ we introduce zeros in the first column of $A$.
However, since we now have to multiply $Q_1 A$ on the left by $Q_1^T$, all those zeros are destroyed.

Instead, let's try choosing a $Q_1$ that fixes $e_1$ and reflects the first column of $A$ into the span of $e_1$ and $e_2$. 
Because $Q_1$ fixes $e_1$, the product $Q_1A$ leaves the first row of $A$ alone, and $(Q_1A)Q_1^T$ leaves the first column of $(Q_1A)$ alone.
If $A$ is a $5 \times 5$ matrix, this looks like
\[
\begin{array}{ccccc}
\begin{pmatrix}
* & * & * & * & * \\
* & * & * & * & * \\
* & * & * & * & * \\
* & * & * & * & * \\
* & * & * & * & *
\end{pmatrix}
&\underrightarrow{Q_1 \cdot }&
\begin{pmatrix}
* & * & * & * & * \\
* & * & * & * & * \\
0 & * & * & * & * \\
0 & * & * & * & * \\
0 & * & * & * & *
\end{pmatrix}
&\underrightarrow{\cdot Q_1^T }&
\begin{pmatrix}
* & * & * & * & * \\
* & * & * & * & * \\
0 & * & * & * & * \\
0 & * & * & * & * \\
0 & * & * & * & *
\end{pmatrix}
\\
A & & Q_1A & & (Q_1 A) Q_1^T
  \end{array}
\]
We now iterate through the matrix until we obtain
\begin{equation*}
Q_3 Q_2 Q_1 A Q_1^T Q_2 ^T Q_3^T =
\begin{pmatrix}
* & * & * & * & * \\
* & * & * & * & * \\
0 & * & * & * & * \\
0 & 0 & * & * & * \\
0 & 0 & 0 & * & *
\end{pmatrix}.
\end{equation*}

The pseudocode for computing the Hessenberg form of a matrix with Householder transformations is shown in Algorithm \ref{Alg:Hessenberg}.
Although the Hessenberg form exists for any square matrix, this algorithm only works for full-rank square matrices.
Notice that this algorithm is very similar to Algorithm \ref{Alg:Householder}.


\begin{algorithm}
\caption{Reduction to Hessenberg form for a nonsingular matrix. 
This algorithm returns orthogonal $Q$ and upper Hessenberg $H$ such that $A = Q^THQ$.}
\label{Alg:Hessenberg}
\begin{algorithmic}[1]
\Procedure{Hessenberg}{$A$}
\State $m, n \gets \text{shape}(A)$
\State $H \gets \text{copy}(A)$
\State $Q \gets I_m$
\For{$0 \leq k < n-2$}
    \State $u_k \gets \text{copy}\left(H_{k+1:, k}\right)$
    \State $u_{k_0} \gets u_{k_0} + \text{sign}(u_{k_0}) \norm{u_k}$
    \State $u_k \gets u_k/\norm{u_k}$
    \State $H_{k+1:,k:} \gets H_{k+1:,k:} - 2u_k(u_k^\mathsf{T} H_{k+1:,k:})$
    \State $H_{:,k+1:} \gets H_{:,k+1:} - 2(H_{:,k+1:} u_k) u_k^\mathsf{T}$
    \State $Q_{k+1:} \gets Q_{k+1:} - 2u_k(u_k^\mathsf{T} Q_{k+1:})$
\EndFor
\State \pseudoli{return} $Q, H$
\EndProcedure
\end{algorithmic}
\end{algorithm}

When $A$ is symmetric, its upper Hessenberg form is a tridiagonal matrix. 
This is because the $Q_i$'s zero out everything below the first subdiagonal of $A$ and the $Q_i^T$'s zero out everything above the first superdiagonal.
Thus, the Hessenberg form of a symmetric matrix is especially useful, since as we saw in Lab \ref{lab:complexity}, tridiagonal matrices make computations fast.




%TODO: This doesn't work if $A$ is singular
\begin{problem}
\label{prob:hessenberg}
Write a function that accepts as input a nonsingular square matrix $A$ and computes its Hessenberg form, returning orthogonal $Q$ and upper Hessenberg $H$ satisfying $A = Q^THQ$. 
Your function should use Algorithm \ref{Alg:hessenberg}. 
What happens when you compute the Hessenberg factorization of a symmetric matrix?
\end{problem}

%Sources: http://www.cs.unc.edu/~krishnas/eigen/node5.html
% http://en.wikipedia.org/wiki/Givens_rotation
%http://en.wikipedia.org/wiki/QR_decomposition
%	Note the Operation count: Householder is 2/3 n^3, MGS is 2 n^3
%http://en.wikipedia.org/wiki/QR_algorithm
%Applied Numerical methods using MATLAB by Yang has some code written for this
%http://www.math.kent.edu/~reichel/courses/intr.num.comp.2/lecture21/evmeth.pdf
%	These are eigenvalue algorithms explained carefully
%http://en.wikipedia.org/wiki/Householder_transformation
%Numerical Linear Algebra, by Lloyd N. Trefethen and David Bau III, Chapters 10 and 16 

