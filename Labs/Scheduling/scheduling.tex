\lab{Applications}{Scheduling}{Scheduling}
\label{lab:Scheduling}
\objective{In this lab, you will determine the schedule that optimizes preferences and reduces spending for nurses in a hospital.}

\section*{Scheduling}

Scheduling is typically an NP-hard combinatorial problem.
It can take many forms, from scheduling teachers and classrooms at a university to scheduling jobs in a queue.
The nurse scheduling problem is highly constrained, which makes it difficult for local search alogrithms to find even a local solution.
Finding optimal solutions is much harder.
We will use a Baysian approach via linear programming to find the optimal solution.


\subsection*{The Problem}
Hospitals employ a lot of nurses around the clock.
Weekly schedules have to satisfy several constraints.
First, each nurse has a certain grade, or level, depending on education, experience and expertise.
At any given time, there must be a minimum number of nurses at each grade.
These numbers depend on how busy the hospital is and what types of procedures they perform.
For example, a hospital in Manhattan will need more higher grade nurses at night than the Utah Valley Regional Medical Center due to the larger population and higher crime rate. 

Second, the working contracts of each nurse is different, with over $400$ possible shift schedules in a two-week pay period.
Most contracts require that the nurse works either all nights or days in a given period, but there are some exceptions for special nurses.
More experienced or specialized nurses have a higher grade, which means that they can take over lower grade shifts, but not vice versa.
Since grades are not independent, one can't schedule the grades separately.
Also complicating the schedule, nurses usually do not work the same number of day shifts as night shifts.  

The third constraint is that schedules must be fair, or appear fair to the nurses.
An obvious aspect is that nurses' requests must be taken into consideration.
If a nurse is stuck with a shift schedule he/she hates, they will be very unhappy and cause problems.
Also included in this is vacation; the hospital still has to meet the requirements when nurses use vacation days.
Since hospitals operate constantly, there are less desirable shifts.
As mentioned above, this can be problematic when these shifts are always assigned to the same nurses.
So the schedule mu evenly distribute unpopular shifts so that nurses don't get angry. 

Finally, the hospital wants to maximize revenue by minimizing the amount of money spent on nurses. 



In order to solve this problem, we'll make some assumptions that will simplify it. 

The first assumption is that nurses are assigned a shift pattern.
Shift patterns can be represented by a $14$-dimensional vector where the first $7$ entries represent the day shifts, and the second $7$ entries represent the nights.
Each nurse is given a set of allowable shift patterns based off their contract.
For example, 
\begin{center}
$(1,1,1,1,0,0,0,0,0,0,0,0,0,0)$
\end{center}
is an allowable shift pattern, which indicates that the nurse works $4$ consecutive days and then has $3$ days off. 
\begin{center}
$(1,1,0,0,0,0,0,1,1,0,0,0,0,0)$
\end{center}
is not allowed because the nurse would work for over $30$ hours straight.

The second assumption involves taking into account the nurses' preferences.
This can be tricky since we have to have some way to measure and compare preferences.
We will not worry about that here. 

For each nurse and their allowable shift patterns, we assign a preference.

\subsection*{Linear Program}
How can we formulate this as an optimization problem?

The preference of a nurse $i$ working a given shift pattern $j$ takes into account the nurse's contract and the fairness aspect by assigning a number to that nurse a pattern.
The number can be seen as the cost for assigning nurse i shift pattern j. 
Thus a high cost indicates that the nurse should probably not work that pattern.
In creating the schedule, the goal is then to minimize the cost of possible schedules. 


We have data for $30$ nurses, $70$ possible shift schedules, and $2$ grades.
$3$ of the nurses are grade $1$ and the rest are grade $3$. 

We can formulate the problem as an integer program in the following manner.


Indices:
\begin{itemize}
\item i = 1...n = 30, nurse index
\item j = 1...m = 70, shift pattern index
\item k = 1...14, day and night index where 1-7 are days and 8-14 are nights
\item s = 1...p = 3, grade index 
\end{itemize}

Decision Variables:
\begin{displaymath}
   x_{ij} = \left\{
     \begin{array}{lr}
       1, & \text{nurse i works pattern j}\\
       0, & \text{else}
     \end{array}
   \right.
\end{displaymath}



Parameters:
\begin{itemize}
\item n = Number of nurses
\item m = Number of shift patterns
\item p = Number of grades
\item $p_{ij}$ = Preference cost of nurse i working pattern j
\item $R_{ks}$ = Demand of nurses with grade s on day/night k
\item F(i) = Set of feasible work patterns for nurse i
\item \begin{displaymath}
%\begin{flalign*}
   a_{jk} = \left\{
     \begin{array}{lr}
       1, & \text{shift pattern j covers day/night k}\\
       0, & \text{else}
     \end{array}
   \right.
\end{displaymath}
\item \begin{displaymath}
   q_{is} = \left\{
     \begin{array}{lr}
       1, & \text{nurse i is of grade s or higher}\\
       0, & \text{else}
     \end{array}
   \right.
\end{displaymath}
%\end{flalign*}
\end{itemize}



Now set up the integer program that describes this problem.


Objective Function:

Minimize the preference cost for all nurses
\begin{center}
min $\displaystyle\sum_{i=1}^{n} \displaystyle\sum_{j\in F(i)}^{m} p_{ij}x_{ij}$
\end{center}


Constraints:

1) Each nurse works exactly one feasible shift pattern
\begin{center}
$\displaystyle\sum_{j\in F(i)} x_{ij} = 1 \forall i$
\end{center}

2) The demand of nurses for each grade is met for every shift
\begin{center}
$\displaystyle\sum_{j\in F(i)} \displaystyle\sum_{i=1}^n q_{is}a_{jk}x_{ij} \geq R_{ks} \forall k,s$
\end{center}


\textbf{Problem}
Solve this optimization problem using cvxopt.glpk.ilp and the data included in this lab.
Assume that $F(i)$ consists of all possible $70$ shifts for each nurse $i$. 

Hints: 

Remember that all data must be a float. 
$G$ should be a $4242$ by $2100$ matrix.
$A$ should be a $30$ by $2100$ matrix.


\subsection*{The Data}
Preferences.csv contains the level of each nurse, as well as their shift preferences.
The $30$ nurses are represented by rows.
The first column is their level, an integer between $1$ and $3$.
The next $70$ columns represent each nurse's preference for that shift with an integer between $0$ and $50$.
The higher the number, the less desirable that shift is. 

SchedulingData.csv is the list of $70$ shift schedules.
Each row is an allowable shift schedule and the $14$ columns represent the $7$ days and $7$ nights.
A $1$ indicates a working shift, while a $0$ represents an off shift, just like in the examples above.

Demands.csv is the list of the number of required nurses for each level.
It has one column with $42$ rows.
The first $14$ rows represent the number of level $1$ nurses for each of the $14$ shifts, the second $14$ rows represent the level $2$ nurses, and the last rows are the number of level $3$ nurses. 

All of these files are comma delimited.

%This solution doesn't necessarily minimize the payout to the nurses, so we need to adapt the problem above. For example, it may be more effective for a nurse to work overtime instead of hiring a new nurse. ? other example
%Each nurse is assigned a salary or wage. Calculate the new schedule to minimize cost.
%Does it differ from part 1?