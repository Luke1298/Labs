\lab{Public Key Cryptography}{RSA}
\label{lab:RSA}

\objective{Implement the RSA cryptosystem as an example of public key cryptography.}

A \emph{public key cryptosystem} uses separate keys for encryption and decryption.
If Alice wishes to send Bob a message, she encrypts it using Bob's \emph{public key}, which is available to everyone.
Then Bob decrypts it with his \emph{private key}, which only he knows.
As long as it is difficult to find the private key from the public key, this is a secure system.

%Public key cryptosystems are advantageous because for $n$ people to have secure communications with each other, they need exactly $n$ public-private key pairs.
%A symmetric cryptosystem would require on the order of $n^2$ keys.
%Also, they eliminate the need for secure key sharing algorithms like the Diffie-Hellman key exchange (see Lab \ref{lab:DiffieHellman}).

As an analogy, consider a locked box with two kinds of keys.
One kind of key, called the public key, is available to anyone and can lock the box but not unlock it.
The other kind of key, called the private key, is only available to one person and can unlock the box.
To send a message to someone with the private key, the sender will put the message in the box and lock it.
Since the only person that can unlock the box is the recipient, the message is safe until it arrives. This is how public key cryptography works.

One of the oldest and most popular public key cryptosystems is RSA cryptography.

%All material in this lab through the section ``Security of RSA'' is required. 
%After that, all sections and subsections are independent of each other and may be covered as desired.

\section*{The RSA system}
Suppose Alice wants to receive secret messages using RSA.
To do so, she first needs to generate a public key (for encryption) and a private key (for decryption).
Alice does this by choosing two prime numbers, $p$ and $q$, and setting $n=pq$.
Then she sets $\phi(n) = (p-1)(q-1)$.\footnote{
The function $\phi: \mathbb{Z} \rightarrow \mathbb{N}$ is called the \emph{Euler phi function}. In general $\phi(n)$ is the number of positive integers less than $n$ that are relatively prime to $n$.}
Alice chooses her encryption exponent to be some integer $e$ that is relatively prime to $\phi(n)$.
Then she uses the Extended Euclidean Algorithm to find $d'$ such that $ed' + \phi(n)x' = 1$, and adds or subtracts multiples of $\phi(n)$ until $d = d'+k\phi(n)$ is between 0 and $\phi(n)$.
Alice publishes her public key $(e, n)$ for the world to see and keeps her private key $(d,n)$ a secret ($e$ for encryption, $d$ for decryption).

Now imagine Bob wants to send Alice a message, which he represents as an integer $m<n$ (say, using the A=01, B=02 scheme).
Bob computes 
\[
c \equiv m^e \pmod{n}
\]
and sends the ciphertext $c$ to Alice.

To decrypt the message, Alice computes
\[
m' \equiv c^d \pmod{n}.
\]
At this point, she uses the following theorem, which is easily proved with a combination of ring and group theory.
\begin{theorem}
For any integers $m$ and $n$ such that $n$ does not divide $m$, the following equality holds:
\[
m^{\phi(n)}\equiv 1 \pmod{n}.
\]
\end{theorem}
Then Alice concludes that
\[
m' = c^d \equiv m^{ed} = m^{\phi(n)(ek-x')+1} \equiv m \pmod{n}.
\]
Now Alice can read Bob's message.

As an example, let us encrypt and decrypt the message SECRET=190503180520.
First we define $p$, $q$, $n$, and $\phi(n)$.
\begin{lstlisting}
>>> p = 1000003
>>> q = 608609
>>> n = p*q
>>> phi_n = (p-1)*(q-1)
\end{lstlisting}

Now we choose an encryption exponent $e = 1234567891$ and compute $d = 589492582555$ using the Extended Euclidean Algorithm.
\begin{lstlisting}
>>> e = 1234567891
>>> d = 589492582555
\end{lstlisting}

Finally we are ready to encrypt the message. 
Note that $m<n$. If this were not the case, we would need to break up $m$ into shorter pieces.
Also, we force $m$ to be a \li{long} integer so that the exponentiation operation does not overflow.
The function \li{pow(a, b, n)} computes $a^b \pmod{n}$.
\begin{lstlisting}
>>> m = long(190503180520)
>>> c = pow(m, e, n)
\end{lstlisting}

We decrypt the message by raising $c$ to the $d^{th}$ power, modulo $n$.
\begin{lstlisting}
>>> m == pow(c, d, n)
True
\end{lstlisting}

\subsection*{Logistical considerations}
The cryptosystem described is not particularly easy to use, since the message must be converted to an integer and back again by hand.
The provided \li{rsa_tools} module contains some functions to fix this problem.
The function \li{string_to_int()} turns any string into an integer (using a mapping more complicated than $A=01, B=02$), and the function \li{int_to_string()} changes it back again.

\begin{lstlisting}
>>> import rsa_tools as r
>>> r.string_to_int('SECRET')
91556947314004
>>> r.int_to_string(91556947314004)
'SECRET'
\end{lstlisting}

At this point we have a problem, because the message 91556947314004 is larger than $n=608610825827$.
We can only use RSA to encrypt messages that are smaller than $n$.
In fact, the function \li{string_size()} provided in \li{rsa_tools} tells us the maximum number of characters we can encrypt with this choice of $n$.

\begin{lstlisting}
>>> r.string_size(608610825827)
4
\end{lstlisting}

The function \li{partition()}, also in the \li{rsa_tools} will break our message into pieces of length 4.
We specify the ``fill value'' \li{x} that will be used to make all pieces the same length.

\begin{lstlisting}
>>> r.partition('SECRET', 4, 'x')
['SECR', 'ETxx']
\end{lstlisting}
Now we can proceed to encrypt and decrypt the strings \li{'SECR'} and \li{'ETxx'} as before.

% Problem 1: Implement the 'myRSA' class.
\begin{problem}
Write a class called \li{myRSA} that can generate keys, encrypt messages, and decrypt messages.
Use methods from the provided \li{rsa_tools.py} module to convert strings to ints and vice-versa.

Write a method called \li{generate_keys} that accepts a pair of primes and an encryption exponent and sets the \li{public_key} and \li{_private_key} attributes.
(Starting \li{_private_key} with an underscore hides the attribute from the user.)
Also include an \li{encrypt} method that accepts a string and encrypts it, using \li{public_key} as the default encryption key.
If a different public key is provided, then the message should be encrypted with the provided key.
Finally, write a \li{decrypt} method that decrypts an encrypted message with \li{_private_key}.

In the next problem we will write a test function for this class.
\label{prob:rsa1}
\end{problem}

\subsection*{Security of RSA}
Suppose an enemy Eve wants to read the message that Bob sent to Alice.
Like Bob, she has access to Alice's public key $(e, n)$.
Let's assume she has also intercepted the ciphertext $c$.

One way for Eve to read Bob's message is to directly find $m$ such that $m^e \equiv c \pmod{n}$.
Such a computation is known as taking a \emph{discrete logarithm}.
When $n$ is very large, this computation is essentially impossible.

Another option is for Eve to compute $(d, n)$ and then find $c^d$.
However, computing $d$ means computing $\phi(n)$.
Computing $\phi(n)$ from $n$ directly requires factoring $n$.
When $n$ has hundreds of digits, finding its factors with known algorithms can take years.

Thus, the security of RSA depends on selecting large primes so that $n$ has many digits.

\section*{Exceptions}

Every programming language has a formal way of handling errors. In Python, we raise and handle \emph{Exceptions}. There are different kinds of exceptions, each with its appropriate usage.
In Python, as in most languages, exceptions are organized into a class hierarchy.
A complete list of Python exceptions can be found \href{https://docs.python.org/2/library/exceptions.html}{here}.
The following code displays some examples of common exceptions:

\begin{lstlisting}
# A 'NameError' exception indicates that a nonexistant name was used.
>>> print(x)
Traceback (most recent call last):
  File "<stdin>", line 1, in <module>
NameError: name 'x' is not defined

# An 'AttributeError' exception indicates that a nonexistant method or
# attribute was called on some object.
>>> x = list()
>>> x.fly()
Traceback (most recent call last):
  File "<stdin>", line 1, in <module>
AttributeError: 'list' object has no attribute 'fly'

# A 'SyntaxError' exception indicates bad coding syntax.
>>> def myFunction(a, b)
  File "<stdin>", line 1
    def myFunction(a, b)
                       ^
SyntaxError: invalid syntax
\end{lstlisting}

Many exceptions, like the ones listed above, are due to coding mistakes and typos.
Exceptions can also be used programmatically to indicate a problem to the user.
To raise an exception, use the keyword \li{raise}, followed by the name of the exception.
As soon as an exception is raised, the program stops running unless the exception is handled.

\begin{lstlisting}
# raise a specific type of exception, with an error message included.
>>> x = 7
>>> if x > 5:
...     raise ValueError("'x' should not exceed 5.")
... 
Traceback (most recent call last):
  File "<stdin>", line 2, in <module>
ValueError: 'x' should not exceed 5.
\end{lstlisting}

\subsection*{Handling Exceptions}

To prevent an exception from stopping the program, it must be handled by
putting any lines of code that throw the exception in a \li{try}.
An \li{except} block then follows.
If an exception is thrown, the compiler exits the \li{try} block and enters the \li{except} block.

\begin{lstlisting}
# the 'try' block should hold any lines of code that might raise the exception
>>> try:
...     raise Exception
...     print("No exception raised")
... # the 'except' block is entered into after the exception is thrown
... except:
...     print("Exception raised!")
... 
Exception raised!

# Specific types of exceptions can also be caught explicitly
>>> try:
...     bad = 0 / 0
... except ZeroDivisionError:
...     print("Divided by zero!")
... 
Divided by zero!

# Finally, the exception object can be used to get information.
>>> try:
...     import magic
... except ImportError as e:
...     print("Sorry! " + e.message)
... 
Sorry! No module named magic
\end{lstlisting}

Documentation on handling exceptions can be found \href{https://docs.python.org/2/tutorial/errors.html}{here}.

% Problem 2: Exceptions and testing my_RSA.
\begin{problem}
Write a \li{test_myRSA} function that accepts a message, two primes, and an encryption exponent.
If the first argument is not a string, raise a \li{TypeError} with error message ``message must be a string.''
If the final three arguments are not integers, raise a \li{TypeError} with error message ``p, q, and e must be integers.''
(Hint: use the built-in \li{type} function.)

Create an instance of the \li{myRSA} class.
Encrypt and decrypt the message using the keys generated by the integer arguments.
If the original message and the decryption are not exacly the same, raise a \li{ValueError} with error message ``decrypt(encrypt(message)) failed.''
Otherwise, return \li{True}.
\end{problem}

\section*{Making RSA Secure}

Since the security of RSA depends on the use of large prime numbers for $p$ and $q$, we need a fast way to find such numbers for RSA to be a practical cryptosystem.
It is very slow to check that a large number is prime by finding its factors.
In fact, it is the difficulty of factoring that makes RSA secure.

However, there exist fast algorithms that determine when a number is ``probably prime.''
One such algorithm comes from a special case of Theorem \ref{thm:fermat} known as \emph{Fermat's Little Theorem}:
\[
a^{p-1} \equiv 1 \pmod{p}
\]
for all primes $p$ and integers $a$ that are not divisible by $p$.
Therefore, to check if a fixed number $p$ is likely a prime, we can compute $a^{p-1} \pmod{p}$ for several values of $a$.
If we ever get something different than 1, then we know that $p$ is composite.
The value of $a$ that proved $p$ was composite is called a \emph{witness number}.
In fact, the converse is true: if $a^{p-1} \equiv 1 \pmod{p}$ for every $a<p$ then $p$ is prime.

With a few exceptions, if $p$ is composite then more than half the integers in $[2, p-1]$ are witness numbers.
Thus, if we run the above test on $p$ just a few times and don't find a witness number, there is a high probability that $p$ is prime.
This test is called Fermat's test for primality.

In practice, a probabilistic algorithm like Fermat's test is used to identify numbers that have a high chance of being prime. 
Then, deterministic algorithms are used to verify the primality of a candidate.

% Problem 3: Implement Fermat's test for primality.
\begin{problem}
Write a function called \li{is_prime} that implements Fermat's test for primality.
Run the test at most five times, using integers randomly chosen from $[2, n-1]$ as possible witnesses.
If a witness number is found, return the number of tries it took to find it.
If no witness number is found after five tries, return $0$.
(Hint: use the built-in \li{pow} function so the exponentiation operation does not overflow.)

For most composite values of $n$, if you call \li{is_prime(n)} one hundred times, you should expect to get $0$ at most five times.
However, some composite numbers do not follow this rule. 
For example, how many times do you have to call \li{is_prime()} on $340561$ to get an answer of $0$?
\label{prob:prime_confidence}
\end{problem}

\section*{PyCrypto: RSA in Python}
% A professional implementation of RSA is found in the PyCrypto module.

PyCrypto is a professional implementation of RSA in Python.
The package is called \li{Crypto}, and is included with most Python distributions.
It can be downloaded separately \href{https://pypi.python.org/pypi/pycrypto}{here}.
Documentation for PyCrypto can be found \href{https://www.dlitz.net/software/pycrypto/api/current/}{here}.
This library contains many random number generators and encryption classes.
Many programs use PyCrypto for their security needs.

\begin{warn}
Make certain that you are using the latest version of PyCrypto.
Security software is updated often to fix security vulnerablilities and bugs.
The current version of PyCrypto at the time of writing is version 2.6.1.
\end{warn}

The RSA module in PyCrypto is located in the PublicKey module.
The library allows us to explicitly construct a key, or generate a key automatically.
\begin{lstlisting}
>>> from Crypto.PublicKey import RSA

# generate a 2048-bit RSA key
>>> keypair = RSA.generate(2048)

# Save the public key as a string for distribution.
>>> public_key = keypair.publickey()
>>> share_this = public_key.exportKey()
\end{lstlisting}

The \li{share_this} variable may be distributed as a public key.

\begin{lstlisting}
>>> encrypter = RSA.importKey(share_this)
>>> ciphertext = encrypter.encrypt('abcde', 2048)
>>> ciphertext
('\xb1\t\xc6L\xd1u\x80.C@\x07$#\x8e\xca\x8a\x05*\xdf\x1f.N\xa9\x80
\x08\xcb*8~7\x1d\x87&&Ke\xd5\xed_H\xb9\xd0x\xac!\xf3\xa9\xdc\xbfy5
s\x92\x8d\x15\xf7vY\x99G\xb6\x03j[\xa3\xc6\x92a\n\x91\x08N\xbc\xe4
\xcd\xe2\x9b\xeb\x1eT\xe5\xef\x96\x83\x10\xb7\x0c\xd1\x9bK]z\xa5!\
xcc\xe0/\xd3L\xd4\xa9xx?*\xfb\xf6\xcaM8\xe6\x9d\xd4u\xbd\xda\xa8tf
X\x02\xfa\xff\x99\xff\xbb"\xd1\x87\'\xb9d\x1c\x1b\x9fcWd\x83\xea}\
x1f\xff\xd3\x9b0\x8e\x0f\x91\n\x16\r\r\xa5\xa5S\xafw\x07N`$\x9c]\x
ac\x96\xe3\x80l\xc9\xe5\xe4Od\xa5\t>\x16j\xa1\xb9\x9c5\xc0\xfe\xe3
\xe5i\xcd\xaf\xdc\xcad\x82\x10u\x91\xa0"\xcf\xe3A\x11\x82\x87\xb9\
xdf\xd7\x86\x87\xff\x11#-\xb5!Q)\xf7\xf1a\x94\xb3e?\xd0\x96W4\xb4\
xca\xcf\x18\xd1I\xcd\x1dl\xd7\xe0Y\xdf}F\x93\x92\xe1(d\xcc\xdc\xa7
 x\xc5`',)
>>> keypair.decrypt(ciphertext)
'abcde'
\end{lstlisting}

\begin{comment}

The RSA encryption and decryption methods on these keys are textbook approaches.
However, to increase security, we will want to pad the messages so every message encrypted with a particular key will become exactly as large (in bits) as the key itself.
A commonly used padding algorithm is implemented in PyCrypto in the \li{Crypto.Cipher.PKCS1_OAEP} module.
\begin{lstlisting}
>>> from Crypto.Cipher import PKCS1_OAEP as oaep

# generate a new key from the original RSA key.
# This key can encrypt and decrypt
>>> paddedkey = oaep.new(keypair)
>>> encrypted = paddedkey.encrypt('hello world')
>>> paddedkey.decrypt(encrypted)
'hello world'
\end{lstlisting}
\end{comment}

% Problem 4: Implement a PyCrypto-powered RSA system.
\begin{problem}
Write a new RSA class called \li{PyCrypto} that acts like the \li{myRSA} class, but implement it with methods from PyCrypto's \li{RSA} package.

Store both of your RSA keys in a single attribute called \li{_keypair}, and store a sharable string representation of the public key in the \li{public_key} attribute.
Initialize \li{_keypair} and \li{public_key} in the constructor.
The \li{encrypt} method accepts a string and encrypts it, using \li{_keypair} as the default encryption key.
If the string representation of a different public key is provided, then the message should be encrypted with the provided key.
The \li{decrypt} method decrypts an encrypted message using \li{_keypair}.

Try testing your class with a partner by exchanging public keys.
\end{problem}

\begin{warn}
The following warning comes from the PyCrypto \href{https://www.dlitz.net/software/pycrypto/}{website}:

``The export of cryptography software is (still) governed by arms control regulations in Canada, the United States, and elsewhere.
The export or re-export of this software may be regulated by law in your country.''
\end{warn}

% 	================ END OF LAB ================	%

% Group project problem from the old lab.
% Assumes we did a problem on digital signatures.
\begin{comment}
\begin{problem}[Group Project]
Split into several groups of at least three individuals.
Each group member should generate a private and public keypair using PyCrypto.
Share your public keys with everyone in the group.

You might find it useful to transmit your messages as strings.
Here are two functions that will read and write specially formatted strings for your messages.
\begin{lstlisting}
def print_msg(m_encrypted):
    #print the encrypted message as a specially formatted string
    out = ("----BEGIN MESSAGE----",
           m_encrypted[0],
           "----END MESSAGE----",
           "----BEGIN SIGNATURE----",
           m_encrypted[1],
           "----END SIGNATURE----")
    return '\n'.join(out)

def read_msg(m_string):
    d1 = m_string.find('----BEGIN MESSAGE----') + 21
    d2 = m_string.find('----END MESSAGE----', d1)
    
    d3 = m_string.find('----BEGIN SIGNATURE----', d2) + 23
    d4 = m_string.find('----END SIGNATURE----', d3)
    message  = m_string[d1:d2]
    sign = m_string[d3:d4]
    return message.strip(), sign.strip()
\end{lstlisting}

\begin{enumerate}
\item Encrypt and sign a message for someone in the group and send it to the entire group.
Attach the sender's name to the encrypted message (a claim that the message came from a particular sender).
In this exercise we will verify that the actual origin of the message is indeed the claimed origin.
Each group member should do the following:
\begin{enumerate}
\item Decrypt and verify the message addressed for you.
\item Attempt to decrypt and verify message addressed for someone else in the group.
\item Report the message and verified signature.
\end{enumerate}

\item Encrypt and sign a different message for someone in the group and send it, anonymously, to the entire group.
For example, if Bob encrypts a message and sends it to Alice, he should claim the message came from another individual.
Each group member should do the following:
\begin{enumerate}
\item Decrypt and verify the message address to you.
\item Report the message and verified signature.
\end{enumerate}
\end{enumerate}
\end{problem}
 
\end{comment}

