\lab{Public Key Cryptography}{RSA}
\label{lab:RSA}

\objective{Implement the RSA cryptosystem as an example of public key cryptography.}

A \emph{public key cryptosystem} uses separate keys for encryption and decryption.
If Alice wishes to send Bob a message, she encrypts it using Bob's \emph{public key}, which is available to everyone.
Then Bob decrypts it with his \emph{private key}, which only he knows.
As long as it is difficult to find the private key from the public key, this is a secure system.

Public key cryptosystems are advantageous because for $n$ people to have secure communications with each other, they need exactly $n$ public-private key pairs.
A symmetric cryptosystem would require on the order of $n^2$ keys.
Also, they eliminate the need for secure key sharing algorithms like the Diffie-Hellman key exchange (see Lab \ref{lab:DiffieHellman}).

One of the oldest and most popular public key cryptosystems is RSA cryptography.\\

All material in this lab through the section ``Security of RSA'' is required. 
After that, all sections and subsections are independent of each other and may be covered as desired.

\section*{The RSA system}
Suppose Alice wants to receive secret messages using RSA.
To do so, she first needs to generate a public key (for encryption) and a private key (for decryption).
Alice does this by choosing two prime numbers, $p$ and $q$, and setting $n=pq$.
Then she sets $\phi(n) = (p-1)(q-1)$.\footnote{
The function $\phi: \mathbb{Z} \rightarrow \mathbb{N}$ is called the \emph{Euler phi function}. In general $\phi(n)$ is the number of positive integers less than $n$ that are relatively prime to $n$.}
Alice chooses her encryption exponent to be some integer $e$ that is relatively prime to $\phi(n)$.
Then she uses the Extended Euclidean Algorithm to find $d'$ such that $ed' + \phi(n)x' = 1$, and adds or subtracts multiples of $\phi(n)$ until $d = d'+k\phi(n)$ is between 0 and $\phi(n)$.
Alice publishes her public key $(e, n)$ for the world to see and keeps her private key $(d,n)$ a secret.

Now imagine Bob wants to send Alice a message, which he represents as an integer $m<n$ (say, using the A=01, B=02 scheme).
Bob computes 
\[
c \equiv m^e \pmod{n}
\]
and sends the ciphertext $c$ to Alice.

To decript the message, Alice computes
\[
m' \equiv c^d \pmod{n}.
\]
At this point, she uses the following theorem, which is easily proved with a combination of ring and group theory.
\begin{theorem}
For any integers $m$ and $n$ such that $n$ does not divide $m$, the following equality holds:
\[
m^{\phi(n)}\equiv 1 \pmod{n}.
\]
\end{theorem}
Then Alice concludes that
\[
m' = c^d \equiv m^{ed} = m^{\phi(n)(ek-x')+1} \equiv m \pmod{n}.
\]
Now Alice can read Bob's message.

As an example, let us encrypt and decrypt the message SECRET=190503180520.
First we define $p$, $q$, $n$, and $\phi(n)$.
\begin{lstlisting}
>>> p = 1000003
>>> q = 608609
>>> n = p*q
>>> phi_n = (p-1)*(q-1)
\end{lstlisting}

Now we choose an encryption exponent $e = 1234567891$ and compute $d = 589492582555$ using the Extended Euclidean Algorithm.
\begin{lstlisting}
>>> e = 1234567891
>>> d = 589492582555
\end{lstlisting}

Finally we are ready to encrypt the message. 
Note that $m<n$. If this were not the case, we would need to break up $m$ into shorter pieces.
Also, we force $m$ to be a \li{long} integer so that the exponentiation operation does not overflow.
The function \li{pow(a, b, n)} computes $a^b \pmod{n}$.
\begin{lstlisting}
>>> m = long(190503180520)
>>> c = pow(m, e, n)
\end{lstlisting}

We decript the message by raising $c$ to the $d^{th}$ power.
\begin{lstlisting}
>>> m == pow(c, d, n)
True
\end{lstlisting}

\begin{problem}
Implement the RSA cryptosystem with three functions: one to generate keys, one to encrypt messages, and one to decrypt messages.
\begin{lstlisting}
def keys(p, q, e):
   '''Create a pair of RSA keys from primes `p' and `q' and encryption 
   exponent `e'.
   
   Return [(e, n), (d, n)] where (e, n) is the public key and (d, n) is 
   the private key.
   '''
    
def encrypt(message, e, n):
    '''Encrypt the `message' using the public key (e, n).'''
    
def decrypt(ciphertext):
    '''Decrypt the `ciphertext' using your private key.'''
\end{lstlisting}
\label{prob:rsa1}
\end{problem}
\subsection*{Security of RSA}
Suppose an enemy Eve wants to read the message that Bob sent to Alice.
Like Bob, she has access to Alice's public key $(e, n)$.
From this, she needs to find $(d, n)$.

However, computing $\phi(n)$ from $n$ directly requires factoring $n$.
There is no known algorithm for fast factorization. 
When $n$ has hundreds of digits, finding its factors with known algorithms can take years.



\subsection*{Logistical considerations (Optional)}
The cryptosystem you wrote in Problem \ref{prob:rsa1} is not particularly easy to use, since you must convert your message to an integer and back again by hand.
The appendix to this lab contains some functions you can use to fix this problem.
The function \li{string_to_int()} turns any string into an integer (using a mapping more complicated than $A=01, B=02$), and the function \li{int_to_string()} changes it back again.

\begin{lstlisting}
>>> string_to_int('SECRET')
91556947314004
\end{lstlisting}

At this point we have a problem, because the message 91556947314004 is larger than $n=608610825827$.
We can only use RSA to encrypt messages that are smaller than $n.$
In fact, the function \li{string_size()} provided in the appendix tells us the maximum number of characters we can encrypt with this choice of $n$.

\begin{lstlisting}
>>> string_size(608610825827)
4
\end{lstlisting}

The function \li{partition()}, also in the appendix, will break our message into pieces of length 4.
We specify the ``fill value'' \li{x} that will be used to make all pieces the same length.

\begin{lstlisting}
>>> partition('SECRET', 4, 'x')
['SECR', 'ETxx']
\end{lstlisting}
Now we can proceed to encrypt and decrypt the strings \li{'SECR'} and \li{'ETxx'} as before.

\begin{problem}[Optional]
Modify your solution to Problem \ref{prob:rsa1}, making it simpler to use, by writing the following functions.
\begin{lstlisting}
def encrypt(message, e, n):
    '''Encrypt `message' using the public key (e, n).
    
    INPUTS:
    message - A string.
    (e, n)  - A pair of integers. Should be an RSA public key.
    
    OUTPUTS:
    Partition `message' into pieces of sufficiently small size, padding if 
    necessary. Return a list of the encryptions of the pieces.
    '''
    
def decrypt(ciphertext):
    '''Decrypt `ciphertext' using your private key.
    
    INPUTS:
    ciphertext - A list of integers.
    
    OUTPUTS:
    Decrypt each integer in `cipertext' and return a single string.
    '''
\end{lstlisting}

You can test your cryptosystem on the example in this lab and also on this larger example:
\begin{itemize}
\item message: \emph{Simon sells seashells by the seashore.}
\item $p=83285677$, $q=2848968679$, $e=65593$
\end{itemize}
\label{rsa2}
\end{problem}

\section*{Digital Signatures (Optional)}
Suppose Bob wants Alice to be able to verify that \emph{he} sent her a message, not Eve.
Bob can do this using a \emph{digital signature}.

\subsection*{Method 1}
One way for Bob to digitally sign a message is to apply his \emph{private key} (his ``signature'') to the message and then apply Alice's public key.
When Bob applies his private key to a message, we say he \emph{signs} the message.
When Alice receives this message, she first applies her private key and then Bob's public key. 
If the message makes sense, she knows it must be from Bob, since he is the only person with his private key.

Symbolically, let $e_B()$ and $d_B()$ be Bob's encryption and decryption functions (these are inverse functions), and let $e_A()$ and $d_A()$ be the corresponding functions for Alice.
Then Bob sends $e_A(d_B(m))$ to Alice, and Alice computes 
\[e_B(d_A(e_A(d_B(m)))) = e_B(d_B(m)) = m.\]

This method requires twice as many RSA encryptions as sending an unsigned message.
Modular exponentiation can be an expensive operation, so if the message is long (and must be broken into several pieces) this is a serious slow-down.


\subsection*{Method 2: hashes}
A more efficient method (which also turns out to be more cryptographically secure) is to ``hash the message, and then sign the hash.''
A hash is a function that takes a string of any size to a fixed number of bytes.
A cryptographically secure hash satisfies two properties:
\begin{itemize}
\item If $m_1$ is different from $m_2$ then hash($m1$) is (almost certainly) different from hash($m2$).
\item Given hash($m$), it is essentially impossible to find $m$.
\end{itemize}
For example, the MD5 hash will map any string to 16 bytes (a character like \li{'a'} is one byte of data).\footnote{
The hash MD5 is not cryptographically secure, but is often used in non-cryptographic applications. 
Because of its small digest size we use it as an example here.}
Signing the hash will be faster than signing the original message, if the original message is more than 16 characters.

In Python, hashes are implemented in the \li{hashlib} library. 
Here we use MD5 to hash the string \li{'SECRET'}.
\begin{lstlisting}
>>> import hashlib
>>> hash = hashlib.md5('SECRET')
\end{lstlisting}
We view the hash with the \li{digest()} method.
\begin{lstlisting}
>>> hash.digest()
'D\xc7\xbeH"n\xba\xd5\xdc\xa8!ft\xca\xd6+'
\end{lstlisting}
The output is 16 bytes, some represented as characters (e.g. \li{'D'}) and some as two hexidecimal digits (e.g. \li{'\\xc7'}).
Here we separate the digest into its bytes.
\begin{lstlisting}
>>> [byte for byte in hash.digest()]
['D', '\xc7', '\xbe', 'H', '"', 'n', '\xba', '\xd5', '\xdc', '\xa8', '!', 'f', 't', '\xca', '\xd6', '+']
\end{lstlisting}

So when Bob sends Alice a message, he hashes the message and then signs the hash (applies his private key).
Then he encrypts the message with Alice's public key. 
When Alice receives the message and signature, she decrypts the message with her private key.
After reading the message, she hashes it and compares the result to Bob's public key applied to the signature.
If they agree, then she knows the message came from Bob.

\begin{problem}
Rewrite your RSA cryptosystem to include digital signatures. 
That is, write the following functions, using the same hash in each.
\begin{lstlisting}
def encrypt(message, e, n):
    '''Encrypt `message' using the public key (e, n) and sign it.
    
    INPUTS:
    message - A string.
    (e, n)  - A pair of integers. Should be an RSA public key.
    
    OUTPUTS:
    ciphertext - A list of the encrypted pieces of `message.'
    signatures - The hash of `message', signed with your private key.
    '''
    
def decrypt(ciphertext, e, n):
    '''Decrypt `ciphertext' using your private key and verify the 
    signature matches the public key (e, n).
    
    INPUTS:
    ciphertext - A list of integers.
    (e, n)     - A public key.
    
    OUTPUTS:
    message  - The decrypted `ciphertext' as a single string.
    verified - A Boolean value indicating whether the digital signature 
               was verified.
    '''
\end{lstlisting}
\end{problem}




















\section*{Making RSA feasible (Optional)}

For RSA to be a reasonably fast cryptosystem, we need a fast algorithm for modular exponentiation.
This algorithm, known as \emph{binary exponentiation}, was discussed in Lab \ref{lab:DiffieHellman}.

In addition, the security of RSA depends on the use of large prime numbers for $p$ and $q$.
For RSA to be a practical cryptosystem, we need a fast way to find such numbers.


\subsection*{Finding large primes}
It is very slow to check that a large number is prime by finding its factors.
In fact, it is the difficulty of factoring that makes RSA secure.

However, there exist fast algorithms that determine when a number is ``probably prime.''
One such algorithm comes from a special case of Theorem \ref{thm:fermat} known as \emph{Fermat's Little Theorem}:
\[
a^{p-1} \equiv 1 \pmod{p}
\]
for all primes $p$ and integers $a$ that are not divisible by $p$.
Therefore, to check if a fixed number $p$ is likely a prime, we can compute $a^{p-1} \pmod{p}$ for several values of $a$.
If we ever get something different than 1, then we know that $p$ is composite.
The value of $a$ that proved $p$ was composite is called a \emph{witness number}.
If  $a^{p-1} \equiv 1 \pmod{p}$ for every $a$ we try, then $p$ is probably prime---but it may still be composite.
This test is called Fermat's test for primality.

With a few exceptions, if $p$ is composite then more than half the integers in $[2, p-1]$ are witness numbers.
Thus, if we run the above test on $p$ just a few times and don't find a witness number, there is a high probability that $p$ is prime.

In practice, the test described is often used as a ``first pass'' to determine whether a number is prime. 
If no witness number is found, then more stringent tests for primality are used.


\begin{problem}
Write the following function, implementing Fermat's test for primality.
\begin{lstlisting}
def is_prime(n):
    '''Use Fermat's test for primality to see if `n' is probably prime.
    
    INPUTS:
    n - An integer.
    
    Run Fermat's test on at most 5 numbers randomly chosen from [2, n-1]. 
    If a witness number is found, return the number of tries it took to 
    find the witness number. If no witness number is found after 5 tries, 
    return 0.
    '''
\end{lstlisting}
For most composite values of \li{n}, if you call \li{is_prime(n)} one hundred times, you should expect to get 0 at most five times.

However, some composite numbers do not follow this rule. 
How many times do you have to call \li{is_prime()} on 340561 to get an answer of 0?
\label{prob:prime_confidence}
\end{problem}






\section*{PyCrypto: RSA in Python (Optional)}
A professional implementation of RSA is found in the PyCrypto module.

% TODO: edit this section. The PyCrypto documentation is extremely poor and I can't figure out how to use it.
\begin{comment}
RSA encryption in Python can be accomplished very easily with a library called PyCrypto.
This library contains many secure hash functions, random number generators, and encryption classes.
Many programs use PyCrypto for their security needs.
\begin{warn}
Make certain that you are using the latest version of PyCrypto.
Security software is updated often to fix security vulnerablilities and bugs.
The current version of PyCrypto at the time of writing is version 2.6.1.
\end{warn}

The RSA module in PyCrypto is located in the PublicKey module.
Let's try a small example using PyCrypto.
The library allows us to explicity construct a key, or generate a key automatically.
\begin{lstlisting}
>>> from Crypto.PublicKey import RSA
>>> from Crypto import Random
>>> keypair = RSA.generate(2048) #generate a 2048-bit RSA key
>>> publickey = keypair.publickey()
>>> share_this = publickey.exportkey()
\end{lstlisting}

The RSA encryption and decryption methods on these keys are textbook approaches.
However, to increase security, we will want to pad the messages so every message encrypted with a particular will become exactly as large (in bits) as the key itself.
A commonly used padding algorithm is implemented in PyCrypto in the \li{Crypto.Cipher.PKCS1_OAEP} module.
\begin{lstlisting}
>>> from Crypto.Cipher import PKCS1_OAEP as oaep

# generate a new key from the original RSA key.
# This key can encrypt and decrypt
>>> paddedkey = oaep.new(keypair)
>>> encrypted = paddedkey.encrypt('hello world')
>>> paddedkey.decrypt(encrypted)
'hello world'
\end{lstlisting}

To sign the message we use PyCrypto's \li{Crypto.Signature.PKCS1_PSS} module.
We first need to hash the message using a cryptographically secure hash, such as SHA.
Just as only private keys can decrypt, only private keys can sign.
Public keys can decrypt and verify.
\begin{lstlisting}
>>> from Crypto.Signautre import PKCS1_PSS as pss
>>> from Crypto.Hash import SHA
>>> sigkey = pss.new(keypair)
>>> mhash = SHA.new("Hello World")
>>> sigkey.sign
\end{lstlisting}


\begin{problem}[Group Project]
Split into several groups of at least three individuals.
Each group member should generate a private and public keypair using PyCrypto.
Share your public keys with everyone in the group.

You might find it useful to transmit your messages as strings.
Here are two functions that will read and write specially formatted strings for your messages.
\begin{lstlisting}
def print_msg(m_encrypted):
    #print the encrypted message as a specially formatted string
    out = ("----BEGIN MESSAGE----",
           m_encrypted[0],
           "----END MESSAGE----",
           "----BEGIN SIGNATURE----",
           m_encrypted[1],
           "----END SIGNATURE----")
    return '\n'.join(out)

def read_msg(m_string):
    d1 = m_string.find('----BEGIN MESSAGE----') + 21
    d2 = m_string.find('----END MESSAGE----', d1)
    
    d3 = m_string.find('----BEGIN SIGNATURE----', d2) + 23
    d4 = m_string.find('----END SIGNATURE----', d3)
    message  = m_string[d1:d2]
    sign = m_string[d3:d4]
    return message.strip(), sign.strip()
\end{lstlisting}

\begin{enumerate}
\item Encrypt and sign a message for someone in the group and send it to the entire group.
Attach the sender's name to the encrypted message (a claim that the message came from a particular sender).
In this exercise we will verify that the actual origin of the message is indeed the claimed origin.
Each group member should do the following:
\begin{enumerate}
\item Decrypt and verify the message addressed for you.
\item Attempt to decrypt and verify message addressed for someone else in the group.
\item Report the message and verified signature.
\end{enumerate}

\item Encrypt and sign a different message for someone in the group and send it, anonymously, to the entire group.
For example, if Bob encrypts a message and sends it to Alice, he should claim the message came from another individual.
Each group member should do the following:
\begin{enumerate}
\item Decrypt and verify the message address to you.
\item Report the message and verified signature.
\end{enumerate}
\end{enumerate}
\end{problem}
\end{comment}


\section*{Appendix: Helper Code}
Here are some functions that you can use to make your implementation of RSA more user-friendly (see the ``Logistical considerations'' section).
Note that you must import \li{izip_longest} from \li{itertools} before using some of these functions.
 
 \begin{lstlisting}
 from itertools import izip_longest
 
 def partition(iterable, n, fillvalue=None):
    '''Partition data into blocks of length `n', padding with `fillvalue' if needed. Return a list of the partitions.
    
    EXAMPLE:
    >>> partition('ABCDEFG', 3, 'x')
    ['ABC', 'DEF', 'Gxx']
    '''
    args = [iter(iterable)] * n
    pieces = izip_longest(fillvalue=fillvalue, *args)
    return [''.join(block) for block in pieces]
    
def string_size(n):
    '''Return the maximum number of characters that can be encoded with the public key (e, n).
    
    In other words, find the largest integer L, such that if `string' has at most L characters, then a2i(`string') will be less than `n.'
    '''
    L=0
    max_int = 0
    while max_int < n:
        max_int += sum([2**i for i in range(8*L, 8*L+8)])
        L += 1
    return L-1
    
def string_to_int(msg):
    '''Convert the string `msg' to an integer.
    '''
    # bytearray will give us the ASCII values for each character
    if not isinstance(msg, bytearray):
        msg = bytearray(msg)
    binmsg = []
    # convert each character to binary
    for c in msg:
        binmsg.append(bin(c)[2:].zfill(8))
    return int(''.join(binmsg), 2)

def int_to_string(msg):
    '''Convert the integer `msg' to a string.
    
    This function is the inverse of string_to_int().
    '''
    # convert to binary first
    binmsg = bin(msg)[2:]
    # we need to pad the message so length is divisible by 8
    binmsg = "0"*(8-(len(binmsg)%8)) + binmsg
    msg = bytearray()
    for block in partition(binmsg, 8):
        # convert block of 8 bits back to ASCII
        msg.append(int(block, 2))
    return str(msg)
 \end{lstlisting}