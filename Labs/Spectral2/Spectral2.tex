\lab{Solitons}{Solitons}
\label{lab:pseudospectral2}

\objective{We study traveling wave solutions of the Korteweg-de Vries (KdV) equation.  We will numerically solve this time-dependent PDE by using a pseudospectral discretization in space and a Runge-Kutta integration scheme in time.  }

There are canonical model equations for various types of phenomena.  Long, small amplitude waves moving in one direction can generally be described by the  Korteweg-de Vries (KdV) equation. The KdV equation is given by 
\[  \frac{\partial u }{\partial t} + u \frac{\partial u}{\partial x} + \frac{\partial^3 u}{\partial x^3} = 0
.\]

The KdV equation possesses traveling wave solutions called solitary waves. These traveling waves have the form 
\[ u(x,t) = 3\alpha^2 \sech^2(\alpha(x-x_0)/2 - \alpha^3 t).
\]

History of solitons: first observed by John Scott Russell in 1834. 
Solution form:  
Interaction of solitons: 

Solitons versus solitary waves. I believe solitary waves are solitons in a fluid dynamics context. 
John Russell 



















