\lab{Solitons}{Solitons}
\label{lab:pseudospectral2}

\objective{We study traveling wave solutions of the Korteweg-de Vries (KdV) equation. This time-dependent PDE can be numerically solved by using a pseudospectral discretization in space and a Runge-Kutta integration scheme in time.  }

\section{A Pseudospectral method for periodic functions}
Let $f$ be a periodic function on $[0,2\pi]$. Let $x_1,\ldots,x_N$ be $N$ evenly spaced grid points on $[0,2\pi].$ Since $f$ is periodic on $[0,2\pi]$, we can ignore the grid point $x_0 = 0$. We will further assume that $N$ is even; similar formulas can be derived for $N$ odd. Let $h = 2\pi/N$; then $\{x_1,\ldots,x_N\} = \{h,2h,\ldots,2\pi-h,2\pi\}$.  

The discrete Fourier transform (DFT) of $f$ is 
\[
\hat{f}(k) = h \sum_{j=1}^N e^{-ikx_j}f(x_j).
\]
The inverse DFT is then given by
\begin{align}
f(x_j) = \frac{1}{2\pi}\sum_{k=-N/2+1}^{N/2}e^{ikx_j}\hat{f}(k), \quad j = 1,\ldots, N. \label{inverse_dft1}
\end{align}

\begin{align}
f(x_j) = \frac{1}{2\pi}\sum_{k=-N/2}^{N/2}\!\!{}^{'} e^{ikx_j}\hat{f}(k), \quad j = 1,\ldots, N. \label{inverse_dft2}
\end{align}
The inverse DFT can then be used to define a natural interpolant (sometimes called a band-limited interpolant) by evaluating (\ref{inverse_dft2}) at any $x$ rather than $x_j$:
\begin{align}
p(x) = \frac{1}{2\pi}\sum_{k=-N/2+1}^{N/2} e^{ikx}\hat{f}(k). \label{interpolant}
\end{align}


The interpolant for $f'$ is then given by 
\[
p'(x) = ik \frac{1}{2\pi}\sum_{k=-N/2}^{N/2} e^{ikx}\hat{f}(k).
\]



\section{The KdV equation}

There are canonical model equations for various types of phenomena.  Long, small amplitude waves moving in one direction can generally be described by the  Korteweg-de Vries (KdV) equation. The KdV equation is given by 
\[  \frac{\partial u }{\partial t} + u \frac{\partial u}{\partial x} + \frac{\partial^3 u}{\partial x^3} = 0.
\]

The KdV equation possesses traveling wave solutions called solitary waves. These traveling waves have the form 
\[ u(x,t) = 3\alpha^2 \sech^2(\alpha(x-x_0)/2 - \alpha^3 t).
\]

History of solitons: first observed by John Scott Russell in 1834. 
Solution form:  
Interaction of solitons: 

Solitons versus solitary waves. I believe solitary waves are solitons in a fluid dynamics context. 
John Russell 



















