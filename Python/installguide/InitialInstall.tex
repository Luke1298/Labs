\lab{Initial Installation}{Initial Installation}
\label{pythoninstall}

\objective{Install fundamental package requirements for the labs on a computer.}

\begin{warn}
The information provided in this appendix is for convenience only.
The reader assumes all the liability and risk involved in making any change to configuration mentioned in this appendix.
The authors of this appendix are not responsible for any damage that may result from any changes covered in following material.
Some of the changes could cause unexpected behavior in the computer system.
\end{warn}

There are a wide variety of ways to install most of the Python packages required for these labs.
Installing these libraries can be difficult; the following are ways that have worked for us.
They still may not work on every computer. The following steps have been verified by the authors of this guide on Windows 7, Windows 8.1, MacOSX, and Ubuntu 14.04.

% We suggest students use Anaconda because it includes nearly all the packages we use.
% They also tend to update their packages more often.
% It is also much less intrusive on the host filesystem.
\section*{Installing Using the Anaconda Python Distribution}

Regardless of which operating system you are using, the basic processes needed to install Anaconda are the same:
\begin{enumerate}
\item Download the installer from \url{http://www.continuum.io/downloads}
\item Ensure that the main \li{Anaconda} directory has been added to the system path
\item Perform some minor tweaks to files in the distribution to prepare for future packages. This step will be the same on all operating systems.
\end{enumerate}

\subsection*{Windows 64-bit Installation}
We will provide detailed steps for Windows 64-bit installation, but as already mentioned, the process as a whole is the same on each operating system.
\begin{enumerate}
\item Note: you can get an academic license for more of the features of the Anaconda Distribution.
The primary benefit is that it will allow many basic linear algebra routines to run much faster. This is not needed for the initial installation but is an option to consider.
To obtain an academic license, see \url{https://store.continuum.io/cshop/academicanaconda}.
To activate your academic license, follow the steps in the confirmation email.

\item Install the Anaconda Python Distribution for Windows from \url{http://www.continuum.io/downloads}

\item Start the installer. Install using the default configuration. As it is installing, make note of the directory where Anaconda is being installed.

\item Ensure the main \li{Anaconda} directory and the \li{bin}, \li{libs}, and \li{include} subdirectories have been added to your system path if they aren't already there.
On Windows, you can find this option by opening the file explorer to ``Computer", right clicking on the background and clicking on ``Properties", then clicking on ``Advanced Settings", then clicking on ``Environment Variables". If the Anaconda was installed at \li{C:\\Anaconda\\}, add the following to the end of the path if it isn't already there:

\li{;C:\\Anaconda\\; C:\\Anaconda\\Scripts; C:\\Anaconda\\libs; C:\\Anaconda\\include}

If the Anaconda directory was installed at \li{C:\\Users\\John\\Anaconda\\}, add the following to the end of the path if it isn't already there:

\li{;C:\\Users\\John\\Anaconda\\; C:\\Users\\John\\Anaconda\\Scripts; C:\\Users\\John\\Anaconda\\libs; C:\\Users\\John\\Anaconda\\include}

This should be done to the path variable in the box for ``System Variables" (the system path instead of the user path).
\end{enumerate}

\subsection*{Minor Edits to Files}
These minor edits are not necessary to get started right away, but will be necessary for some of the libraries we will be using in the future.
\begin{enumerate}
% This is what allows f2py to work on 64 bit windows.
% The version of MinGW that comes as a part of Anaconda is
% currently a 64 bit version of gcc and gfortran 4.7.
% Using MinGW64 with f2py is not officially supported, but it works.
\item Open \li{C:\\Anaconda\\Lib\\site-packages\\numpy\\distutils\\fcompiler\\gnu.py} and comment out the two lines that read 
\begin{lstlisting}
else:
    raise NotImplementedError("Only MS compiler supported with gfortran on win64")
\end{lstlisting}
They are probably somewhere around line $330$.
Once you have modified them, they should read
\begin{lstlisting}
#else:
#    raise NotImplementedError("Only MS compiler supported with gfortran on win64")
\end{lstlisting}

% The first definition eliminates the issue with "unable to find vcvarsall.bat" error.
% It is currently already included as a part of Anaconda.
% The second, as of this writing, is not currently included in Anaconda.
% It is needed for pyximport to work properly.
\item Open the file \li{C:\\Anaconda\\Lib\\distutils\\distutils.cfg} (create it if it doesn't already exist) and make sure it contains the following two definitions
\begin{lstlisting}
[build]
compiler=mingw32

[build_ext]
compiler=mingw32
\end{lstlisting}

\end{enumerate}

\begin{info}
If you update NumPy, you will need to modify the file \li{C:\\Anaconda\\Lib\\site-packages\\numpy\\distutils\\fcompiler\\gnu.py} the same way you did when you first installed everything.
\end{info}

\section*{Workflows}
There are several different ways to write your programs and run them in Python. It will be valuable for you to try a variety of workflows to find which one you prefer.

\subsection*{Text editor + Python Interpreter/IPython}
This is the most basic workflow to use.
It involves using your favorite text editor to edit source files and then executing
those scripts in your command terminal.
Some popular editors for python source code include:
\begin{itemize}
\item Vim
\item Emacs
\item Sublime
\item Notepad++ (Windows)
\item TextWrangler (OS X)
\item Geany (Linux)
\end{itemize}

IPython is an enhanced interactive interpreter for Python.
It has many features that cater to productivity.
One very useful feature is object introspection.
This feature allows you to examine the properties and methods of any object in the
interpreter.
Another useful feature is tab completion where the interpreter automaticalls fills in partially typed commands when the tab key is pressed.
Using the IPython interpreter with a text editor is a good way to work in Python.

\subsection*{Integrated Development Environments}
Integrated Development Environments (IDEs) provide a comprehensive environment for
application development.
Most IDEs have many tightly integrated tools that are easily accessible.
Some popular IDEs for developing in Python are discussed in this section.

\subsubsection*{IPython Notebook}
The IPython Notebook is a browser-based interface to Python.
It is similar to the notebook interfaces used by Mathematica, Maple, and Sage.
Running a notebook is similar to running a Python interpreter session except that
the input is stored in cells and can be modified and re-evaluated as needed.
You can also save notebooks and reload them later.
The IPython notebook is included as part of the Anaconda Python Distribution, the
Enthought Python Distribution, and Python(x,y).

\subsubsection*{Eclipse + PyDev}

Eclipse: \url{http://www.eclipse.org/} \\
PyDev: \url{http://pydev.org/}

Eclipse is a general purpose IDE that supports many languages.
The PyDev plugin for Eclipse contains all the tools needed to start working in Python.
It includes a built-in debugger, and has a very nice code editor.
Eclipse + PyDev is available for Windows, Linux, and Mac OSX.

\subsubsection*{Spyder}

Spyder: \url{http://code.google.com/p/spyderlib/}

Spyder is a Python IDE that is designed specifically for scientific computing.
Its interface is reminiscent of MATLAB and requires PyQT4 to be installed.
Spyder is available for Windows and Linux.
There is a version available for Mac OSX through MacPorts.


