\lab{Appendix}{Windows 64 bit Installation}{Windows 64 bit}
\label{win64install}

\objective{Install fundamental package requirements for the labs on a Windows 64 bit machine.}

\begin{warn}
The information provided in this appendix is for convenience only.
The reader assumes all the liability and risk involved in making any change to configuration mentioned in this appendix.
The authors of this appendix are not responsible for any damage that may result from any changes covered in following material.
Some of the changes could cause unexpected behavior in the computer system.
\end{warn}

There are a wide variety of ways to install most of the Python packages required for these labs.
There are, however, some problems with Cython and F2py since the distutils package included in Python is currently not very good for building new extension modules on 64 bit Windows machines.
Installing these libraries can be difficult; the following are ways that have worked for us.
They still may not work on every computer.

\section*{Installing Using the Anaconda Python Distribution}
\begin{enumerate}
\item Note: you can get an academic license for more of the features of the Anaconda Distribution.
To activate your academic license, follow the steps in the confirmation email.

\item Install the Anaconda Python Distribution from \url{http://www.continuum.io/downloads}

\item Add the main Python directory and the bin/ , libs/ , and include/ subdirectories to your system path if they aren't already there.
You can find this option by opening the file explorer to "Computer", right clicking on the background and clicking on "Properties", then clicking on "Advanced Settings", then clicking on "Environment Variables", then add the following to the end of the path if it isn't already there: ";C:/Anaconda/; C:/Anaconda/Scripts; C:/Anaconda/libs; C:/Anaconda/include". This should be done to the path variable in the box for "System Variables" (the system path instead of the user path).

\item Open "c:/Anaconda/Lib/site-packages/numpy/distutils/fcompiler/gnu.py" and comment out the two lines that read \li{else:} and \li{raise NotImplementedError("Only MS compiler supported with gfortran on win64")}.
They are probably somewhere around line $330$.

\item Search for the file "libmsvcr90.a" in your Python directory and copy it into "C:/Anaconda/libs"

\item Open the file "C:/Anaconda/Lib/distutils/distutils.cfg" (create it if it doesn't already exist) and make sure it contains the following two definitions
\begin{lstlisting}
[build]
compiler=mingw32

[build_ext]
compiler = mingw32 

\end{lstlisting}

\item Install and install the following additional packages from \url{http://www.lfd.uci.edu/~gohlke/pythonlibs/}
\begin{enumerate}
	\item pyfftw
	\item pywavelets
	\item cvxopt
	\item mpi4py
	\item pycairo (optional for students, required for anyone that contributes to these labs)
\end{enumerate}

\item Download and install the Git version control system.
You can get an installer at \url{http://git-scm.com/downloads}.

\item Download and install MikTex from \url{http://miktex.org/download}.
You can use either the 32 or 64 bit version in this case.

\end{enumerate}

\section*{Other Miscellaneous Issues}
\begin{itemize}
\item If Python scripts are not recognizing their inputs modify the windows registry values as described \href{http://stackoverflow.com/questions/2640971/windows-is-not-passing-command-line-arguments-to-python-programs-executed-from-t}{here}.
\end{itemize}
