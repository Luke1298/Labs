\lab{Data Structures I: Lists}{Data Structures I}
\label{lab:Python_DataStructures}

\objective{Discuss the purpose of data structures and implement several kinds of linked lists.}

\section*{Introduction}

Analyzing and manipulating data are essential skills in scientific computing.
Storing, retrieving, and manipulating data take time.
As a dataset grows, so does the amount of time it takes to access and manipulate it.
The structure of how the data are stored determines how efficiently the data may be processed.

Data structures are specialized objects for organizing data.
There are many kinds, each with specific strengths and weaknesses.
For example, some data structures take a long time to build, but once built their data are quickly accessible.
Others are built quickly, but are not as efficiently accessible.
Different applications require different structures for optimal performance.

\section*{Nodes}

Booleans, strings, floats, and integers are some of the built-in data types in Python.
Most data in applications take one of these forms.
However, as the size of a dataset increases, these types prove inefficient.
Many data structures use \emph{nodes} to overcome these inefficiencies.

If we thought of data as several types of objects that need to be stored in a warehouse, a node would be a standard size box that can hold all the different types of objects.
Suppose a warehouse needs to store lamps of various sizes.
Rather than trying to stack lamps of different shapes on top of each other efficiently, it is preferable to put them in the boxes of standard size.
Then adding new boxes and retrieving stored ones becomes much easier.

A \li{Node} class is usually simple.
In Python, the data in the Node is stored as an attribute.
Other attributes may be added (or inherited) specific to a particular data structure.
The data structure links the nodes together in a way that is efficient for its particular application.

\begin{lstlisting}
# Node.py

class Node(object):
	"""A Node class for storing data."""
	def __init__(self, data):
		"""Construct a new node that stores some data."""
		self.data = data
\end{lstlisting}

\begin{lstlisting}
# Import the Node class from Node.py
>>> from Node import Node

# Create some nodes. Note that any data type may be stored.
>>> int_node = Node(1)
>>> str_node = Node('abc')
>>> list_node = Node([1,'abc'])

# Access a node's data.
>>> list_node.data
[1, 'abc']
\end{lstlisting}

% Problem 1: Node magic methods
\begin{problem}
Add extra functionality to the \li{Node} class by implementing the \li{__str__} magic method so that it returns a string representation of its data.
Also implement the \li{__lt__}, \li{__eq__}, and \li{__gt__} comparison magic methods so that the data stored inside of two Nodes is compared.
For example, a Node \li{x} is less than a Node \li{y} if and only if the data contained in \li{x} is less than the data contained in \li{y}.
\end{problem}

\section*{Linked Lists}

A linked list is a data type that chains nodes together.
Each node instance in a linked list stores a reference to the next link in the chain.
A linked list class also stores a reference to the first node in the chain, called the \li{head}.
See Figure \ref{fig:singly_linked}.

\begin{figure}
\centering
\begin{tikzpicture}[->,>=stealth',shorten >=1pt,auto, node distance=1.5cm,thick,main node/.style={rectangle,draw}, minimum size=.5cm]
\tikzset{rect node/.style={rectangle, draw, minimum height = .5cm, minimum width=.2cm}}
  \node[main node] (1) {B};
  \node[main node] (2) [right of=1] {F};
  \node[main node] (3) [right of=2] {G};
  \node[main node] (4) [right of=3] {C};
  \node[draw = none, black!20!blue, node distance=1.5cm] [above left of=1](H) {Head};
\foreach \r in {1, 2, 3, 4}{
  \node[rect node][right of=\r, node distance = .36cm]{};}
\node[draw = none, node distance = 1.5cm] [right of=4]{};  % Centralize this particular figure
\foreach \s/\t  in {1/2, 2/3, 3/4}{
    \path[draw](\s) edge[shorten <=.1cm](\t);}
  \draw[black!20!blue] (H) edge (1.north);
\end{tikzpicture}
\caption{Singly-linked List}
\label{fig:singly_linked}
\end{figure}

\begin{lstlisting}
# Node.py

class LinkedListNode(Node):
	"""A Node class for linked lists. Inherits from the 'Node' class.
	Contains a reference to the next node in the list.
    """
	
	
    def __init__(self, data):
        """Construct a Node and initialize an attribute for the next
        node in the list.
        """
        Node.__init__(self, data)
        self.next = None
\end{lstlisting}

A basic implementation of a linked list will have a constructor and a method for adding new nodes to the end of the list.
To get to the end of the list, start at the \li{head} of the list.
Then traverse the list by going from node to node until the end is reached.
Then, set the \li{next} attribute of the last node to be the new node.
This is done in the following class.
See Figure \ref{fig:add} for an illustration.

\begin{lstlisting}
# spec.py

from Node import LinkedListNode

class LinkedList(object):
	"""Singly-linked list data structure class.
	The first node in the list is referenced to by 'head'.
	"""
	
	def __init__(self):
		"""Create a new empty linked list. Create the head
		attribute and set it to None since the list is empty.
		"""
		self.head = None

	def add(self, data):
		"""Create a new Node containing 'data' and add it to
		the end of the list.
		"""

		new_node = LinkedListNode(data)
		if self.head is None:
			# If the list is empty, point the head attribute to the new node.
			self.head = new_node
		else:
			# If the list is not empty, traverse the list
			# and place the new_node at the end.
			current_node = self.head
			while current_node.next is not None:
				# This moves the current node to the next node if it is not
				# empty. Then when we break out of the loop, current_node
				# points to the last node in the list.
				current_node = current_node.next

			current_node.next = new_node
\end{lstlisting}

\begin{figure}
\centering
\caption{Include a picture of the \li{add} method and describe it.}
\label{fig:add}
\end{figure}

% Problem 2: __str__ for the LinkedList class.
\begin{problem}
Write the \li{__str__} method for the \li{LinkedList} class so that when a \li{LinkedList} instance is printed, its output matches that of a Python list.
\end{problem}

In addition to adding new nodes to the end of a list, it is also useful to remove nodes and insert new nodes at specified locations.
To delete a node, all references to the node must be removed.
Then Python will automatically delete the object, since there is no way for the user to access it.
Na{\"i}vely, this might be done by finding the previous node to the one being removed, and setting its \li{next} attribute to none.

\begin{lstlisting}
# A naive node removal - Does not work.

def remove(self, data):
	"""Remove the Node containing 'data'."""

	# Find the node whose next node contains data
	current_node = self.head
	while current_node.next.data != data:
		current_node = current_node.next

	# Remove the next reference to the target node.
	current_node.next = None
\end{lstlisting}

Since the only reference to the node that is deleted is the previous node's next attribute, this will delete the node.
However, since the only reference to the next node came from the deleted node, it also will deleted.
This will continue to the end of the list.
Thus, deleting one node in this manner deletes the remainder of the list.
This can be remedied by pointing the previous node's next attribute to the node after the deleted node.
Then there will be no reference to the removed node and it will be deleted. 
See Figure \ref{fig:remove} for an illustration.

\begin{figure}
\centering
\caption{Include two pictures of the \li{remove} method: the na{\"i}ve method that doesn't work, and the true method that does. Describe them.}
\label{fig:remove}
\end{figure}


\begin{lstlisting}
# A node removal that works

def remove(self, data):
	"""Remove the Node containing 'data'."""

	# First, check if the head is the node to be removed. If so, set the
	# new head to be the first node after the old head. This removes
	# the only reference to the old head, so it is then deleted.
	if self.head.data == data:
		self.head = self.head.next
	else:
		current_node = self.head
		# Move current_node through the list until it points to the node that
		# precedes the target node.
		while current_node.next.data != data:
			current_node = current_node.next
	
		# Point current_node to the node after the target node.
		new_next_node = current_node.next.next
		current_node.next = new_next_node
\end{lstlisting}

\begin{warn}

Python keeps track of the variables in use and automatically deletes a variable if there is no access to it.
In many other languages, leaving a reference to an object without explicitly deleting it could cause a serious memory leak.
See \href{https://docs.python.org/2/library/gc.html}{here} for more information on python's auto-cleanup system.
\end{warn}

% Problem 3: Complete LinkedList.remove()
\begin{problem}
Though the above code works to remove specified nodes, it is not quite complete.
Modify the \li{remove} method to account for possible errors: if the list is empty, or if the target node is not in the list, raise a \li{ValueError} with error message ``\li{data} is not in the list.''
\end{problem}

\begin{figure}
\centering
\caption{Include a picture of the \li{insert} method and describe it.}
\label{fig:insert}
\end{figure}

% Problem 4: LinkedList.insert()
\begin{problem}
Inserting a node at a specified location is similar to removing a node.
Add an \li{insert} method to the \li{LinkedList} class that inserts a new node before the first node in the list that contains the data specified by the user.
This function accepts two arguments: the data for the new node (\li{data}), and the data of the node before which the new node will be inserted (\li{place}).
If the list is empty, or there is no node containing \li{place} in the list, raise a \li{ValueError} with error message ``\li{place} is not in the list.''

See Figure \ref{fig:insert} for an illustration of the \li{insert} method.
Note that since \li{insert} inserts a node before a specified node that is already in the list, it is not possible to \li{insert} at the end of the list or to an empty list.
\end{problem}

\section*{Doubly-Linked Lists}

\begin{figure}
\centering
\begin{tikzpicture}[->,>=stealth',shorten >=1pt,auto, node distance=1.5cm, thick,main node/.style={rectangle,draw}, minimum size=.5cm]
\tikzset{rect node/.style={rectangle, draw, minimum height = .5cm, minimum width=.9cm}}
  \node[main node] (1) {B};
  \node[main node] (2) [right of=1] {F};
  \node[main node] (3) [right of=2] {G};
  \node[main node] (4) [right of=3] {C};
  \node[draw = none, black!20!blue, node distance = 1.5cm] [above right of=4] (T) {Tail};
  \node[draw = none, black!20!blue, node distance = 1.5cm] [above left of=1] (H) {Head};
  \node[rect node](1.5)[]{};
  \node[rect node](2.5)[right of=1.5]{};
  \node[rect node](3.5)[right of=2.5]{};
  \node[rect node](4.5)[right of=3.5]{};

\foreach \s/\t  in {1/2, 2/3, 2/1, 3/2, 3/4, 4/3}{
	\path[draw](\s) edge[shorten <=.1cm, shorten >=.1cm](\t);}	
  \draw[black!20!blue] (H) edge (1.north);
  \draw[black!20!blue] (T) edge (4.north);
\end{tikzpicture}
\caption{Doubly-linked List}
\label{fig:doubly_linked}
\end{figure}

A doubly-linked list is a linked list where each node keeps track the node that precedes it as well as the node that follows.
The end of the list is also typically kept track of with a \li{tail} attribute.
See Figure \ref{fig:doubly_linked} for an illustration.


\begin{lstlisting}
# Node.py

class DoublyLinkedListNode(LinkedListNode):
	"""A Node class for doubly-linked lists. Inherits from the 'Node' class.
	Contains references to the next and previous nodes in the list.
	"""
	def __init__(self,data):
		"""Set the next and prev attributes."""
		Node.__init__(self,data)
		self.next = None
		self.prev = None
\end{lstlisting}

\begin{figure}
\centering
\caption{Illustrate insertion for doubly-linked lists.}
\label{fig:d_insertion}
\end{figure}

\begin{figure}
\centering
\caption{Illustrate removal for doubly-linked lists.}
\label{fig:d_removal}
\end{figure}


All of the methods for linked lists can be implemented for doubly-linked lists.
See Figures \ref{fig:d_insertion} and \ref{fig:d_removal} for illustrations of the insert and remove methods.

% Problem 5: doubly-linked list
\begin{problem}
Implement a \li{DoublyLinkedList} class that inherits from \li{LinkedList} and uses \li{DoublyLinkedListNode} instances to build the list.
Add an attribute called \li{tail} that keeps track of the node at the end of the list.
Overwrite the \li{add}, \li{remove}, and \li{insert} methods.
Use the \li{tail} attribute to make the \li{add} method more efficient.
Raise the same exceptions in the \li{insert} and \li{remove} methods as before.
\end{problem}

% Problem 6: sorted linked list
\begin{problem}
Implement a sorted linked list.
This data structure adds new nodes strategically so that the data is always kept in order.
Inherit this class from \li{DoublyLinkedList}, and override the \li{add} and \li{insert} methods.
When a new node is added, traverse the list until the data in the next node is greater than or equal to the data for the new node.
Then insert the new node, thereby preserving the ordering.
Also override the \li{insert} method with the following:

\begin{lstlisting}
def insert(self, *args):
	raise ValueError("insert() has been disabled for this class.")
\end{lstlisting}

This effectively disables \li{insert} for the \li{SortedLinkedList} class and prevents the user from accidentally inserting a node in a location that would disrupt the ordering.
The \li{*args} argument allows \li{insert} to receive any number of arguments without raising a \li{TypeError} exception.

To test this data structure, import the provided \li{WordList} module.
This includes a method called \li{create_word_list} that reads each line of text from a file and returns it as a randomly-ordered list of strings.
Write a function called \li{sort_words} that sorts the list generated by \li{create_word_list} by adding them to a \li{SortedLinkedList} object.
Then return the object.

\begin{warn}
\li{English.txt}, the default source file for \li{create_word_list}, contains over 58,000 English words.
Sorting the entire data set should take about 15 minutes.
Test your data structure on small data sets first.
\end{warn}
\end{problem}

Python has many quick sorting methods.
Even on the seemingly large data set of over 58,000 words used in the preceding problem, the \li{sort} method for Python lists is almost instantaneous.
In the next lab we will turn our attention to \emph{trees}, special kinds of linked lists that allow for much quicker sorting and data retrieval.

