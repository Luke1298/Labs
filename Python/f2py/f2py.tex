\lab{F2PY}{F2PY}
\label{lab:f2py}
\objective{Learn how to use F2PY, especially for wrapping Fortran code involving arrays.}

Here we will explore another simple way to access external code from Python.
Various utilities exist to do this, and many are specialized for specific tasks.
F2PY is a utility that can be used to automatically wrap Fortran subroutines so that they are callable from Python.
It is included with NumPy and in most cases does not usually require heavy configuration for use.
The primary requirement for using F2PY is a working Fortran compiler.
If you have access to gfortran, the Fortran equivalent of gcc, it should work well.

\section*{Examples}

F2PY is an interface generator designed to be used primarily to generate Python bindings for array based functions and subroutines from Fortran.
Many simple Fortran files can be compiled \emph{automatically} without any hand-written interfaces between languages.
Windows 64 bit users, if you encounter errors, see the Windows 64 bit install guide.
Here we will wrap a Fortran function that performs a basic tridiagonal solve.
This function is essentially the same as the one wrapped in Lab \ref{lab:cythonwrap}.
In Lab \ref{lab:cythonwrap} the Fortran function had been modified so that it was already callable from C.
F2PY takes care of necessary name and type changes so that pure Fotran functions and subroutines can be wrapped without modification.
We will wrapp the following unmodified version of the algorithm for a tridiagonal solve.
\lstinputlisting[style=fromfile, language=Fortran, name=ftridiag.f90]{./ftridiag/ftridiag.f90}
This subroutine can be wrapped by running the following in the terminal:
\begin{lstlisting}[style=ShellInput]
f2py -c ftridiag.f90 -m ftridiag
\end{lstlisting}
Depending on the way your system is configured, the following command may be needed
\begin{lstlisting}[style=ShellInput]
python f2py.py -c ftridiag.f90 -m ftridiag
\end{lstlisting}
Additional optimization flags can be passed to the compiler using the \li{--opt} flag, as in:
\begin{lstlisting}[style=ShellInput]
f2py -c ftridiag.f90 -m ftridiag --opt="-O3 -march=native"
\end{lstlisting}

Running these commands will create a file called \li{ftridiag.pyd} on a Windows computer or \li{ftridiag.so} on a Linux or Mac based computer.
This file is a Python module that is linked against the compiled Fortran code.
All the details about array type, shape, size, etc. are infered by F2PY.
This wrapper can be tested using a test similar to the one used in Lab \ref{lab:cythonwrap}
\lstinputlisting[style=fromfile, name=ftridiag_test.py]{./ftridiag/test.py}
\begin{info}
This example shows how to compile a single subroutine.
If the subroutine were a part of a Fortran module, F2PY would have created a submodule for the compile module.
For example, if the subroutine were contained in the module \li{tridiag}, after compilation, it would be available in the Python submodule \li{ftridiag.tridiag} as \li{ftridiag.tridiag.ftridiag}.
\end{info}

\begin{info}
In many cases F2PY can automatically infer the correct values for arguments that are used to convey information about array shape.
These arguments are usually included as optional arguments in the calling signature for the Python function.
F2PY also verifies that the shapes of the arrays passed to each subroutine match the shapes of arrays shown in the declaration of the interface for the subroutine.
\end{info}

\begin{warn}
When working with arrays that have more than one dimension, F2PY expects arrays that are Fortran ordered, i.e. arrays that have their columns stored in contiguous blocks.
By default it will raise an error if it receives a C-contiguous array, though it can be configured to silently make a copy instead.
A Fortran ordered array can be made by taking the transpose of a C contiguous array.
This becomes very important when working with arrays with two or more dimensions.
\end{warn}

Again, similar to working with Cython, we can also create a \li{setup.py} file to create our Python extension module.
The following \li{setup.py} file will do the same thing as the terminal command used above.
\lstinputlisting[style=fromfile, name=ftridiag_setup.py]{./ftridiag/setup.py}

\begin{problem}
Included with this lab is a pure-fortran version of the \li{fssor} subroutine that you wrapped in Lab \ref{lab:cythonwrap}.
Wrap it again using f2py.
Write a \li{setup.py} file as shown above.
When wrapping automatically in this way, F2PY was able to infer that \li{m} and \li{n} are the dimensions of the array.
On the other hand, we have not defined default vaues for the tolerance and the maximum number of iterations, so those must still be passed in as arguments.
\end{problem}

\section*{Detailed Compilation Steps}
Though F2PY can be used automatically as shown, it can be used to generate more carefully customized wrappers for Fortran functions and subroutines.
When compiling an extension, F2PY performs two basic steps that are accessible to users:
\begin{itemize}
\item Create a \li{.pyf} signature file defining the interface for the Fortran subroutines
\item Compile the Fortran code and use Python's C API to make an extension module that calls it in the way defined in the signature file.
\end{itemize}
If we want a more specific interface for the Fortran code we are wrapping we can generate the signature file, then modify it so that F2PY will make a Python function with the desired calling signature.

Consider the \li{ftridiag} subroutine used above.
F2PY can be used to generate a signature file based on the contents of the file \li{ftridiag.f90} via the command
\begin{lstlisting}[style=ShellInput]
f2py ftridiag.f90 -m ftridiag -h ftridiag.pyf
\end{lstlisting}
The signature file created by F2PY is
\lstinputlisting[style=fromfile, language=Fortran, name=ftridiag.pyf]{./ftridiag/ftridiag.pyf}
This looks almost like a function signature in Fortran, except that F2PY has accounted for the fact that \li{n} is one more than the dimension of the first array.

Notice that F2PY has automatically recognized that the last argument to the Fortran subroutine can be inferred directly from the length of the arrays given as arguments.
When it detects that an argument can be inferred from array shapes, it includes it as an optional argument in the Python binding.
F2PY also automatically checks that the dimensions of the arrays passed are consistent with the dimensions listed in the subroutine declaration.

A F2PY signature file can, in turn, be used to create a Python extension module that wraps the Fortran subroutines as described in the signature file.
The following command will use the signature file and Fortran source file above to create a Python extension module.
\begin{lstlisting}[style=ShellInput]
f2py -c ftridiag.pyf ftridiag.f90
\end{lstlisting}

Signature files can also be used in \li{setup.py} scripts like this:
\lstinputlisting[style=fromfile, language=Python, name=setup.py]{./ftridiag/setup_with_sig.py}

F2PY's signature files are more useful when working with arrays with more than one dimension.
Consider the following Fortran subroutine that uses a simple iterative method to find an approximate solution to Laplace's equation on a square.
\lstinputlisting[style=fromfile, language=Fortran, name=flaplace.f90]{./flaplace/flaplace.f90}

% Wrap a C function using f2py to teach about signature files.

% Wrap a LAPACK function (dgtsv?), have them do it too.

\begin{warn}
The LAPACK routines available in \li{scipy.linalg.lapack} do not raise errors when passed a C contiguous array.
Instead, the array is copied into Fortran contiguous form before it is passed to the Fortran subroutine.
This means that code is less likely to raise errors, but the implicit copying of arrays may slow a program down unnecessarily.
\end{warn}

% Discuss how memory layout affects performance.

% Discuss programmer time value.
