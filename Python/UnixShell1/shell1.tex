\lab{Unix Shell}{Unix Shell}
\label{lab:Shell}
% Written by Tanner Christensen, Summer 2015
% If more content in the Problems is needed, in Problem 6, we could add:
%    "edit count_files.py to exclude the files in the test directory"

\objective{Introduce the basics of the Unix Shell commands the Vim text editor}

Unix was first developed by AT\&T Bell Labs in the 1970s. In the 1990s, Unix became the foundation of Linux and MacOSX. The majority of servers are written in Linux, so having a knowledge of Unix shell commands allows us to interact with these servers. 

The more you learn about Unix, you will find it is easy to learn but difficult to master. We will build a foundation of simple file system management and a basic introduction to the vim text editor.  
We will address some of the basics in detail and also include lists of commands that interested learners are encouraged to research further.

\section*{File System}
\subsection*{Navigation}

Begin by opening the Terminal. The text you see in the upper left of the Terminal window is called the \emph{prompt}. We will begin using three very basic commands: \li{pwd}, \li{ls}, and \li{cd}. The \li{pwd} command stands for \textbf{p}rint \textbf{w}orking \textbf{d}irectory. The \li{ls} command lists the files and subdirectories within the working directory. The \li{cd} command stands for \textbf{c}hange \textbf{d}irectory.

%The only reason this problem exists is to make sure it is VERY clear not to mess with the Shell-Lab directory

\begin{problem}
Using these commands, navigate to the \li{Shell-Lab} directory provided with this lab. We will use this directory for the remainder of the lab. Use the \li{ls} command to list the contents of this directory. NOTE: You will find a directory within this directory called \li{test} that is availabe for you to experiment with the concepts and commands found in this lab. The other files and directories are necessary for the exercises we will be doing, so take care not to modify them.
\end{problem}

\subsection*{Flags}
Most commands can be customized using flags. The \li{ls} command has dozens of optional flags. Table \ref{table:ls_flags} contains some of the most common flags for the \li{ls} command.

\begin{table}
\begin{tabular}{l|l} 
Flags & Description
\\ \hline 
\li{-a} & Do not ignore hidden files and folders \\ 
\li{-l} & List files and folders in long format \\ 
\li{-r} & reverse order while sorting \\
\li{-s} & print item name and size \\
\li{-t} & Sort output by date modified \\ 
\li{-R} & Print files and subdirectories recursively \\ 
\li{-S} & Sort by size \\ 
\end{tabular} 
\caption{Common flags of the \li{ls} command.}
\label{table:ls_flags} 
\end{table} 

Multiple flags can be combined as one flag. For example, if we wanted to list all the files in a directory in long format sorted by date modified, we would use \li{ls -a -l -t} or \li{ls -alt}. To view the reference manual for any command, use \li{man}. For example, to view the reference manual for the \li{ls} command, use \li{man ls}.

\subsection*{Other Useful Commands}
Table \ref{table:other_commands} contains a list commonly-used commands and their uses. Many of these commands will be needed throughout this lab and in any typical session with the Unix shell. After some of the commands are flags listed in square brackets that are worth exploring using \li{man}. We highly recommend experimenting with all these commands to become familiar with them. Remember you may freely experiment with these commands in the \li{test} directory.

\begin{lstlisting}
$ cd test
$ touch data.txt				# create new empty file data.txt
$ mkdir New						# create directory New
$ ls							# list items in test directory
New 	data.txt
$ cp data.txt New/				# copy data.txt to New directory
$ cd New/						# enter the New directory
$ ls							# list items in New directory
data.txt
$ mv data.txt new_data.txt		# rename data.txt new_data.txt
$ ls							# list items in New directory
 new_data.txt
$ cd ..							# Return to test directory
$ rm -rv New/					# Remove New directory and its contents
 removed 'New/data.txt'
 removed directory: 'New/'
$ clear							# Clear terminal screen
\end{lstlisting}

\begin{table}
\begin{tabular}{l|l} 
Commands & Description
\\ \hline 
\li{clear} & Clear the terminal screen \\
\li{cp file1 dir1} & Create a copy of \li{file1} and move it to \li{dir1} \\
\li{cp file1 file2} & Create a copy of file1 and name it file2 \\
\li{cp -r dir1 dir2} & Create a copy of dir1 and all its contents into \li{dir2} \\
\li{mkdir dir1} & Create a new directory named \li{dir1} \\
\li{mkdir -p path/to/new/dir1} & Create \li{dir1} and all intermediate directories \\
\li{mv file1 dir1} & Move \li{file1} to \li{dir1} \\
\li{mv file1 file2} & Rename \li{file1} as \li{file2} \\
\li{rm file1} & Delete \li{file1} [\li{-i}, \li{-v}] \\
\li{rm -r dir1} & Delete \li{dir1} and all items within \li{dir1} [\li{-i}, \li{-v}] \\
\li{touch file1} & Create an empty file named \li{file1} \\
\li{.} & Current directory \\
\li{..} & Parent directory \\
\li{\~} & Home directory \\
\li{/} & Root directory \\
\end{tabular} 
\caption{Other useful commands dealing with the file system.}
\label{table:other_commands} 
\end{table} 

\begin{problem}
Inside the \li{Shell-Lab} directory, delete the \li{Audio} folder along with all its contents. Create \li{Documents}, \li{Photos}, and \li{Python} directories.
\end{problem}

\subsection*{Wildcards}
As we are working in the file system, there will be times that we want to perform the same command to a group of similar files. For example, if you needed to move all text files within a directory to a new directory, the naive way to do this would be to move each text file individually. However, this same result can be achieved using \emph{wildcards}. We use wildcards as placeholder text.  We will use the \li{*} and \li{?} wildcards. The \li{*} wildcard represents any string and the \li{?} wildcard represents any single character. Though these wildcards can be used in almost every Unix command, they are particularly useful when dealing with files. See Table \ref{table:wildcards}

\begin{problem}
Within the \li{Shell-Lab} directory, there are many files. We will organize these files into directories. Using wildcards, move all the \li{.jpg} files to the \li{Photos} directory, all the \li{.txt} files to the \li{Documents} directory, and all the \li{.py} files to the \li{Python} directory. You will see a few other folders in the \li{Shell-Lab} directory. Do not move any of the files within these folders at this point.
\end{problem}

\subsection*{Pipes and Redirects}
Unix becomes even more versatile and powerful when you chain multiple commands together. This is accomplished using \emph{pipes}. Rather than printing the output of a command, the output is passed, or \emph{piped}, to the next command. Two commands are piped together using the \li{|} operator. To demonstrate the power of pipes, we will first introduce a few commands that allow us to view the contents of a file in Table \ref{table:print}

In the first example below, the \li{cat} command output is piped to \li{wc -l}. The \li{wc} command stands for word count. This command can be used to count words or lines. The \li{-l} flag tells the \li{wc} command to count lines. Therefore, this first example counts the number of lines in \li{assignments.txt}. 
In the second example below, the command lists the files in the current directory sorted by size in descending order. For details on what the flags in this command do, consult \li{man sort}.

\begin{table}
\begin{tabular}{l|l} 
Command & Description
\\ \hline 
\li{*.txt} & All files that end with \li{.txt}. \\
\li{image*} & All files that have \li{"image"} as the first 5 characters. \\
\li{*py*} & All files that contain \li{"py"} in the name. \\
\li{doc*.txt} & All files of the form \li{doc1.txt}, \li{doc2.txt}, \li{docA.txt}, etc. \\
\end{tabular} 
\caption{Common uses for wildcards.}
\label{table:wildcards} 
\end{table} 


\begin{lstlisting}
$ cd Shell-Lab/Files/Feb
$ cat assignments.txt | wc -l
9

$ ls -s | sort -nr
12 project3.py
12 project2.py
12 assignments.txt
 4 pics
total 40
\end{lstlisting}
%$

In the previous example, we pipe the contents of \li{assignments.txt} to \li{wc -l} using \li{cat}. When working with files specifically, it is better to use \emph{redirects}. The same output from the first example above can be achieved by running the following command:

\begin{lstlisting}
$ wc -l < assignments.txt
9
\end{lstlisting}
%$

\begin{table}
\begin{tabular}{l|l} 
Command & Description
\\ \hline 
\li{cat} & Print the contents of a file in its entirety \\ 
\li{more} & Print the contents of a file one page at a time \\
\li{less} & Like more, but you can navigate forward and backward \\
\li{head} & Print the first 10 lines of a file \\
\li{head -nK} & Print the first \li{K} lines of a file \\ 
\li{tail} & Print just the last 10 lines of a file \\
\li{tail -nK} & Print the last \li{K} lines of a file \\
\end{tabular} 
\caption{Commands for printing contents of a file}
\label{table:print} 
\end{table} 

If you are wanting to save the resulting output of a command to a file, use \li{>} or \li{>>}. The \li{>} operator will overwrite anything that may exist in the output file whereas \li{>>} will append the output to the end of the output file. For example, if we want to append the number of lines in \li{assignments.txt} to \li{word_count.txt}, we would run the following commmand:

\begin{lstlisting}
$ wc -l < assignements.txt >> word_count.txt
\end{lstlisting}
%$

Since \li{grep} is used to print lines matching a pattern, it is also very useful to use in conjunction with piping. For example, \li{ls -l | grep root} prints all files associated with the root user.

\begin{problem}
The \li{words.txt} file in the \li{Documents} directory contains a list of words that are not in alphabetical order. Write the number of words in \li{words.txt} and an alphabetically sorted list of words to \li{sortedwords.txt} using pipes and redirects. Save this file in the \li{Documents} directory. Try to accomplish this with a total of two commands or fewer.
\end{problem}


\section*{Searching the File System}
There are two powerful commands we use for searching through our directories. The \li{find} command is used to find files or directories in a directory hierarchy. The \li{grep} command is used to find lines matching a string. More specifically, we can use \li{grep} to find words inside files. We will provide a basic template in Table \ref{table:find} for using these two commands and leave it to you to explore the uses of the other flags.

\begin{problem}
In addition to the .jpg files you have already moved into the \li{Photots} folder, there are a few other .jpg files in a few other folders within the \li{Shell-Lab} directory. Find where these files are using the \li{find} command and move them to the \li{Photos} folder.
\end{problem}

\section*{Archiving and Compression}
In file management, the terms archiving and compressing are commonly used interchangeably. However, these are quite different. To archive is to combine a certain number of files into one file. The resulting file will be the same size as the group of files that were archived.
To compress is to take a file or group of files and shrink the file size as much as possible. The resulting compressed file will need to be extracted before being used.

The \li{ZIP} file format is the most popular for archiving and compressing files. If by chance the \li{zip} Unix command is not installed on your system, you can download it by running \li{sudo apt-get install zip}. Note that you will need to have administrative rights to download this package. To unzip a file, use \li{unzip}.

\begin{table}
\begin{tabular}{l|l} 
Command & Description
\\ \hline 
\li{find dir1 -type f -name "word"} &  Find all files in \li{dir1} with the name \li{"word"} \\ 
 & (\li{-type f} is for files \li{-type d} is for directories)\\
\li{grep -nr "word" dir1} & Find all occurances of \li{"word"} within the files inside \li{dir1} \\ 
 & (\li{-n} lists the line number and \li{-r} performs a recursive search)\\
\end{tabular} 
\caption{Commands using \li{find} and \li{grep}.}
\label{table:find} 
\end{table} 

\begin{lstlisting}
$ zip zipfile.zip doc?.txt
 adding: doc1.txt (deflated 87%)
 adding: doc2.txt (deflated 90%)
 adding: doc3.txt (deflated 85%)

# use -l to view contents of zip file
$ unzip -l zipfile.zip
Archive:  zipfile.zip
  Length     Date     Time    Name
---------  ---------- -----   ----
     5234  2015-08-26 21:21   doc1.txt
     7213  2015-08-26 21:21   doc1.txt
     3634  2015-08-26 21:21   doc1.txt
---------                     -------
    16081                     3 files
    
$ unzip zipfile.zip
  inflating: doc1.txt                
  inflating: doc2.txt                
  inflating: doc3.txt
\end{lstlisting}
%$

While the zip file format is more popular on the Windows platform, the \li{tar} utility is more common in the Unix environment. The following commands use \li{tar} to archive the files and \li{gzip} to compress the archive. 

Notice that all the commands below have the \li{-z}, \li{-v}, and \li{-f} flags. The \li{-z} flag calls for the \li{gzip} compression tool, the \li{-v} flag calls for a verbose output, and \li{-f} indicates the next parameter will be the name of the archive file. 

\begin{lstlisting}
# use -c to create a new archive
$ tar -zcvf docs.tar.gz doc?.txt 
doc1.txt
doc2.txt
doc3.txt

# use -t to view contents
$ tar -ztvf <archive>                    
-rw-rw-r-- username/groupname 5119 2015-08-26 16:50 doc1.txt
-rw-rw-r-- username/groupname 7253 2015-08-26 16:50 doc2.txt
-rw-rw-r-- username/groupname 3524 2015-08-26 16:50 doc3.txt

# use -x to extract
$ tar -zxvf <archive>                   
doc1.txt
doc2.txt
doc3.txt
\end{lstlisting}
%$

\begin{problem}
Archive and compress the files in the \li{Photos} directory using \li{tar} and \li{gzip}. Name the arhive \li{pics.tar.gz} and save it inside the \li{Photos} directory. Use \li{ls -l} to see how much the files were compressed in the process.
\end{problem}

\section*{Vim: A Terminal Text Editor}
Today many have become accustomed to having GUIs (Graphic User Interfaces) for all their applications. Before modern text editors (i.e. Microsoft Word, Pages for Mac, Google Docs) there were terminal text editors. 
These text editors are accessed, as the name suggests, from the terminal. Vim is one of the most popular terminal text editors. For a beginner, the learning curve may be intimidating, but as you become familiar with vim, it may become one of your preferred text editors for writing code.

One of the major philosophies of vim is to be able to keep your fingers on the keyboard at all times. There are countless keyboard shortcuts that allow you to navigate the file and execute commands without relying on a mouse, toolbars, or even the arrow keys.

In this section, we will go over the basics of navigation and a few of the most common commands. We will also provide a list of commands that interested readers are encouraged to research. 

It has been said that at no point does somebody finish learning vim. You will find that you will constantly be able to add something new to your arsenal.

\subsection*{Getting Started}
We start vim with the following command:

\begin{lstlisting}
$ vim my_file.txt
\end{lstlisting}
%$

When executing this command, if \li{my_file.txt} already exists, vim will open the file and we may begin editing the existing file. If \li{my_file.txt} does not exist, it will be created and we may begin editting the file. For our purposes, we want to create a new file.

You may notice if you start typing, the characters may or may not appear. This is because vim has multiple modes. When vim starts, we are placed in \emph{command mode}. We want to be in \emph{insert mode} to begin entering text. To enter insert mode from command mode, hit the \li{i} key. You should see \li{-- INSERT --} at the bottom of your terminal window. Now that we are in insert mode we may begin typing.

\begin{problem}
Create a new file in the \li{Documents} directory named \li{first_vim.txt}. Write at least multiple lines to this file. To return to command mode, hit the \li{Esc} key. 
\end{problem}

Insert mode should only be used for inserting text. It is not efficient to delete large portions of text while in insert mode. Try to get in the habit of leaving insert mode as soon as you are done adding the text you want to add.

\subsection*{Navigation}
We are accustomed to navigating GUI text editors using a mouse and arrow keys. In vim, we navigate using keyboard shortcuts while in command mode.

\begin{table}
\begin{tabular}{l|l} 
Command & Description
\\ \hline 
\li{a} & \textbf{a}ppend text after cursor \\
\li{A} & \textbf{A}ppend text to end of line \\
\li{o} & Begin a new line below the cursor \\
\li{O} & Begin a new line above the cursor \\
\li{s} & Substitute characters under cursor \\
\end{tabular} 
\caption{Commands for entering insert mode}
\label{table:viminsert} 
\end{table} 

\begin{problem}
Become accustomed to navigating in command mode using the following keys:
\begin{itemize}
\item k - up 
\item j - down
\item h - left 
\item l - right
\item w - beginning of next \textbf{w}ord
\item e - \textbf{e}nd of next word
\item b - \textbf{b}eginning of previous word
\item 0 - (zero) beginning of line
\item \$ - end of line
\item g10 - \textbf{g}o to line \textbf{10} 
\item gg - beginning of file
\item G - end of file
\end{itemize}
\end{problem}


\subsection*{Alternative Ways to Enter Insert Mode}
Hitting the \li{i} key is not the only way to enter insert mode. 
%As you have already seen, the \li{i} key enters insert mode at the current location. 
Alternative methods are described in Table \ref{table:viminsert}.

\subsection*{Visual Mode}
Visual mode allows you to select multiple characters. Among other things, we can use this to replace words with the \li{s} command, and we can select text to cut or copy.

\begin{problem}
While in command mode, enter visual mode by pressing the \li{v} key. Using the navigation keys discussed earlier, move the cursor to select a few words. Copy this text using the \li{y} key (stands for \textbf{y}ank). Return to command mode by pressing \li{Esc}. Move the cursor to where you would like to paste the text and press the \li{p} key to paste. Similarly, select text in visual mode and hit \li{d} to \textbf{d}elete the text and paste it somewhere else with the \li{p} key.
\end{problem}

\subsection*{Deleting Text in Command Mode}
As mentioned already, you should use insert mode only for adding new text. Deleting text is much more efficient and versatile in command mode. The \li{x} and \li{X} commands are used to delete single characters. The \li{d} command is always accompanied by another navigational command. See Table \ref{table:delete} for a few examples.

\begin{table}
\begin{tabular}{l|l} 
Command & Description
\\ \hline 
\li{dd} & delete line \\
\li{dl} & \textbf{d}elete \textbf{l}etter \\
\li{d4l} & \textbf{d}elete 4 \textbf{l}etters \\
\li{dw} & \textbf{d}elete \textbf{w}ord \\
\li{d2w} & \textbf{d}elete 2 \textbf{w}ords \\
\end{tabular} 
\caption{Commands for deleting in command mode}
\label{table:delete} 
\end{table}   

\subsection*{Quitting Vim}
We quit vim by first enter last line mode. We do this by pressing the \li{:} key. When exiting vim, we most will want to save and quit or quit without saving. To save and quit, run \li{:wq}. To save without quiting, run \li{:q!}.

\begin{problem}
Save and exit the file you have created.
\end{problem}

\subsection*{Customizing Vim}
If you wish to customize vim commands, this is accomplished using the \li{:map} command. For example, if you plan on using vim extensively, we highly recommend you remap \li{Esc} to a more convenient key sequence like \li{jk}. This would be accomplished by running \li{:map jk <Esc>}. You can save these customizations in the \li{vimrc} file. 

\subsection*{A Few Closing Remarks}
In the next lab, we will introduce how to access another machine through the terminal. Vim will be essential in this situation since GUIs will not be an option.

If you are interested in continuing to use vim, you may be interested in checking out \emph{gvim}. Gvim is a GUI that uses vim commands in a more traditional text editor window.

Also, in Table \ref{table:vim}, we have listed a few more commands that are worth exploring.
If you are interested in any of these features of vim, we encourage you to research these features further on the internet. Additionally, many people have published their \li{vimrc} file on the internet so other vim users can learn what options are worth exploring. It is also worth noting that we can use vim navigation commands in many other places in the shell. For example, try using the navigation commands when viewing the \li{man vim} page.

\begin{table}
\begin{tabular}{l|l} 
Command & Description
\\ \hline 
\li{:help} & view vim docs \\
\li{cw} & \textbf{c}hange \textbf{w}ord \\
\li{u} & undo \\
\li{Ctrl-R} & redo \\
\li{.} & Repeat the previous command \\
\li{*} & find next occurrence of word under cursor \\
\li{#} & find previous occurrence of word under cursor \\
\li{/str} & find \li{"str"} in file \\
\li{n} & find next match \\
\li{N} & find previous match \\
\end{tabular} 
\caption{Commands for entering insert mode}
\label{table:vim} 
\end{table}  

