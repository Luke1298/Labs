\lab{Exceptions and I/O}{OOP2}
\label{lab:OOP2}

\objective{%Understanding Object Oriented Programming in Python reveals important insights about how Python works internally. For example,
Python uses \emph{Exceptions} to handle errors.
Understanding and utilizing the exception class hierarchy is an important skill for regulating program usage and informing the programmer of problems.
%Every programming language also has a way of interacting with exterior files.
Python also has a powerful \li{file} object that allows data to be read in from and out to files easily.
Understanding how this object's class attributes and methods is essential to data manipulation and analysis.
In this final introductory lab, we explore exceptions and file I/O protocol from the view of Object Oriented Programming.}

\section*{Exceptions}

Every programming language has a formal way of handling errors.
In Python, we \li{raise} and handle \emph{Exceptions}.
There are different kinds of exceptions, each with its appropriate usage.
In Python, as in most languages, exceptions are organized into a class hierarchy.
A complete list of Python exceptions can be found \href{https://docs.python.org/2/library/exceptions.html}{here}.
The following code displays some examples of common exceptions:

\begin{lstlisting}
# A 'NameError' exception indicates that a nonexistant name was used.
>>> print(x)
<<Traceback (most recent call last):
  File "<stdin>", line 1, in <module>
NameError: name 'x' is not defined>>

# An 'AttributeError' exception indicates that a nonexistant method or
# attribute was called on some object.
>>> [1, 2, 3].fly()
<<Traceback (most recent call last):
  File "<stdin>", line 1, in <module>
AttributeError: 'list' object has no attribute 'fly'>>

# A 'SyntaxError' exception indicates bad Python syntax.
>>> def myFunction(a, b)             # forgot the colon!
<<  File "<stdin>", line 1
    def myFunction(a, b)
                       ^
SyntaxError: invalid syntax>>
\end{lstlisting}

Many exceptions, like the ones listed above, are due to coding mistakes and typos.
Exceptions can also be used programmatically and intentionally to indicate a problem to the user or programmer.
To raise an exception, use the keyword \li{raise}, followed by the name of the exception class.
As soon as an exception is raised, the program stops running unless the exception is handled properly.

Exception objects can also be initialized with any number of arguments.
These arguments are stored in the exception object as a class attribute called \li{args}, and also serve as the string representation of the object.
Most commonly, we provide a simple message with information about the error.

\begin{lstlisting}
# Raise a generic exception, without an error message.
>>> if 7 is not 7.0:
...     raise Exception
... 
<<Traceback (most recent call last):
  File "<stdin>", line 2, in <module>
Exception>>

# Without a message, the error traceback can be rather unhelpful.
# Now raise a specific type of exception, with an error message included.
>>> for x in range(10):
...     if x > 5:
...         raise ValueError("'x' should not exceed 5.")
...     print(x),
... 
0 1 2 3 4 5
<<Traceback (most recent call last):
  File "<stdin>", line 3, in <module>
ValueError: 'x' should not exceed 5.>>
\end{lstlisting}

% Problem 1: Raising Exceptions 101
\begin{problem}
Python functions do not require specific types for each parameter.
Sometimes, however, it is important to validate functional inputs before they are utilized.
Consider the following function.
\begin{lstlisting}
def my_func(a, b, c, d, e):
    print("The first argument is " + a)
    x = sum([b, c])
    y = d + e
    return a, x, y
\end{lstlisting}
Modify \li{my_func} so that it raises a \li{TypeError} if:
\begin{enumerate}
\item $a$ is not a string (of type \li{str}),
\item $b$ or $c$ is not a numerical type (\li{int}, \li{float}, \li{long}, or \li{complex}), or
\item $d$ and $e$ are not the same type.
\end{enumerate}
Include an informative error message with each exception.

(Hint: investigate the \li{type} and \li{isinstance} built-in funcitons.)
\end{problem}

\subsection*{Handling Exceptions}

To prevent an exception from stopping the program, it must be handled by
putting the problematic lines of code in a \li{try} block.
An \li{except} block then follows.
If an exception is thrown, the compiler exits the \li{try} block and enters the \li{except} block.

\begin{lstlisting}
# the 'try' block should hold any lines of code that might raise an exception.
>>> try:
...     raise Exception("for no reason.")
...     print("No exception raised")
... # the 'except' block is entered into after the exception is thrown
... except:
...     print("Exception raised!")
... print("Got to the end.")
... 
Exception raised!
Got to the end.                         # Compilation continued as usual.
\end{lstlisting}

Exceptions can be caught generally or by class.
Having \li{except} by itself will catch \emph{any} exception raised in the \li{try} block, but this approach is usually too broad and is typically considered bad coding practice.
Instead, be specific about the kind (or kinds) of exceptions you expect to encounter.

Additionally, an exception that is caught by an \li{except} statement can be used within the \li{except} block if it is declared using the keyword \li{as}.

\begin{lstlisting}
# Catch a specific exception class.
>>> try:
...     bad = 100 / 0
... except ZeroDivisionError as e:
...     print(e)
... 
<<integer division or modulo by zero>>

# Here a different exception is raised than the one in the except statement.
>>> try:
...     1 + 'a' + 2 + 'b' + 3
... except ValueError as e:
...     print(e)
<<Traceback (most recent call last):
  File "<stdin>", line 2, in <module>
TypeError: unsupported operand type(s) for +: 'int' and 'str'>>

>>> try:
...     import magic
... except ImportError as e:
...     print("Sorry!", e)
... 
Sorry! No module named magic
\end{lstlisting}

Multiple kinds of exceptions can be caught by a single except statement using a parenthesized list of exceptions.
There can also be more than one \li{except} statements that correspond to a single \li{try} statement.
An \li{else} statement can also be attached after \li{except} statements
%This control flow idea is similar to \li{if}, \li{elif}, and \li{else} blocks.
Last of all, another optional block can be added with the header \li{finally}.
The code in a \li{finally} block executes after either the try or any subsequent block completes, regardless of whether or not an exception was actually raised.
% TODO: Combine these last two blocks.

\begin{lstlisting}
# Catch two kinds of exceptions in the same except statement.
>>> try:
...     print("Started in the try block.")
...     raise ValueError("This is a ValueError!")
...     print("Got the the end of the try block.")
... except (ValueError, TypeError, ArithmeticError) as e:
...     print("An exception was raised. Error: ", e)
... else:
...     print("No exceptions were raised.")
... finally:
...     print("Finished with the finally block.")
<<Started in the try block.
Made it to the except block. Error: This is a ValueError!
Finished with the finally block.>>

# Equivalently, catch each kind of exception with a separate except statement.
# Here, no exception is raised and the except block is skipped.
>>> try:
...     print("Started in the try block.")
...     raise TypeError("This is a TypeError!")
... except ValueError as e:
...     print("Bad value:", e)
... except TypeError as e:
...     print("Bad type:", e)
... except ArithmeticError as e:
...     print("Bad math:", e)
... else:
...     print("No exceptions were raised.")
... finally:
...     print("Finished with the finally block.")
<<Started in the try block.
Bad type: This is a TypeError!
Finished with the finally block.>>

# Note that the else block executes if no exceptions are raised.
>>> try:
...     print("Started in the try block.")
... except (ValueError, TypeError, ArithmeticError) as e:
...     print("An exception was raised. Error:", e)
... else:
...     print("No exceptions were raised.")
... finally:
...     print("Finished with the finally block.")
<<Started in the try block.
No exceptions were raised.
Finished with the finally block.>>
\end{lstlisting}

Complete documentation on raising and handling exceptions can be found at \url{https://docs.python.org/2/tutorial/errors.html}.

%Chaining Exceptions in Python 3...

\begin{problem}
A \li{KeyboardInterrupt} is a special exception that can be triggered at any time by entering \li{ctrl c} (on most systems) in the keyboard.
Usually a \li{KeyboardInterrupt} is used to manually escape faulty code that runs forever, but they can also be used intentionally to truncate a process.

Consider the following function.
\begin{lstlisting}
def forever():
    iters = 0
    while True:
        iters += 1
    return i
\end{lstlisting}
Modify \li{forever} so that after ten seconds, a \li{RuntimeError} is raised to exit the loop (Hint: use the \li{time} module).
Catch the exception, print ``Process terminated'', and return \li{iters}.

If a \li{KeyboardInterrupt} is raised before time runs out, catch the exception, print ``Process interrupted'', and return \li{iters}.
\end{problem}

\begin{comment}
In [2]: def forever():
   ...:     i = 0
   ...:     start = time()
   ...:     try:
   ...:         while True:
   ...:             i += 1
   ...:             if i % 100 == 0:
   ...:                 if time() - start > 10:
   ...:                     raise RuntimeError
   ...:     except KeyboardInterrupt:
   ...:         print("Process interrupted")
   ...:     except RuntimeError:
   ...:         print("Process terminated")
   ...:     finally:
   ...:         return i
\end{comment}

\begin{problem}
Function wrapper problem.
\end{problem}

\subsection*{The Exception Hierarchy}

A \li{ValueError} is a \li{StandardError}, which is an \li{Exception}.
\begin{lstlisting}
try:
    raise ValueError("This is a test")
    print("Checkpoint 1")
except Exception as e:
    print e

# But...

try:
    raise Exception("This is another test")
    print("Checkpoint 2")
except ValueError as e:
    print e
\end{lstlisting}

The complete Exception hierarchy is found at \url{https://docs.python.org/2/library/exceptions.html#exception-hierarchy}.

\subsection*{Custom Exceptions}

Custom exception are made by writing a class that inherits from some existing exception (the \li{Exception} class is the most generic).
They can be very useful for handling problems that you expect, but that the compiler won't catch.

\begin{problem}
Create a custom class of Exceptions called \li{BadDataError}.
Have it inherit from the basic \li{Exception} class.
\end{problem}

\section*{Input and Output}

Python has a useful \li{file} object which acts as an interface to all kinds of different file streams.
The built-in function \li{open} creates a \li{file} object.

\begin{lstlisting}
>>> myfile = open('filename.txt','r')   # Open the file with read-only access.
>>> for line in myfile:                 # Print out each line of the file.
...     print line
...
>>> myfile.close()                      # Close the file connection.
\end{lstlisting}

The \li{open} command accepts up to three arguments: the filename, mode, and buffering options.
The mode determines the kind of access to use when opening the file.
Possible mode strings are:
\begin{description}
\item \li{'r'} Opens a file for read-only access.
This is the default mode.
\item \li{'w'} Opens a file for write-only access.
This mode creates the file if it doesn't already exist, and overwrites \textbf{everything} in the file if it does exist.
\item \li{'a'} Opens a file for appending.
This mode is like write-only, but instead of overwriting existing data any new data is appended to the end of the file.
This mode also creates a new file if it doesn't already exist.
\end{description}

\subsection*{The \li{with} Statement}

If your file cannot be opened for any reason, an exception is raised (usually an \li{IOException}) and the file object is not initialized.
Files can be opened and closed safely with a careful \li{try} \li{except} \li{else} control flow.

The keyword \li{with} can also be used in conjunction with an \li{open} statement to create an indented block in which the file is open.
Upon exiting the block, the file is closed safely and automatically.
This is the preferred method when a file only needs to be accessed briefly, but it is not quite as informative as the previous approach.

\begin{lstlisting}
>>> try:
...     myfile = open('in.txt', 'r')    # Open the file with read-only access.
...     contents = myfile.readlines()   # Read in the content by line.
... except IOError as e:                # Check for errors.
...     print("File error:", e)
... else:
...     myfile.close()                  # Explicitly close the file.

# Equivalently, use a 'with' statement to take care of errors.
>>> with open('in.txt', 'r') as myfile:     # Open the file using 'with'.
...    contents = myfile.readlines()        # Read in the content by line.
...                                         # The file is closed automatically.
\end{lstlisting}

\subsection*{Methods for Reading and Writing}

Every file object has several attributes and methods.
Some of the most notable are described in Table \ref{table:fileattribs}.
% Talk about the so-called file pointer, an imaginary cursor in the open file.

\begin{table}
\begin{tabular}{|l|l|}
\hline
Attribute & Description \\
\hline
\li{closed} & True if file object is closed. \\
\li{mode} & The access mode used to open the file object. \\
\li{name} & The name of the file. \\
\hline
\hline
Method & Description \\
\hline
\li{close()} & Flush any delayed writes and close the file object. \\
\li{read()} & Read the next string of the file. \\
\li{readline()} & Read a line of the file. \\
\li{readlines()} & Read lines of the file until the end of file (returned as a \li{list}). \\
\li{write()} & Write a string to the file. \\
\li{writelines()} & Write a sequence of strings to the file (input a \li{list}). \\
\hline
\end{tabular}
\caption{File object attributes and methods.}
\label{table:fileattribs}
\end{table}

Talk about \li{read}, \li{readlines}, \li{write}, \li{writelines}, \li{next}, \li{tell}, and the cursor thing.

Only strings can be written to files.
To write a non-string type, first cast it as a string using \li{str}.
Again we present two equivalent methods.
\begin{lstlisting}
>>> try:
...     outfile = open('out.txt', 'w')  # Open the file with write-only access.
...     for i in xrange(10):
...         outfile.write(str(i**2))        # Write to the file.
... except IOError as e:
...     print("Problem writing to file:", e)
... finally:
...     outfile.close()                     # Explicitly close the file.

# The flow is much less cumbersome using a 'with' statement.
>>> with open('out.txt', 'w') as outfile:   # Open the file using 'with'.
...     for i in xrange(10):                
...         outfile.write(str(i**2))        # Write some strings to the file.
...
>>> outfile.closed                          # The file is closed automatically.
True
\end{lstlisting}

\subsection*{String Formatting}

Talk about the is stuff (\li{isalnum}/\li{isspace}), formatting (\li{capitalize}/\li{lower}/\li{swapcase}), and \li{split} and \li{strip}. 

\begin{problem}
Make a cool class for reading in data from a file and writing it out to other files in wonky ways.
\end{problem}
