\lab{CVXOPT}{CVXOPT}
\label{lab:Optimization 2}
\objective{Introduce some of the basic optimization functions available in the CVXOPT package}

You can learn about more about CVXOPT at

\url{http://abel.ee.ucla.edu/cvxopt/documentation/}.

\section*{Linear Programs}

%%Cvxopt has linear program solver and can implement integer programming through the Gnu Linear Programming Kit, glpk.
CVXOPT is a package of Python functions and classes for the purpose of convex optimization.
In this lab we will focus on linear and quadratic programming.
A \emph{linear program} is a linear constrained optimization problem. Such a problem can be stated in several
different forms, one of which is
\begin{align*}
\text{minimize}\qquad &c^Tx \\
\text{subject to}\qquad &Gx + s = h\\
&Ax = b \\
 &s \geq 0.
\end{align*}
This is the formulation used by CVXOPT.
In this formulation, we require that the matrix $A$ has full row rank,
and that the block matrix $[G \quad A]^T$ has full column rank.

Note that the constraint $Gx +s = h$ includes the term $s$, which is not part of the objective
function, and is known as the \emph{slack variable}. Since $s  \geq 0$, the constraint
$Gx + s = h$ is equivalent to $Gx \leq h$.

The corresponding \emph{dual program} for the above linear program has the form
\begin{align*}
\text{maximize}\qquad &-h^Tz - b^Ty \\
\text{subject to}\qquad &G^Tz + A^Ty + c = 0\\
 &z \geq 0.
\end{align*}
CVXOPT provides functions to solve both the original (\emph{primal}) linear program and its dual program.

Consider the following example.
\begin{align*}
\text{minimize}\qquad &-4x_1-5x_2 \\
\text{subject to}\qquad &x_1+2x_2 \leq 3 \\
	        &2x_1+x_2 \leq 3 \\
		&x_1, x_2 \geq 0
\end{align*}
The final two constraints, $x_1, x_2 \geq 0$, need to be adjusted to be $\leq$ constraints.
This is easily done by multiplying by $-1$, resulting in the constraints $-x_1, -x_2 \leq 0$.
If we define
\[
G = \begin{bmatrix}
  1 & 2\\
  2 & 1\\
  -1 & 0\\
  0 & -1
\end{bmatrix}
\]
and
\[
h = \begin{bmatrix}
  3\\
  3\\
  0\\
  0
\end{bmatrix},
\]
then we can express the constraints compactly as
\[
Gx \leq h,
\]
where
\[
x = \begin{bmatrix}
  x_1\\
  x_2
\end{bmatrix}.
\]
By adding a slack variable $s$, we can write our constraints as
\[
Gx + s = h,
\]
which matches the form discussed above. In the case of this particular example, we ignore the extra constraints
\[
Ax = b,
\]
since we were given no equality constraints.

Now we proceed to solve the problem using CVXOPT.
We need to initialize the arrays $c$, $G$, and $h$, and then pass them to the appropriate function.
CVXOPT uses its own data type for arrays and matrices, and while similar to the NumPy array, it
does have a few differences, especially when it comes to initialization.
Below, we initialize CVXOPT matrices for $c$, $G$, and $h$.

\begin{lstlisting}
>>> from cvxopt import matrix
>>> c = matrix([-4., -5.])
>>> G = matrix([[1., 2., -1., 0.],[2., 1., 0., -1.]])
>>> h = matrix([ 3., 3., 0., 0.])
\end{lstlisting}
Observe that CVXOPT matrices are initialized column-wise rather than row-wise (as in the case of NumPy).

Alternatively, we can initialize the arrays first in NumPy (a process with which you should be familiar),
and then simply convert them to the CVXOPT matrix data type:
\begin{lstlisting}
>>> import numpy as np
>>> c = np.array([-4., -5.])
>>> G = np.array([[1., 2.],[2., 1.],[-1., 0.],[0., -1]])
>>> h = np.array([3., 3., 0., 0.])

>>> #Now convert to CVXOPT matrix type
>>> c = matrix(c)
>>> G = matrix(G)
>>> h = matrix(h)
\end{lstlisting}
Use whichever method is most convenient. Note that we made sure the entries in the matrices are floats.

Having initialized the necessary objects, we are now ready to solve the problem.
We will use the function \li{solvers.lp}, and we simply need to pass $c$, $G$, and $h$ as arguments.
\begin{lstlisting}
>>> from cvxopt import solvers
>>> sol = solvers.lp(c, G, h)
     pcost       dcost       gap    pres   dres   k/t
 0: -8.1000e+00 -1.8300e+01  4e+00  0e+00  8e-01  1e+00
 1: -8.8055e+00 -9.4357e+00  2e-01  1e-16  4e-02  3e-02
 2: -8.9981e+00 -9.0049e+00  2e-03  1e-16  5e-04  4e-04
 3: -9.0000e+00 -9.0000e+00  2e-05  1e-16  5e-06  4e-06
 4: -9.0000e+00 -9.0000e+00  2e-07  1e-16  5e-08  4e-08
Optimal solution found.
>>> print sol['x']
[ 1.00e+00]
[ 1.00e+00]
>>> print sol['primal objective']
-8.99999981141
\end{lstlisting}
The function \li{solvers.lp} returns a dictionary containing useful information.
For the time being, we will focus just on the values of $x$ and the primal objective value (i.e. the minimum value achieved by
the objective function).
\begin{problem}
Solve the following convex optimization problem
\begin{align*}
\text{minimize } &2x_1+x_2+3x_3 \\
\text{subject to } &x_1+2x_2 \geq 3 \\
	        &2x_1+x_2+3x_3 \geq 10 \\
		&x_1 \geq 0 \\
		&x_2 \geq 0 \\
		&x_3 \geq 0
\end{align*}
Report the values of $x$ and the primal objective function that you obtain.
Remember to make the necessary adjustments so that all inequality constraints $\leq$ rather than $\geq$.
\end{problem}

\section*{The Transportation Problem}

Consider the following transportation problem:
A piano company needs to transport thirteen pianos from their three  supply centers (denoted by 1, 2, 3) to two demand centers (4, 5).
Transporting a piano from a supply center to a demand center incurs a cost, listed in Table \ref{tab:cost}.
The company wants to minimize shipping costs for the pianos while meeting the demand.
How many pianos should each supply center send each demand center?

\begin{table}[h]
\centering
\begin{tabular}{|c|c|}
Supply Center & Number of pianos available\\
\hline
1 & 7\\
2 & 2\\
3 & 4\\
\end{tabular}

\caption{Number of pianos available at each supply center}
\label{tab:supply}
\end{table}

\begin{table}[h]
\centering
\begin{tabular}{|c|c|}
Demand Center & Number of pianos needed\\
\hline
4 & 5\\
5 & 8\\
\end{tabular}

\caption{Number of pianos needed at each demand center}
\label{tab:demand}
\end{table}

\begin{table}[h]
\centering
\begin{tabular}{|c|c|c|c|}
Supply Center & Demand Center & Cost of transportation & Number of pianos\\
\hline
1 & 4 & 4 & p\\
1 & 5 & 7 & q\\
2 & 4 & 6 & r\\
2 & 5 & 8 & s\\
3 & 4 & 8 & t\\
3 & 5 & 9 & u\\
\end{tabular}
\caption{Cost of transporting one piano from supply center to demand center}
\label{tab:cost}
\end{table}

The variables $p,q,r,s,t,$ and $u$ must be nonnegative and satisfy the following three supply and two demand constraints:
\begin{align*}
p + q  &= 7\\
r + s  &= 2\\
t + u  &= 4\\
p + r + t &= 5\\
q + s + u &= 8
\end{align*}

The objective function is the number of pianos shipped from each location multiplied by the respective cost:
\[
4p + 7q + 6r + 8s + 8t + 9u.
\]

There a several ways to solve this linear program. We want our answers to be integers, and this added constraint turns out to be an NP-hard problem
in general. There is a whole field devoted to dealing with integer constraints, called integer linear programming, which is beyond the scope of this lab.
Fortunately, we can treat this particular problem as a standard linear program and still obtain integer solutions.

Here, $G$ and $h$ constrain the variables to be non-negative.
Because CVXOPT uses the format $Gx \leq h$, we see that $G$ must be a $6 \times 6$ identity matrix multiplied by $-1$, and
$h$ is just a column vector of zeros.
The matrices $A$ and $b$ represent the supply and demand constraints, since these are equality constraints.
Try initializing these arrays and solving the linear program by entering the code below. (Notice that
we pass more arguments to \li{solvers.lp} since we have equality constraints.)
\begin{lstlisting}
>>> c = matrix([4., 7., 6., 8., 8., 9])
>>> G = matrix(-1*np.eye(6))
>>> h = matrix(np.zeros(6))
>>> A = matrix([[1., 0., 0., 1., 0.],
                [1., 0., 0., 0., 1.],
                [0., 1., 0., 1., 0.],
                [0., 1., 0., 0., 1.],
                [0., 0., 1., 1., 0.],
                [0., 0., 1., 0., 1.]])
>>> b = matrix([7., 2., 4., 5., 8])
>>> sol = solvers.lp(c, G, h, A, b)
     pcost       dcost       gap    pres   dres   k/t
 0:  8.9500e+01  8.9500e+01  2e+01  4e-17  2e-01  1e+00
Terminated (singular KKT matrix).
>>> print sol['x']
[ 3.00e+00]
[ 4.00e+00]
[ 5.00e-01]
[ 1.50e+00]
[ 1.50e+00]
[ 2.50e+00]
>>> print sol['primal objective']
89.5
\end{lstlisting}
Notice that some problems occurred. First, CVXOPT alerted us to the fact that the algorithm terminated prematurely (due to a singular matrix).
Further, the solution that was obtained does not consist of integer entries.

So what went wrong? Recall that the matrix $A$ is required to have full row rank, but we can easily see that the rows of $A$
are linearly dependent. We rectify this by converting some of the equality constraints into \emph{inequality} constraints, so that
the remaining equality constraints define a new matrix $A$ with linearly independent rows.

Rather than fuss about which equality
constraints to convert into inequality constraints, let us simply convert all of the equality constraints.
This is done as follows. Suppose we have the equality constraint
\[
x + 2y - 3z = 4.
\]
This is equivalent to the pair of inequality
constraints
\begin{align*}
x + 2y - 3z &\leq 4, \\
x + 2y - 3z &\geq 4.
\end{align*}
Of course, we require only $\leq$ constraints, so we obtain the pair
of constraints
\begin{align*}
x + 2y - 3z &\leq 4, \\
-x - 2y + 3z &\leq -4.
\end{align*}

Apply this process to each of the equality constraints. You will obtain a new matrix $G$ with several additional rows (to account for the new inequality
constraints), and a new vector $h$, also with more entries. Having done this, we no longer have equality constraints $A$ and $b$, so these can be ignored.
\begin{problem}
Solve the problem by converting all equality constraints into inequality constraints.
Report the optimal values for $x$ and the primal objective function.
\end{problem}

\begin{comment}
\section*{Example}

Why are all of the terms in $G$ and $h$ non-positive?

\begin{lstlisting}
>>> from cvxopt import matrix, solvers
>>> G = matrix([ [-1., 0., 0., -1., 0.,  -1., 0., 0., 0., 0., 0.],
             [-1., 0., 0., 0., -1.,  0., -1., 0., 0., 0., 0.],
             [0., -1., 0., -1., 0.,  0., 0., -1., 0., 0., 0.],
             [0., -1., 0., 0., -1.,  0., 0., 0., -1., 0., 0.],
             [0., 0., -1., -1., 0.,  0., 0., 0., 0., -1., 0.],
             [0., 0., -1., 0., -1.,  0., 0., 0., 0., 0., -1.] ])

>>> h = matrix([-7., -2., -4., -5., -8.,  0., 0., 0., 0., 0., 0.,])
>>> c = matrix([4., 7., 6., 8., 8., 9])
>>> sol = solvers.lp(c,G,h)
>>> print sol['x']
>>> print sol['primal objective']
\end{lstlisting}

Another method is to use an integer linear program.
Cvxopt is configured to work with  Gnu, which does have an integer linear program.
It will work with either of the methods above.

\textbf{Example}

glpk.ilp returns a tuple.
The first entry describes the optimality of the result, while the second gives the $x$ values.

\begin{lstlisting}
>>> from cvxopt import matrix, solvers, glpk
>>> G = matrix([ [-1., 0., 0., -1., 0.,  -1., 0., 0., 0., 0., 0.],
             [-1., 0., 0., 0., -1.,  0., -1., 0., 0., 0., 0.],
             [0., -1., 0., -1., 0.,  0., 0., -1., 0., 0., 0.],
             [0., -1., 0., 0., -1.,  0., 0., 0., -1., 0., 0.],
             [0., 0., -1., -1., 0.,  0., 0., 0., 0., -1., 0.],
             [0., 0., -1., 0., -1.,  0., 0., 0., 0., 0., -1.] ])

>>> h = matrix([-7., -2., -4., -5., -8.,  0., 0., 0., 0., 0., 0.,])
>>> o = matrix([4., 7., 6., 8., 8., 9])
>>> sol = glpk.ilp(o,G,h)
>>> print sol[1]
\end{lstlisting}

or
\begin{lstlisting}
>>> from cvxopt import matrix, solvers, glpk
>>> G = matrix([ [-1., 0., 0., 0., 0., 0.],
             [0., -1., 0., 0., 0., 0.],
             [0., 0., -1., 0., 0., 0.],
             [0., 0., 0., -1., 0., 0.],
             [0., 0., 0., 0., -1., 0.],
             [0., 0., 0., 0., 0., -1.] ])

>>> h = matrix([ 0., 0., 0., 0., 0., 0.,])
>>> o = matrix([4., 7., 6., 8., 8., 9])
>>> A = matrix([ [1., 0., 0., 1., 0.],
             [1., 0., 0., 0., 1.],
             [0., 1., 0., 1., 0.],
             [0., 1., 0., 0., 1.],
             [0., 0., 1., 1., 0.],
             [0., 0., 1., 0., 1.] ])
>>> b = matrix([7., 2., 4., 5., 8])
>>> sol = glpk.ilp(o,G,h,A,b)
>>> print sol[1]
\end{lstlisting}

\textbf{Problem 2}
Choose one of these methods and compare the optimal values for the integer linear program to the result you received above.

\textbf{Problem 3}
Create the dual problem for the linear program and solve.
Compare your answer to the dual value cvxopt returned.
\end{comment}

\section*{Quadratic Programming}

Quadratic programming is similar to linear programming except that the objective function is quadratic rather
than linear. However, the constraints, if there are any, are still of the same form.
Thus $G, h, A$, and $b$ are optional. The formulation that we will use is
\begin{lstlisting}[mathescape]
minimize $\frac{1}{2}x^TQx + p^Tx$
subject to $Gx\leq h$
	        $Ax = b$.
\end{lstlisting}
\begin{align*}
\text{minimize}\qquad &\frac{1}{2}x^TQx + p^Tx \\
\text{subject to}\qquad &Gx \leq h\\
 &Ax = b,
\end{align*}
where $Q$ is a positive semidefinite symmetric matrix.
In this formulation, we require again that $A$ have full row rank, and that the block matrix
$[P \quad G \quad A]^T$ have full column rank.

As an example, let us minimize the quadratic function
\[
f(x,y) = 2x^2 +2xy + y^2 +x -y.
\]
Note that there are no constraints, so we only need to initialize the matrix $Q$ and the vector $p$.
\begin{lstlisting}
>>> Q = matrix([[4., 2.], [2., 2.]])
>>> p = matrix([1., -1.])
>>> sol=solvers.qp(Q, p)
>>> print(sol['x'])
[-1.00e+00]
[ 1.50e+00]
>>> print sol['primal objective']
-1.25
\end{lstlisting}
Building the matrix $Q$ from the function $f$ is straightforward. The coefficients for each squared term are doubled and then placed on the main diagonal of $Q$ (so the term $2x^2$ yields $4$ in the upper left entry of
$Q$, and the term $y^2$ yields $2$ in the lower right entry).
The coefficient of each mixed term appears twice according to the row and column corresponding to the two
variables in the mixed term (so the term $2xy$ yields a $2$ placed in the first row, second column and in
the second row, first column). 

\begin{problem}
Find the minimizer and minimum of
\begin{equation*}
g(x,y,z) = \frac{3}{2}x^2 +2xy + xz+ 2y^2 +2yz+\frac{3}{2}z^2+3x + z
\end{equation*}
\begin{comment}
\begin{equation}
f(x) = \frac{1}{2}x^TQx - x^Tp
\end{equation}
where

\begin{center}
$Q =
\begin{bmatrix}
3 & 2 & 1\\
2 & 4 & 2\\
1 & 2 & 3\\
\end{bmatrix}
$
and $p =
\begin{bmatrix}
3\\
0\\
1\\
\end{bmatrix}
$
\end{center}
\end{comment}

\end{problem}

\section*{Allocation Models}
Allocation models lead to simple linear programs. An allocation model seeks to allocate a valuable resource among competing needs. The following example is taken from ``Optimization in Operations Research" by Ronald L. Rardin. %%pg 132

The U.S. Forest service has used an allocation model to deal with the task of managing national forests. 
The model begins by dividing the land into a set of analysis areas. Several land management policies (also 
called prescriptions) are then proposed and evaluated for each area. 
An \emph{allocation} is an assignment of land (in acreage) in each analysis area to each of the 
prescriptions for that analysis area.
We seek to find the best possible allocation, subject to forest-wide restrictions on land use.

The file \li{ForestData.npy} contains data for a fictional national forest (you can also find the data
in Table \ref{tab:forest}). There are 7 areas of analysis and 3 prescriptions for each of them. 
The first column is the area of analysis $i$. The second column is size of the analysis area (in thousands of acres), denoted $s_i$. The third column is a prescription number denoted $j$. The forth column is net present value (NPV) per acre of all uses in area $i$ under prescription $j$, denoted $p_{i,j}$. The fifth column is protected timber yield (in board feet per acre) in area $i$ under prescription $j$, denoted $t_{i,j}$. The sixth column is protected grazing capability (in animal unit months per acre) for area $i$ under prescription $j$, denoted
$g_{i,j}$. The seventh and last column is the wilderness index rating (0 to 100) for area $i$ under prescription $j$, denoted $w_{i,j}$. Let $x_{i,j}$ be the amount of land in area $i$ allocated to prescription $j$.

\begin{table}[h]
\centering
    \begin{tabular}{c c c c c c c}
&&&Forest Data&&& \\
\hline
Analysis & Acres &Prescrip-&NPV,&Timber,&Grazing,&Wilderness \\
Area,&(1000)'s &tion&(per acre) &(per acre)&(per acre)& Index,\\
$i$ &$s_i$&$j$& $p_{i,j}$ & $t_{i,j}$&$g_{i,j}$&$w_{i,j}$ \\\hline
1&	75	&1	&503	&310	&0.01&	40\\
&&		2&	140&	50&	0.04	&80\\
&&		3&	203&	0&	0&	95\\ \hline
2&	90&	1	&675&	198&	0.03&	55\\
&&		2&	100&	46&	0.06&	60\\
&&		3&	45&	0&	0&	65\\ \hline
3&	140&	1	&630&	210	&0.04&	45\\
&&		2&	105&	57&	0.07&	55\\
&&		3&	40	&0&	0&	60\\ \hline
4	&60&	1&	330&	112&	0.01&	30\\
&&		2	&40&	30&	0.02&	35\\
&&		3&	295&	0&	0	&90\\ \hline
5	&212&	1	&105	&40	&0.05&	60\\
&&		2	&460&	32	&0.08&	60\\
&& 3	&120&0&	0	&70\\ \hline
6	&98	&1	&490	&105	&0.02	&35\\
&&		2&	55	&25	&0.03	&50\\
&&		3	&180	&0	&0	&75\\ \hline
7&	113&	1	&705	&213&	0.02	&40\\
&&		2&	60	&40	&0.04&	45\\
&&		3	&400	&0	&0	&95\\
\hline
    \end{tabular}
\label{tab:forest}
\end{table}

Under this notation, and allocation is just a vector consisting of the $x_{i,j}$'s. For this particular
example, the allocation vector is of size $7\cdot 3 = 21$. 
Our goal is to find the allocation vector that maximizes net present value, while producing at least 40 million
board feet of timber, at least 5 thousand animal unit months of grazing, and keeping the average wilderness index at least 70.

Of course, the allocation vector is also constrained to be nonnegative, and all the land must be allocated 
precisely. 

Note that since acres are in thousands we also divide out 1000 from the constraints of timber and animals months of grazing. We can summarize our problem as follows:
\begin{align*}
\text{maximize } &\sum\limits_{i=1}^7 \sum\limits_{j=1}^3 p_{i,j}x_{i,j} \\
\text{subject to } &\sum\limits_{j=1}^3 x_{i,j} = s_i  \text{ for } i=1,..,7 \\
	        &\sum\limits_{i=1}^7 \sum\limits_{j=1}^3 t_{i,j}x_{i,j} \geq 40,000 \\
		&\sum\limits_{i=1}^7 \sum\limits_{j=1}^3 g_{i,j}x_{i,j} \geq 5 \\
		&\frac{1}{788} \sum\limits_{i=1}^7 \sum\limits_{j=1}^3 w_{i,j}x_{i,j} \geq 70 \\
		&x_{i,j} \geq 0 \text{ for } i=1,...,7  \text{ and } j=1,2,3
\end{align*}

\begin{problem}
Solve the above problem. Output the value of each $x_{i,j}$ and the maximum total net present value (return the primal objective multiplied by -1000).
\end{problem} 