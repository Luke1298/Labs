\lab{Getting Started}{Getting Started}
\objective{Introduce basic coding procedures and objects usage in Python.}
\label{lab:Essential_Python}

\section*{Python}
Python is a powerful general-purpose programming language. As an interpreted
language, it can be used interactively.
In the interactive mode, statements will be executed one at a time as you enter them.
It is extremely useful and allows for very rapid programming.
It is quickly gaining momentum as a fundamental tool in scientific computing because it has the following features:
\begin{itemize}
\item Clear, readable syntax
\item Full object orientation
\item Complete memory management (via garbage collection)
\item High-level, dynamic datatypes
\item Extensibility via C
\item Ability to interface with other languages such as R, C, C++, and Fortran
\item Embeddability in applications
\item Portability across many platforms (Linux, Windows, Mac OSX)
\item Open source
\end{itemize}
In addition to these, Python is freely available and can also be freely distributed.

\section*{Installation}
Python 2.7 is required for the labs in this text. We will download everything we need to get started using Anaconda. Anaconda is a free Python distribution that includes Python 2.7 and many of the commonly used libraries for scientific computing. Most of the libraries we will use in these labs are included in Anaconda. We will add more libraries in the future as needed. To download Anaconda, go to \url{http://continuum.io/downloads}, download the installer corresponding to your operating system, and follow the on-screen installation instructions. Make sure the installer you use corresponds to Python 2.7. Although later versions of Python are available, they do not yet have many of the features needed for scientific computing. For a detailed guide on initial installation, see Appendix \ref{win64install}. 


\section*{Running Python}
Many free IDEs (Integrated Development Environments) and text editors are compatible with Python. We recommend you use IPython, which provides three different interfaces: commandline, QTConsole, and Notebook. You can open these interfaces by running \li{ipython}, \li{ipython qtconsole}, or \li{ipython notebook} respectively.
The commandline interface is the simplest of the three, as it merely adds colored syntax to the text in the terminal window.
The QTConsole provides some extra features not available in the commandline interface.
The Notebook interface has the most features and is displayed in a web browser.

Some people prefer to use a text editor and IPython side-by-side. Some of the more popular text editors are described in Appendix A. It will be valuable for you to try a variety of programming environments and configurations until you find what suits you best.

For more information on installing Python and various libraries, see Appendices \ref{pythoninstall} and \ref{updateinstall}.

\section*{Learning Python}
The remainder of this lab will introduce you to some basic Python data types and control flow blocks. The text for this lab is intended to tell you just enough so that you can create these objects and experiment with them. Moreover, as you begin your study of Python, we \emph{strongly} suggest you read the following:
\begin{enumerate}
\item Chapters 3, 4, and 5 of the Official Python Tutorial \\
        (\url{http://docs.python.org/2.7/tutorial/introduction.html}).
\item Section 1.2 of the SciPy Lecture Notes\\
        (\url{http://scipy-lectures.github.io/}).
\item PEP8 - Python Style Guide \\
        (\url{http://www.python.org/dev/peps/pep-0008/}).
\end{enumerate}

In addition to these resources, there are other ways to learn Python. One useful aspect of the IPython interfaces is \emph{object introspection}, which allows you to see all the methods associated to an object. Also, you can use a question mark to learn about an object or method.

\begin{lstlisting}
# You can use a pound sign to write a single-line comment.

# To see the methods associated to an object, type the object name followed by a period, and press tab.
In [1]: list.
list.append   list.extend   list.insert   list.pop      list.reverse  
list.count    list.index    list.mro      list.remove   list.sort

# To learn about a method, use "?".
In [2]: list.append?
Docstring:  L.append(object) -- append object to end
Type:         method_descriptor
\end{lstlisting}

IPython also has a built-in function \li{help()} that provides interactive help.  Lastly, if you cannot answer your question using these strategies, try doing a Google search.  Many common questions about Python programming have been answered on internet forums.  StackOverflow.com, in particular, is often a valuable resource.

\begin{problem}
We are going to start by making a simple script that outputs \li{"Hello World"}.  Create a new file in the text editor of your choice that contains the following line:
\begin{lstlisting}
print "Hello World"
\end{lstlisting}
Save this file as \li{hello.py}. Now that this simple script has been created, we will execute it. First, open the command prompt and navigate to the directory where you saved \li{hello.py}. There are two different ways to execute the \li{hello.py} script. To execute the script using the Python distribution, run the following command in your command line:
\begin{lstlisting}
$ python hello.py
\end{lstlisting}
To execute the script using IPython, run the following in the command prompt:
\begin{lstlisting}
$ ipython

In [1]: run hello.py
\end{lstlisting}
\end{problem}

\section*{Data Types}

%=======================
%              Numerical Types
%=======================

\subsection*{Numerical Types}
There are four numerical data types, \li{int}, \li{long}, 
\li{float}, and \li{complex}, each of which stores a certain kind of number. 

\begin{lstlisting}
>>> type(3)
int

>>> type(3.0)
float
\end{lstlisting}

Python can be used as a calculator. Use \li{**} for exponentiation.
\begin{lstlisting}
>>> 3**2 + 2*5
19

>>> 13 % 3
1

# The underscore character is a variable with the value of the previous
# command's output.  You may think of it as the ANS button on 
# many calculators.
>>> 12 * 3
36
>>> _ / 4
9

# Run the following command to gain access to the sqrt function.
# More information on importing modules can be found in Volume 2, Lab 1.
>>> from math import sqrt
>>> sqrt(25)
5.0
>>> sqrt(2)
1.4142135623730951
\end{lstlisting}

Some extra care will be necessary when performing division.
\begin{lstlisting}
>>> 9 / 3
3
>>> 15 / 4	# Notice the expected result would be 3.75
3

# If either the dividend or divisor is a float, traditional division will be
# performed. This can be implemented in two different ways. 
>>> 15 / 4.0
3.75
>>> float(15) / 4	# This is an example of type casting.
3.75

\end{lstlisting}

We can also create and manipulate variables with Python.

\begin{lstlisting}
# Use a SINGLE equals sign to create a variable.
# Note that we do not need to specify a data type.
>>> x = 12
>>> y = 2 * 6

# If we wanted to change the default type that Python assigned to
# our variable, we use type casting
>>> y = float(y)	# changes y to a float

# Use a DOUBLE equals sign to check equality of variables.
>>> x == y  
True

# Using the is operator checks for the equality of objects.  Even though
# x and y have the same value, they are not the same object because
# x is an integer and y is a float
>>> x is y
False                   
\end{lstlisting}

We can also create and manipulate complex numbers.
\begin{lstlisting}
# There are two different ways to create complex numbers.
>>> x = complex(2,3)
>>> y = 2 + 3j	# Note that we use the letter j not i

# We can also access the real and imaginary parts separately if desired.
>>> x.real
2
>>> x.imag
3
\end{lstlisting}

Comparisons act as we would expect.
\begin{lstlisting}
>>> 3 > 2
True
>>> 6 < 6
False
>>> 6 <= 6
True
\end{lstlisting}

%======================
%            Function Definition
%======================

\subsection*{Function Definition}
To define a function, use the \li{def} keyword followed by the function name and a parenthesized list of formal parameters. Then indent the function body.  Whereas other languages use curly braces to delimit blocks, Python uses whitespace indentation.  This should be done using four spaces.  Many text editors will allow you to set the indentation type to spaces so you can use the Tab key on your keyboard to insert four spaces.

Because Python uses whitespace indentation, the interpreter will assume the code following a function definition to be indented and part of the code block.  If you wish to have an empty code block, use \li{pass}.
\begin{lstlisting}
>>> def add(x, y):
...return x + y
...
IndentationError: expected an indented block

# Here is an example of a function implemented correctly. The colon denotes the
# start of the code block.
>>> def add(x,y):
...    return x + y

>>> def add(x, y):
...    pass		# do nothing
...

# For multi-line functions, remember to indent all the lines that are supposed to
# be included in the function.
>>> def mult(x,y):
...    product = x*y
...    return product
...
\end{lstlisting}

We define a function with \emph{parameters} and call it with \emph{arguments}. In the example below, \li{width} and \li{height} are parameters for the function \li{area}. The values \li{2} and \li{5} are the arguments that we pass when calling the function. In practice, the terms \emph{parameter} and \emph{argument} are often used interchangeably.

\begin{lstlisting}
# Compute the area of a rectangle
>>> def area(width, height):
...     return width*height
...         
>>> area(2, 5)
10
\end{lstlisting}

It is also possible to specify default values for formal parameters, as in the following example.
\begin{lstlisting}
>>> def fn(a, b, c=0):
...    print a, b, c
\end{lstlisting}
The function \li{fn} has three formal parameters, and the value of \li{c} defaults to 0.
We can pass arguments to \li{fn} based on position (positional arguments) or name (named arguments or keyword arguments). We must define positional arguments before keyword arguments.
\begin{lstlisting}
# Call fn with 2 positional arguments (c=0 by default)
>>> fn(1, 2)
1 2 0

# Call fn with 3 positional arguments
>>> fn(4, 5, 6)
4 5 6

# Call fn with 1 positional argument and 2 named arguments
>>> fn(1, c=2, b=3)
1 3 2
\end{lstlisting}

%The final example demonstrates the flexibility of Python but is somewhat confusing. Whenever possible, you should pass arguments to a function in the order they are defined in the function. Thus, do the following.
%\begin{lstlisting}
%# Call fn with 1 positional argument and 2 named arguments
%>>> fn(1, b=3, c=2)
%1 3 2
%\end{lstlisting}

It is important to note a big difference between the two functions we have defined in this section.  The first function has a \li{return} statement where as the second function has a \li{print} statement.  The \li{print} statement does nothing more than displaying the value of a given object in the terminal.  If you have any intention of using the results of a function, always use a \li{return} statement.

The most general form of a function definition is as follows.
\begin{lstlisting}
def fn(*args, **kwargs):
\end{lstlisting}
This means that ``\li{fn} takes some arguments and keyword arguments.''
The arguments, \li{args}, are stored as a tuple; and the keyword arguments, \li{kwargs}, are stored in a dictionary.
The function \li{fn} can accept any number of arguments and keyword arguments.
\begin{lstlisting}
>>> def fn(*args, **kwargs):
...     print "Positional: ", args
...     print "Keyword: ", kwargs
...     
>>> fn("Hello", 2, 1, apples = 3, oranges = 2)
Positional: ('Hello', 2, 1)
Keyword: {'apples': 3, 'oranges': 2}
\end{lstlisting}

\begin{problem}
Write a function \li{sphere_volume} that accepts a parameter \li{r} and returns the volume of a sphere of radius \li{r}.
\end{problem}

%===========================
%           Strings 
%===========================

\subsection*{Strings}
A Python \li{string} can be created with either single or double quotes. They can be concatenated with the \li{+} operator.
\begin{lstlisting}
>>> str1 = "Hello"
>>> str2 = 'World'
>>> my_string = str1 + " " + str2 + "!"
>>> my_string
'Hello World!'

# Count the number of characters in a string
>>> len(my_string)
12

# You can type cast a number to a string using str()
>>> str(5)
'5'
\end{lstlisting}

We can access single characters of strings using brackets and a range of characters using \emph{slicing}. Slicing syntax is \li{[start:stop:step]}. The parameters \li{start} and \li{stop} default to the beginning and end of the string, respectively. The parameter \li{step} defaults to 0. For more information on slicing, see the section ``Lists.''

\begin{lstlisting}
# Indexing begins at 0 and negative numbers count backward from the end.
>>> my_string[4]
'o'
>>> my_string[-1]
'!'

# Splice the first 5 characters.  The syntax here means start at the beginning
# and go up to (but not including) the 5th character.
>>> my_string[:5]
'Hello'

# Splice from the 9th character to the end.
>>> my_string[9:]
'orld!'

# Splice between the 3rd and 8th characters.
>>> my_string[3:8]
'lo Wo'

# Pick out every other character in a string.
>>> my_string[::2]	# string[start:stop:step]
'HloWrd'
\end{lstlisting}

\begin{problem}
Write the following functions:
\begin{enumerate}
\item Write a function \li{first_half} that accepts a parameter \li{my_string} and returns the first half of \li{my_string}.
\item Write a function \li{reverse} that accepts a parameter \li{my_string} and reverses the characters in \li{my_string} using splicing.  Hint: The \li{step} parameter used in splicing can be negative.
\end{enumerate}
\end{problem}

%===========================
%          Lists
%===========================

\subsection*{Lists}
A Python \li{list} is created by enclosing comma-separated values with square brackets. Entries of a \li{list} do not have to be of the same type. You can access a single entry of a \li{list} or a range of entries with the same indexing or slicing operations we used on strings. 
\begin{lstlisting}
>>> my_list = ["Remi", 21, "08/06", 1993]
>>> my_list
['Remi', 21, '08/06', 1993]
>>> my_list[0]
'Remi'
>>> my_list[-2]
'08/06'
>>> my_list[:2]
['Remi', 21]
\end{lstlisting}

Some of most common methods of the \li{list} data type are \li{append}, \li{insert}, \li{remove}, and \li{pop}.
\begin{lstlisting}
>>> my_list = [1, 2]
>>> my_list.append(4)
>>> my_list.insert(2, 3)	# insert(index, value)
>>> my_list
[1, 2, 3, 4]
>>> my_list.remove(3)	# remove(value)
>>> my_list.pop()		# remove and return last value in my_list
4
>>> my_list
[1, 2]
\end{lstlisting}

Splicing is also very useful for replacing values in a list.
\begin{lstlisting}
>>> my_list = [1, 2, 3, 4, 5]
>>> my_list[3:] = [9, 10]
[1, 2, 3, 9, 10]
\end{lstlisting}

Python has some built-in functions that are particularly helpful with lists.
\begin{lstlisting}
>>> my_list = [1, 2, 3, 4, 5]
>>> sum(my_list)
15
>>> max(my_list)
5
>>> min(my_list)
1
\end{lstlisting}

We can quickly check to see if a certain value is in a list using the \li{in} operator.
\begin{lstlisting}
>>> my_list = [1, 2, 3, 4, 5]
>>> 2 in my_list
True
>>> 6 in my_list
False
\end{lstlisting}

\begin{problem}
Start with the list \li{my_list = ["ant","baboon","cat","dog"]} and perform the following operations:
\begin{enumerate}
\item Append the word, \li{"elephant"}.
\item Remove \li{"ant"}.
\item Remove index 1.
\item Replace the word at index 2 with the word, \li{"donkey"}.
\item Append the word, \li{"fox"}.
\end{enumerate}
\end{problem}

Whenever possible, you should create your lists using a \emph{list comprehension}, which is demonstrated below.  We will talk about list comprehension in more detail after we talk about \li{for} loops.
\begin{lstlisting}
>>> [float(n) for n in xrange(5)]
[0.0, 1.0, 2.0, 3.0, 4.0]
\end{lstlisting}

\begin{problem}
Write a function that accepts a list of numbers and converts each entry into a string using list comprehension.  For example, the list \li{[1, 2, 3]} would be changed to \li{["1", "2", "3"]}.
\end{problem}

%Required readings: 
%\begin{itemize}
%\item \url{http://scipy-lectures.github.io/intro/language/basic_types.html#lists} 
%\item \url{https://docs.python.org/2.7/tutorial/introduction.html#lists}
%\item \url{https://docs.python.org/2.7/tutorial/datastructures.html#list-comprehensions}
%\end{itemize}

% =================
%       Sets
% =================

\subsection*{Sets}
A Python \li{set} is an unordered collection of distinct objects and can be created from a list. We can add or remove members from a set after its creation. There are two different ways to initiate a set.
\begin{lstlisting}
>>> library_members = {"Lytle, Josh", "Henriksen, Amelia", "Grout, Ryan"}
>>> gym_members = set(["Doe, John", "Doe, John", "Smith, Jane", "Brown, Bob"])
>>> gym_members
{'Brown, Bob', 'Doe, John', 'Smith, Jane'}
>>> gym_members.discard("Doe, John");
>>> gym_members
{'Brown, Bob', 'Smith, Jane'}
>>> gym_members.add("Lytle, Josh")
>>> gym_members
{'Brown, Bob', 'Lytle, Josh', 'Smith, Jane'}
\end{lstlisting}

Like mathematical sets, a \li{set} has operations like union, intersection, 
difference, and symmetric difference.
\begin{lstlisting}
# Set intersection returns a new set object.
>>> set.intersection(gym_members, library_members)
{'Lytle, Josh'}
\end{lstlisting}

% ===================
%     Dictionaries 
% ===================


\subsection*{Dictionaries}
Like a \li{set}, a Python \li{dictionary} is an unordered data type.  A \li{dictionary} stores \li{key:value} pairs, which are called \emph{items}. The values of a dictionary are indexed by its keys.  Here are various examples of how to use dictionaries.
\begin{lstlisting}
>>> my_dictionary = {"business": 4121, "math": 2061, "visual arts" : 7321} 
>>> my_dictionary["math"]
2061
>>> my_dictionary["science"] = 6284
>>> del my_dictionary["business"]
{"math":2061, "visual arts":7321, "science":6284}
>>> my_dictionary.keys()
["math", "visual arts", "science"]
>>> my_dictionary.values()
[2061, 7321, 6284]
>>> my_dictionary.has_key("math")
True
\end{lstlisting}

The keys of a dictionary must be \emph{immutable}, which means that they must be objects that cannot be modified after creation. Numerical types and strings are immutable objects. Lists, dictionaries, and sets are mutable.

%==================
%    Control Flow Tools
%==================

\section*{Control Flow Tools}
Control flow blocks control the order in which your code is executed.
Python supports the usual control flow statements 
including \li{if} statements, \li{while} loops and \li{for} loops. 

\subsection*{The If Statement}
An \li{if} statement executes the 
indented code \emph{if} the given condition holds.
The \li{elif} statement is short for ``else if'' and can be used multiple times following an \li{if} statement, or not at all. 
The \li{else} keyword may be used at most once at the end of a series of \li{if}/\li{elif} statements.
\begin{lstlisting}
>>> food = "bagel"         
>>> if food == "apple":    # the colon denotes the start of the code block.
...     print "72 calories"
... elif food == "banana":
...     print "105 calories"
... elif food == "egg":
...     print "102 calories"
... else: 
...     print "calorie count unavailable"
...     
calorie count unavailable
\end{lstlisting}

\begin{problem}
Implement a function that translates a word into Pig Latin.  As a reminder, Pig Latin takes the first character in a word, moves it to the end, and adds \li{"ay"}.  For example, the word \li{"banana"} would be translated to \li{"ananabay"}.  There is a special case if the word starts with a vowel.  In this case, just add \li{"hay"} to the end of the word.  For example, \li{"apple"} would be translated to \li{"applehay"}.  Hint: use the \li{in} operator to check if the first letter of the word is a vowel.
\end{problem}

\subsection*{The While Loop}
A \li{while} loop executes an indented block of code \emph{while} the given condition holds. 

\begin{lstlisting}
>>> i = 0
>>> while i < 10:        
...     print i,            
...     i = i+1       
...     
0 1 2 3 4 5 6 7 8 9
\end{lstlisting}
In the above example, the comma in the line \li{print i,} makes Python print all the numbers on the same line (by stripping off newline characters). Try running this example without the comma and see what happens.

An alternative implementation of the previous \li{while} loop is the following:
\begin{lstlisting}
>>> i = 0
>>> while True:
...    print i,
...    i = i+1
...    if i >= 10:
...        break		# exits the loop
...
0 1 2 3 4 5 6 7 8 9
\end{lstlisting}

\subsection*{The For Loop}
A \li{for} loop iterates over the items in any \emph{iterable}. Iterables include lists, sets, and dictionaries.  There are a few different ways to define a \li{for} loop.  The most common ways are demonstrated here. The method we choose solely depends on what we are trying to accomplish.

\begin{lstlisting}
# the xrange function is defined as xrange(start,stop,step)
>>> for i in xrange(10):
...    print i,
...     
0 1 2 3 4 5 6 7 8 9

>>> for i in xrange(50,100,10):
...    print i,
...
50 60 70 80 90

# When needing to access the elements of a list and the index
# of the entry is not important, use this method.
>>> my_list = ["red", "green", "blue", "yellow"]
>>> for entry in my_list:
...    print entry,
...
red green blue yellow

# When keeping track of indices is important, use this method.
>>> for i in xrange(len(my_list)):
...    print my_list[i] + str(i),
...
red0 green1 blue2 yellow3
\end{lstlisting}

\begin{problem}
Write a function that translates a list of integers between 1-26 to the corresponding letter of the alphabet. For example, if the input was \li{[1, 2, 3]}, the output would be \li{["a", "b", "c"]}. Implement this function using a dictionary. 
\end{problem}

% ====================
%      Generators
% ====================

\subsection*{Generators}

Let's say we are given a large data set and we need to loop through data as we analyze it. If we were to use the techniques we have discussed thus far, we would need to add each piece of data to a list before being able to loop through it. Or let's say we needed to pass this large data set to a function. Passing the entire list would be very inefficient. We get around this problem using \emph{generators}. 

When we want to produce a generator from a function, we use \li{yield} instead of \li{return}. Because \li{return} only gets called once, all the data that needs to be returned by a function must be processed before the return statement. When using \li{yield}, each piece of data can be processed and returned one and a time.

See the following code box for an example of implementation:

\begin{lstlisting}
>>> def range_generator(n):
...    i = 0
...    while i < n:
...        yield i 
...        i += 1
...

>>> three = range_generator(3)
>>> three.next()
0
>>> three.next()
1
>>> three.next()
2

# Generators can be used as iterators in for loops
>>> six = range_generator(6)
>>> for i in three:
...     print i,
...
0 1 2 3 4 5

# We can even make an infinite generator. Clearly, this would not be possible using a list.
>>> def inf_generator():
...    i = 0
...    while True:
...        yield i
...        i += 1

\end{lstlisting}

\begin{problem}
Write a function \li{squares} that accepts a parameter \li{n} and creates a generator of all squares less than \li{n}. For example, \li{squares(10)} would generate \li{0, 1, 2, 4,} and \li{9}.
\end{problem}

% ====================
%   List Comprehension
% ====================

\subsection*{Revisiting List Comprehension}
List Comprehension is a powerful, efficient way to build lists.  The code is more concise and also runs faster.  The following \li{for} loop and list comprehension produce the same list but he list comprehension takes about two-thirds the time to execute.

\begin{lstlisting}
>>> loop_output = []
>>> for i in xrange(5):
...    loop_output.append(i**2)
...
>>> list_output = [i**2 for i in xrange(5)]
\end{lstlisting}

\begin{problem}
Recall that the alternating harmonic series (1 - 1/2 + 1/3 - 1/4 + 1/5 - ...) converges to ln(2). Using the first \li{n} terms of this series, approximate ln(2).  Implement your function using list comprehension.  The sum of the first 500,000 terms of this series approximates ln(2) to five decimal places.
\end{problem}

\begin{comment}
\section*{Specifications}
We suggest that you submit your solutions in a file called \li{solutions.py}, using the following format.
\lstinputlisting[language=, abovecaptionskip=2pt, belowcaptionskip=2pt]{template.txt}
\end{comment}
