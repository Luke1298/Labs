\lab{Python}{Getting Started}{Getting Started}
\objective{Introduce basic coding procedures and objects usage in Python.}
\label{lab:Essential_Python}

\section*{Python}
Python is a powerful general-purpose programming language. As an interpreted
language, it can be used interactively. 
It is quickly gaining momentum as a fundamental tool in scientific computing because it has the following features:
\begin{itemize}
\item Clear, readable syntax
\item Full object orientation
\item Complete memory management (via garbage collection)
\item High level, dynamic datatypes
\item Extensibility via C
\item Ability to interface to(with?) other languages such as R, C, C++, and Fortran
\item Embeddability in applications
\item Portability across many platforms (Linux, Windows, Mac OSX)
\item Open source
\end{itemize}
In addition to these, Python is freely available and can also be freely distributed.

\section*{Running Python}
Python 2.7 is required for the labs in this text and can be downloaded from \url{http://www.python.org/}. Although later versions of 
Python are available, they do not have many of the features needed for scientific computing. 

Many free IDEs (Integrated Development Environments) and text editors are compatible with Python. We recommend you use IPython, which provides three different interfaces: commandline, QTConsole, and Notebook. You can open these interfaces by running \li{ipython}, \li{ipython qtconsole}, or \li{ipython notebook} respectively.
The commandline interface is the simplest of the three, as it merely adds colored syntax to the text in the terminal window.
The QTConsole provides some extra features not available in the commandline interface.
The Notebook interface has the most features and is displayed in a web browser.

For more information on downloading and installing various packages, please 
reference the Appendix. TODO


\section*{Learning Python}
The following examples and problems highlight important functionality and syntax in Python, but they are by no means comprehensive.  As you familiarize yourself with Python and begin to work through the problems, we \emph{strongly} suggest you read the following:
\begin{enumerate}
\item Chapters 3, 4, and 5 of the Official Python Tutorial \\
        (\url{http://docs.python.org/2.7/tutorial/introduction.html}).
\item Section 1.2 of the SciPy Lecture Notes \\
        (\url{http://scipy-lectures.github.io/}).
\item PEP8 - Python Style Guide \\
        (\url{http://www.python.org/dev/peps/pep-0008/}).
\end{enumerate}



The next sections introduce some standard data types of Python, including numbers, strings, lists, sets, and dictionaries. 

\subsection*{Numbers}
\begin{example} There are four numerical data types, \li{int}, \li{long}, 
\li{float}, and \li{complex}, each of which stores a certain kind of number. These data types are all immutable, meaning that when you change the value of a number, you allocate a new variable.

\begin{lstlisting}
# You can use a pound sign to write a single-line comment.
>>> '''
... We can also use multiline strings as comments. 
... These are enclosed by 3 single (or double) quotations at 
... the beginning and end. 
... '''

# Python can be used as a simple calculator.
>>> 9*8 + 4*5
92

# Integer division returns the floor.
>>> 100/3
33

>>> # Use a floating point operand to return a floating point result.
>>> 100/3.0
33.333333333333336

# Use a SINGLE equals sign to create a variable.
>>> x = 12
>>> y = 2 * 6

# Use a DOUBLE equals sign to check equality of variables.
>>> x == y
True

# Here, we define the variables a, b, and c at the same time.
>>> a, b, c = 1, 2, 3
>>> (a + b)**c       # Exponentiation is denoted '**'
27
\end{lstlisting}
\end{example}

\begin{problem}
It is useful to know how to find help in the official Python documentation (\li{https://docs.python.org/2.7/}). Use the official Python documentation to find the answer to each of the following:
\begin{enumerate}

\item What are the two ways to create a complex number? 
How do you extract the real part or the imaginary part?
\item How do you convert an integer to a float? (This is called \emph{casting}.)
\item Is there a way to require integer division when using floats?
\end{enumerate}
\end{problem}

\begin{problem}
Why does \li{7/3} return 2 in Python 2.7? 
\end{problem}

\subsection*{Strings}
\begin{example} A Python \li{string} can be created with either single or double quotes. They can be concatenated with the \li{+} operator.
\begin{lstlisting}
>>> str1 = "I love"
>>> str2 = 'the ACME program'
# Use a semicolon to write multiple commands on the same line.
>>> my_string = str1 + " " + str2 + "!"; mystring
'I love the ACME program!'
\end{lstlisting}

Like numerical data types, a \li{string} is immutable. We can cast other data types to strings using the \li{str()} function.

We can access single characters of strings using brackets.

\begin{lstlisting}
# Indexing begins at 0.
>>> my_string[0]
'I'

# Negative numbers count backwards from the end.
>>> my_string[-1]
'!'
\end{lstlisting}

We can access a range of characters using \emph{slicing}.
\begin{lstlisting}
>>> my_string[2:6]
'love'
>>> my_string[:-1]
'I love the ACME program'
>>> my_string[2:]
'love the ACME program!'
\end{lstlisting}
Note that Python always includes the first index in a range and excludes the last index. Also, a blank first index defaults to 0, and a blank second index defaults to the end of the string. We can also specify the \emph{step} with slicing.
\begin{lstlisting}
# Pick out every other character in a string.
>>> my_string[::2]	# string[start:stop:step]
'Ilv h CEporm'
\end{lstlisting}
\end{example}

\begin{problem}
Answer the following questions. Use the official Python documentation, if necessary.
\begin{enumerate}
\item What does it mean for a string to be an immutable object? 
\item What happens in \li{my_string[::3]} and \li{my_string[::-1]} provided 
that \li{my_string = "I love the new ACME program\!"}? 
\item How can you print a \li{string} backwards?
\end{enumerate}
\end{problem}

\subsection*{Lists}
\begin{example}
You can create a Python \li{list} by enclosing comma-separated values with square brackets. 
\begin{lstlisting}
>>> my_list = ["Remi", 21, "08/06", 1993]; my_list
['Remi', 21, '08/06', 1993]
\end{lstlisting}
You can access a single entry of a \li{list} or a range of entries with the same indexing or slicing operations as we used on strings. 
\begin{lstlisting}
>>> my_list[0]
'Remi'
\end{lstlisting}
However, unlike a number or a \li{string}, a \li{list} is mutable. 
\begin{lstlisting}
>>> my_list[2] =  "07/10"; my_list
['Remi', 21, '07/10', 1993]
\end{lstlisting}
Some of the methods of the \li{list} data typed are demonstrated below, including \li{append}, \li{insert}, \li{remove}, \li{sort}, and \li{reverse}.

\begin{lstlisting}
# The append method adds an item to the end of your list.
>>> my_list.append("student"); my_list
['Remi', 21, '07/10', 1993, 'student']

# The insert method inserts an item at the specified position. 
>>> my_list.insert(0, "female"); my_list
['female', 'Remi', 21, '07/10', 1993, 'student']

# The remove method removes an item from your list.
>>> my_list.remove("student"); my_list
['female', 'Remi', 21, '07/10', 1993]

# The sort method sorts a list in place. This changes the original list.
>>> my_list.sort(); my_list
[21, 1993, '07/10', 'Remi', 'female']

# The reverse method reverses your list.
>>> my_list.reverse(); my_list
['female', 'Remi', '07/10', 1993, 21]
\end{lstlisting}

Finally, list comprehensions provide a powerful way to create lists. 
\begin{lstlisting}
>>> [x**2 for x in range(10)]
[0, 1, 4, 9, 16, 25, 36, 49, 64, 81]
\end{lstlisting}

Whenever possible, you should create your lists using list comprehension.
\label{exp:lists}
\end{example}

\begin{problem}
Answer the following questions. Use the official Python documentation, if necessary.

\begin{enumerate}
\item What is the difference between mutable and immutable objects?
\item If \li{my_list = ["mushrooms", "rock climbing", 1947, 1954, "yoga"]}
\begin{enumerate}
\item How would you access ``yoga''? 
\item How would you view a copy of \li{my_list}?
\item How would you clear the entire list? 
\item How would you find the length? 
\item How would you assign ``mushrooms'' (first entry) and ``rock climbing'' (second entry) 
of \li{my_list} to ``Peter Pan'' and ``camelbak''? 
(This should be done in one line of code and is known as a slice assignment)
\item How would you add ``Jonathan, my pet fish'' to the end of the list?
\end{enumerate}
\item What sequence of commands would you use to implement the following?
\begin{enumerate}
\item Create an empty list. 
\item Add 5 integers to your list. 
\item Cast the integer at index 3 as a float.
\item Remove the integer at index 2. 
\item Sort your list backwards.
\end{enumerate}
\end{enumerate}
\end{problem}


\subsection*{Sets}
\begin{example}
A Python \li{set} is an unordered collection of distinct objects, and can be created from a list.
\begin{lstlisting}
>>> gym_members = set(["Doe, John", "Doe, John", "Smith, Jane", "Brown, Bob"])
>>> gym_members
set(['Brown, Bob', 'Doe, John', 'Smith, Jane'])
\end{lstlisting}
Like a \li{list}, a \li{set} is mutable. 
\begin{lstlisting}
>>> gym_members.discard("Doe, John"); gym_members
set(['Brown, Bob', 'Smith, Jane'])
>>> gym_members.add("Lytle, Josh"); gym_members
set(['Brown, Bob', 'Lytle, Josh', 'Smith, Jane'])
\end{lstlisting}

Like mathematical sets, a \li{set} has operations like union, intersection, 
difference, and symmetric difference.

\begin{lstlisting}
# Set intersection returns a new set object.
>>> library_members = set(["Lytle, Josh", "Henriksen, Ian", "Grout, Ryan"])
>>> set.intersection(gym_members, library_members)
set(['Lytle, Josh'])
\end{lstlisting}

We can also do set comprehensions like we did list comprehensions.
\begin{lstlisting}
>>> set(x**2 for x in range(10))
set([0, 1, 4, 9, 16, 25, 36, 49, 64, 81])
\end{lstlisting}
\end{example}

\begin{problem}
Answer the following questions. Use the official Python documentation, if necessary.
\begin{enumerate}
\item What are the two ways to create sets? How do you create an empty set?
\item How would you take the union of sets?
\item How are sets indexed?
\end{enumerate}
\end{problem}

\subsection*{Dictionaries}
\begin{example}
Like a \li{set}, a Python \li{dictionary} is an unordered data type.  A \li{dictionary} stores \li{key:value} pairs. The keys of a dictionary must be immutable objects (such as strings or numbers). The values of a dictionary are indexed by the keys.
\begin{lstlisting}
>>> tel = {"marriott": 4121, "math": 2061, "visual arts" : 7321} 
>>> tel["math"]
2061
\end{lstlisting}
A \li{dictionary} is mutable.
\begin{lstlisting}
>>> tel["math"] = 1234; tel
{'visual arts': 7321, 'marriott': 4121, 'math': 1234}
\end{lstlisting}
\end{example}

\begin{problem}
Answer the following questions. Use the official Python documentation, if necessary.
\begin{enumerate}
\item What are the two ways to create dictionaries? How do you create an empty dictionary?
\item In Example \ref{exp:lists} above we created a list using list comprehension. Create a dictionary using a dictionary comprehension to produce the following output:

\begin{lstlisting}
{2: 4, 4: 16, 6: 36, 8: 64, 10: 100}
\end{lstlisting}

(Note that because dictionaries are unordered, your output may not exactly the output displayed above)
\item How do you delete a key-value pair?
\item How do you access a list of all the values in your dictionary?  
How do you access a list of all the items?
\end{enumerate}
\end{problem}


\section*{Control Flow Tools}
Control flow blocks control the order in which your code is executed.
Python supports the usual control flow statements 
including while loops, if statements, for loops, and function definitions. 

\subsection*{The while Loop}
\begin{example}
Python uses indentation to identify the beginning and end of blocks of code, so you must indent each line of an execution block the same way.
The convention is to indent blocks of code with four spaces. A \li{while} loop executes an indented block of code \emph{while} the given condition holds. 
Be aware that you can create an infinite loop if the condition is always true.

\begin{lstlisting}
# The Fibonacci sequence can be formulated using a while statement. 
>>> a, b = 0, 1
>>> while b < 10:           # The colon is required.
...     print a,             # This indented line is executed if b<10.
...     a, b = b, a+b       # This indented line is also executed if b<10.
...     
0 1 1 2 3 5
\end{lstlisting}
\end{example}

\subsection*{The if Statement}
\begin{example}
An \li{if} statement executes the 
indented code \emph{if} the given condition holds.
The \li{elif} statement is short for ``else if'' and can be used multiple times following an \li{if} statement, or not at all. 
The \li{else} keyword may be used at most once at the end of a series of \li{if}/\li{elif} statements.
\begin{lstlisting}
>>> food = "bagel"
>>> if food == "apple":
...     print "72 calories"
... elif food == "banana":
...     print "105 calories"
... elif food == "egg":
...     print "102 calories"
... else: 
...     print "calorie count unavailable"
...     
calorie count unavailable
\end{lstlisting}
\end{example}

\subsection*{The for Loop}
\begin{example}
A \li{for} loop iterates over the items in any list, set, dictionary, tuple or other iterable. 
The \li{range} function creates a list of numbers, stopping just before the inputed value. 

\begin{lstlisting}
>>> for i in range(5):
...     print i,
...     
0 1 2 3 4
\end{lstlisting}
\end{example}


\subsection*{Function Definition}
\begin{example}
To define a function, use the \li{def} keyword followed by the function name and a parenthesized list of formal parameters. Then indent the function body.
\begin{lstlisting}
>>> def fibonacci(n):
...     a, b = 0, 1
...     while a < n:
...         print a,
...         a, b = b, a+b
...         
>>> fibonacci(10)
0 1 1 2 3 5 8
\end{lstlisting}

It is also possible to specify default values for formal parameters, as in the following example.
\begin{lstlisting}
>>> def fn(a, b, c=0):
...    print a, b, c
\end{lstlisting}
The function \li{fn} has three formal parameters, and the value of \li{c} defaults to 0.
We can pass arguments to \li{fn} based on position (positional arguments) or name (named arguments or keyword arguments). We must define positional arguments before keyword arguments.
\begin{lstlisting}
# Call fn with 2 positional arguments (c=0 by default)
>>> fn(1, 2)
1 2 0

# Call fn with 3 positional arguments
>>> fn(4, 5, 6)
4 5 6

# Call fn with 1 positional argument and 2 named arguments
>>> fn(1, c=2, b=3)
1 3 2
\end{lstlisting}
The final example demonstrates the flexibility of Python but is somewhat confusing. Whenever possible, you should pass arguments to a function in the order they are defined in the function. Thus, do the following.
\begin{lstlisting}
# Call fn with 1 positional argument and 2 named arguments
>>> fn(1, b=3, c=2)
1 3 2
\end{lstlisting}

\begin{info}
When defining a function, we define the \emph{parameters} of the function.  We then call the function with \emph{arguments}.
In the example below, \li{a} and \li{b} are parameters for the function \li{add_numbers}.
The values \li{5.2} and \li{10} are the arguments that we pass when calling the function.

\begin{lstlisting}
>>> def add_numbers(a, b):
...    return a + b
...
>>> add_numbers(5.2, 10)
15.2
\end{lstlisting}
In practice, the terms \emph{parameter} and \emph{argument} are often used interchangeably.
\end{info}

\begin{problem}
Define the following function in your interpreter.
\begin{lstlisting}
>>> def track(n, my_list=[]):
...    my_list.append(n)
...    return my_list
\end{lstlisting}
What does the function do?
Try calling the function multiples times with different values of n and printing the list returned each time.
This function demonstrates something very important: the default values of a function are only evaluated once.
This means that if you call \li{fn} with the default value for \li{my_list} multiple times, the same list is used each time.
What is the best way to modify \li{fn} so that the default value for \li{my_list} is not shared between calls?
\end{problem}

The most general form of a function definition is as follows.
\begin{lstlisting}
>>> def f(*args, **kwargs):
\end{lstlisting}
This means that ``\li{f} takes some arguments and keyword arguments.''
The arguments, \li{args}, are stored as a tuple; and the keyword arguments, \li{kwargs}, are stored in a dictionary.
The function \li{f} can accept any number of arguments or keyword arguments.
\begin{lstlisting}
>>> def f(*args, **kwargs):
...     print "Positional: ", args
...     print "Keyword: ", kwargs
...     
>>> f("Hello", 2, 1, apples = 3, oranges = 2)
Positional: ('Hello', 2, 1)
Keyword: {'apples': 3, 'oranges': 2}

# The stars are operators and have special meaning in this case
# These operators unpack tuples and dictionaries into arguments or keyword arguments
>>> f(*range(5)) # Equivalent to f(0, 1, 2, 3, 4)
Positional: (0, 1, 2, 3, 4)
Keyword: {}
>>> d = {'a': 1, 'b': 2}
>>> f(**d) # Equivalent to f(a=1, b=2)
Positional: ()
Keyword: {'a': 1, 'b': 2}
\end{lstlisting}
\end{example}

\begin{problem}
Answer the following questions. Use the official Python documentation, if necessary.

\begin{enumerate}
\item Explain what the \li{print} and \li{return} statements do. How are they different?
\item What is wrong with the following code?
\begin{lstlisting}
Grocery List = ['pineapple', 'orange juice', "avocados", "pesto sauce"]
for i in range(Grocery List)
if i % 2 = 0
print i, Grocery List(i)
\end{lstlisting}
provided you want the following output:
\begin{lstlisting}
0 pineapple
2 avocados
\end{lstlisting}
Note that this code may contain one error, many errors, or no errors at all.
\item When you call the function ``groceries'' below, it returns an error. Why?
\begin{lstlisting}
def groceries(food, drink):
    print food
    print drink    

groceries(food="Bananas", "Juice")
\end{lstlisting}
\end{enumerate}
\end{problem}



\section*{Specifications}
We suggest that you submit your solutions in a file called \li{solutions.py}, using the following format.
\lstinputlisting[language=, abovecaptionskip=2pt, belowcaptionskip=2pt]{template.txt}
