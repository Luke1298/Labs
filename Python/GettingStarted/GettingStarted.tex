\lab{Python}{Getting Started}{Getting Started}
\objective{To introduce basic coding procedures and objects usage in Python.}
\label{lab:Essential_Python}

Python is a powerful general-purpose programming language. It is an interpreted
language and can even be used interactively. 
It is quickly gaining momentum as a tool in scientific computing and has several very
nice key features:
\begin{itemize}
\item Clear, readable syntax
\item Full object orientation
\item Complete memory management (via garbage collection)
\item High level, dynamic datatypes
\item Extensibility via C
\item Ability to interface to other languages such as R, C, C++, and Fortran
\item Embeddability in applications
\item Portability across many platforms (Linux, Windows, Mac OSX)
\item Open source

\end{itemize}
In addition to these, Python is freely available and can also be freely distributed.
For more information on installation and other libraries, check out the Appendix.


\section*{Introducing Python}
There are many well written introductions to the Python language.
We recommend you review the following sources if you are learning Python for the
first time.

\begin{itemize}
\item Official Python Tutorial (Start with Chapter 3): \url{http://docs.python.org/tutorial/index.html} 
\item SciPy Lecture Notes: \url{http://scipy-lectures.github.com/}
\item Dive Into Python 3: \url{http://getpython3.com/diveintopython3/}
\item Python Style Guide: \url{http://www.python.org/dev/peps/pep-0008/}
\end{itemize}

The following examples and problems highlight important functionality
and syntax in Python, but are by no means comprehensive.
As you familiarize yourself with Python and begin to work through the problems,  we strongly suggest you read the following:
\begin{enumerate}
\item Chapters 3, 4, and 5 of the Official Python Tutorial \\
        (\url{http://docs.python.org/2/tutorial/introduction.html}).
\item Section 1.2 of the SciPy Lecture Notes \\
        (\url{http://scipy-lectures.github.io/}).
\item PEP8 - Python Style Guide \\
        (\url{http://www.python.org/dev/peps/pep-0008/}).
\end{enumerate}


As you review the following examples, we recommend that you open your command line or Ipython Notebook to verify and work through them.

\subsection*{Numbers and Strings}

\begin{example}
Python can be used as a calculator. It also 
includes the \li{string} module.

\begin{lstlisting}
>>> #this is a comment! Note that it only extends to the end of the line

>>> #this is a multiline comment!
''' 
Note that comments that require multiple lines are multiline strings.
They can be enclosed by 3 single or 3 double quotations at the beginning and end of your comment. 
'''

>>> 9 * 8
72

>>> 100/3 #integer division returns the floor.
33

>>> 100/3.0 #this converts the integer operand to a floating point operand. When at least one of the values is a float, the result is also a float. 
33.333333333333336

>>> x = 12

>>> y = 3 * 4

>>> x * y
144

>>> a, b, c = 1, 2, 3 #this is called multiple assignment

>>> my_string = "I love the new ACME program!" #this is a string

>>> my_string[15:19] #strings can be sliced to access a specific portion of the string.
'ACME'

>>> my_string[:6] #or to get the first 6 characters. This is equivalent to my_string[0:6]
'I love'

>>> my_string[::2] #or every other character! #SYNTAX: my_string[start:stop:step]
'Ilv h e CEporm'

\end{lstlisting}
\end{example}

\begin{problem}
Answer the following questions with the best answer:

\begin{enumerate}
\item Why does \li{7/3} return \li{2} in Python 2? 
\item What are the two ways to create a complex number? 
How do you extract just the real part and just the imaginary part?
\item How would you cast an integer as a float?
\item Is there a way to explicitly express integer division when using floats?
\item What does it mean for \li{string} to be an immutable object? 
\item What happens in \li{my_string[::2]} and \li{my_string[27:0:-1]} provided 
that \li{my_string = "I love the new ACME program\!"}? 
How can you access the entire string in reverse?

\end{enumerate}
\end{problem}


\subsection*{Containers}

\begin{example}
Python contains a variety of container datatypes like \li{list}, \li{set}, 
\li{dict}, and \li{tuple}.

\begin{lstlisting}
>>> my_list = ["Remi", 21, "02/05", 1993] #lists are written as comma-separated items 

>>> my_list[0] #lists are indexed starting at zero.
'Remi'

>>> my_list[2] = "February 05"

>>> len(my_list) #returns the number of items in the list.
4

>>> basket = set(["banana", "sandwich", "apple"]) #a set is an unordered collection with no duplicate elements. 

>>> "banana" in basket
True

>>> #List comprehensions provide a concise way to create lists, which are mutable sequences. 

>>> squares = [x**2 for x in range(10)] #this creates a list of squares. SYNTAX: range(start, stop, step) produces [start, start + step, start + 2* step, ...]

>>> tel = {"accounting":4234, "admissions": 2507, "financial aid": 4104, "marriott": 4121, "math": 2061, "visual arts" : 7321} #dictionaries are unordered sets of key:value pairs. Note that keys are unique.

>>> tel["math"] #just as sequences are indexed by numbers, dictionaries are indexed by keys. This key, "math", corresponds to the value 2061.
 2061

>>> my_tuple = 123, 456, 789, "000" #tuples are sequences, like lists, but unlike lists are immutable (like strings). 

>>> my_tuple
(123, 456, 789, '000')



\end{lstlisting}
\end{example}

\begin{problem}
Answer the following questions with the best answer:
\begin{enumerate}
\item What is the difference between mutable and immutable objects?
\item If \li{a = ["mushrooms", "rock climbing", 1947, 1954, "yoga"]}
\begin{enumerate}
	\item How would you access "yoga"? 
	\item How would you view a copy of the entire list?
	\item How would you clear the entire list? 
	\item How would you find the length? 
	\item How would you assign "mushrooms" (first entry) and "rock climbing" (second entry) 
	of \li{a} to "Peter Pan" and "camelbak"? 
	(This should be done in one line of code and is recognized as a slice assignment)
	\item How would you add "Jonathan, my pet fish" to the end of the list?
\end{enumerate}
\item What are the two ways to create sets? Which way must be used to 
create an empty set?
\item What are the two ways to create dictionaries? Which way must be used to
create an empty dictionary?
\item Just as the example above created a list using list comprehension, you can create dictionaries using dictionary comprehensions.
Create a dictionary using dictionary comprehension to produce the following output:
\begin{lstlisting}
{2: 4, 4: 16, 6: 36, 8: 64, 10: 100}
\end{lstlisting}
(Note: because dictionaries are unordered sets, your output need not match the order of the output displayed above).
\item How do you delete a key:value pair?
\item How do you access a list of all the keys in your dictionary? And all the values?


\end{enumerate}
\end{problem}


\subsection*{Control Flow Tools}

\begin{example}
Control flow tools control the order in which your code is executed.
Python supports the usual control flow statements used in other languages
including the \li{while} statement, the \li{if} statement, and function definition. 
Python also supports the \li{for} statement, but it differs a bit from other programming languages by iterating over the items of a sequence (rather than only over a progression of numbers).

\begin{lstlisting}
>>> #the Fibonacci series can easily be formulated with a while statement.

>>> a, b = 0, 1

>>> while b < 10: #while this condition holds, do the following
   ....:     print a
   ....:     a, b = b, a+b #update your variables
   ....:     
0
1
1
2
3
5

>>> food = "bagel"

>>> if food == "apple":
   ....:     print "72 calories"
   ....: elif food == "banana":
   ....:     print "105 calories"
   ....: elif food == "egg":
   ....:     print "102 calories"
   ....: elif food == "oatmeal":
   ....:     print "147 calories"
   ....: elif food == "pizza":
   ....:     print "298 calories"
   ....: else: 
   ....:     print "calorie count unavailable"
   ....:     
calorie count unavailable

>>> ''' note that the else statement is optional, but if used
   ....: needs no condition. '''
   
>>> my_list = ["Henry XXI", "beta fish", "asparagus", 4]

>>> for i in range(len(my_list)): #the range() function allows iteration over a sequence of numbers. 
   ....:     my_list.append(i)
   ....:     

>>> my_list
['Henry XXI', 'beta fish', 'asparagus', 4, 0, 1, 2, 3]

>>> def fibonacci(n): #the keyword def introduces a function definition
   ....:     a, b = 0, 1
   ....:     while a < n:
   ....:         print a, b
   ....:         a, b = b, a+b
   ....:         

>>> #now the fibonacci function can be called. Try fibonacci(1000)!

>>> fibonacci(10) #available to make sure you have the same output.
0 1
1 1
1 2
2 3
3 5
5 8
8 13

\end{lstlisting}
\end{example}

\begin{problem}
Answer the following questions with the best answer:
\begin{enumerate}
\item Explain what the print and return statements do, respectively. How are they different?
\item What is wrong with the following code?
\begin{lstlisting}
Grocery List = ['pineapple', 'orange juice', "avocados", "pesto sauce"]
for i in range(Grocery List)
if i % 2 = 0
print i, Grocery List(i)
\end{lstlisting}
provided you want the following output:
\begin{lstlisting}
0 pineapple
2 avocados
\end{lstlisting}
(Note: This code may contain one error, many possible errors, or no errors at all)
\end{enumerate}
\end{problem}


\subsection*{Data Structures}
\begin{example}
The list and set data types have many methods. 
\begin{lstlisting}
>>> my_list = ["a", "b", "c"]

>>> my_list.append("d") #adds "d" to the end of the list.

>>> my_list.insert(0, 123) #inserts 123 at index 0 in list.

>>> my_list.remove("c") #removes item "c" from the list.

>>> my_list.sort() #sorts the items of the list in place.

>>> my_list.reverse() #reverses the elements of the list in place. 

>>> del my_list[3] #deletes the item at index 3.

>>> gym_members = set(["Doe, John", "Smith, Jane", "Brown, Bob", "Jones, Sally"])

>>> gym_members #note that sets are not ordered by the user
{'Brown, Bob', 'Doe, John', 'Jones, Sally', 'Smith, Jane'}

>>> gym_members.add("Lytle, Josh") #sets are very efficient with insertion

>>> gym_members.add("Doe, John") #sets do not allow for duplicates

>>> gym_members
{'Brown, Bob', 'Doe, John', 'Jones, Sally', 'Lytle, Josh', 'Smith, Jane'}

>>> library_members = set(["Lytle, Jane", "Henriksen, Ian", "Smith, Jane", "Grout, Ryan"])

>>>: library_members.discard("Smith, Jane") #sets are very efficient with removal

>>> library_members.add("Lytle, Josh") 

>>> library_members
>>> {'Grout, Ryan', 'Henriksen, Ian', 'Lytle, Jane', 'Lytle, Josh'}

>>> inter = set.intersection(gym_members, library_members) # set objects also support mathematical operations like union, intersection, difference, and symmetric difference.

>>> inter
{'Lytle, Josh'}

>>> library_members & gym_members #The & symbol takes the intersection of the two sets.
{'Lytle, Josh'}



\end{lstlisting}

\end{example}


\begin{problem}
What sequence of commands would you use to implement the following?

\begin{enumerate}
\item Create an empty list. 
Add 5 integers to your list. 
Cast the integer at index 3 as a float.
Remove the integer at index 2. 
Sort your list backwards. 
\item Create two empty sets.
Add 5 integers to the first set and 5 strings to the second set.
Take the union of these sets.

\end{enumerate}
\end{problem}

\section*{Specifications}
We suggest that you submit your \li{solutions.py} file using the following format.
\begin{lstlisting}
# Problem 1
'''
1. 
2. 
3. 
4. 
5. 
6. 
'''

# Problem 2	
'''
1. 
2.
3. 
4. 
5. 
6. 
7. 
'''

# Problem 3
'''	
1.
2.
'''

# Problem 4
'''
1. 
2. 
'''
\end{lstlisting}	
