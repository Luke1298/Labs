\lab{Python}{Web Technologies}{Web Technologies}
\objective{Learn about serialization and markup languages.
You will also learn how to communicate using the HTTP protocol.}
\label{lab:webtech}

Since the dawn of computing, the ability to make computers talk to each other has captivated the minds of many.


The web is part of everyday life.
Ever
It is important that 

\section*{Serialization}
How would you store a Python list or dictionary outside of the interpreter?
Suppose we calculated some results and stored them in a list that we wanted to send to a friend or colleague?
However we choose to store our list, we need to be able to load it back into the Python interpreter and use it as a list.
What if we wanted to store more complex objects?
The process of serialization seeks to address this situation.
Serialization is the process of storing an object and its properties in a form that can be saved or transmitted and later reconstructed back into an identical copy of the original object.

\subsection*{JSON}
JSON, pronounced ``Jason'', stands for \emph{JavaScript Object Notation}.
It is easy for both humans and machines to read and write the format.
Despite its name, it is a completely language independent format.
JSON is built on top of two types of data structures: a collection of key/value pairs and an ordered list of values.
These data structures are more familiarly called dictionaries and lists in Python.
Python's standard library has a module that can read and write JSON.
Most JSON libraries, though, have a fairly standard interface.
If performance is critical, there are Python modules for JSON that are written in C such as ujson and simplejson.

Let's begin with an example.
\begin{lstlisting}
>>> import json
>>> json.dumps(range(5))
'[0, 1, 2, 3, 4]'
>>> json.dumps({'a': 34, 'b': 483, 'c':"Hello JSON"})
'{"a": 34, "c": "Hello JSON", "b": 483}'
\end{lstlisting}
As you can see, the JSON representation of a Python list and dictionary are very similar to their respective string representations.
You can also see that each JSON message is enclosed in a pair of curly braces.
We can even nest multiple messages.
\begin{lstlisting}
>>> a = """{"car": {
    "make": "Ford",
    "model": "Focus",
    "year": 2010,
    "color": [255, 30, 30]
        }
    }"""
>>> t = json.loads(a)
>>> print t 
{u'car': {u'color': [255, 30, 30], u'make': u'Ford', u'model': u'Focus', u'year': 2010}}}
>>> print t['car']['color']
[255, 30, 30]
\end{lstlisting}

Most JSON libraries support the dump[s]/load[s] interface.
To generate a JSON message, we use \li{dump} which will accept the Python object and generate the message and write it to a file.
\li{dumps} does the same, but just returns the string rather than writing it to a file.
To perform the inverse operation, we use \li{load} or \li{loads} for reading from a file or string respectively.

Many websites and web APIs make extensive use of JSON.
Twitter, for example, return JSON messages for all queries.

\subsection*{XML}
XML is another data interchange format.  It is a markup language rather than a object notation language.
To understand XML, we need to understand what tags are.
A tag is a special command enclosed in angled brackets (<>) that describe something about the data it encloses.
For example, we can represent our car from above in the XML below.
\begin{lstlisting}[language=XML]
<car>
    <make>Ford</make>
    <model>Focus</model>
    <year>2010</year>
    <color model='rgb'>255,30,30</color>
</car>
\end{lstlisting}
There are two strategies for reading XML data.
We can read the data as a tree or as a stream.
Since XML is a hierarchical storage format, it is very easy to build a tree of the data.
The advantage is random access to any part of the document at any time.
However, all of the XML must be loaded into memory to build this tree.
Large XML files will consume huge amounts of memory if read as a tree.

To alleviate the burden of loading an entire XML document into memory all at once, we can read the file sequentially.
When streaming the XML data, we are only reading a small chunk of the file at a time.
There is no limit to size of XML document that we can process this way as memory usage will be constant.
However, we sacrifice the random access that the tree gives us.

\subsection*{DOM}
The DOM (Document Object Model) API allows you to work with an XML document as a tree.
Python's XML module includes two version of DOM: \li{xml.dom} and \li{xml.minidom}.
MiniDOM is a minimal, more simple implementation of the DOM API.

The motivation behind DOM is to represent an XML as a hierarchy of elements.
This is accomplished by building a tree of the elements as the XML tags are read from the file.
DOM is useful when we want random access to all of the XML document.
This requires loading the entire file into memory.
If you have a large XML file (a couple of megabytes)

DOM reads an entire XML into memory and builds a tree with the tag hierarchies on every parse.
For large XML files, this could lead to massive memory consumption.
The DOM tree of the car above would have \li{<car>} at the root element.
This root element would have four children, \li{<make>}, \li{<model>}, \li{<year>}, and \li{<color>}.
We would traverse this DOM tree just like we would any other tree structure.
DOM trees can be searched by tag as well.

\subsection*{SAX}
SAX, Simple API for XML, is essentially an XML state machine.
This method of reading an XML file requires that you read the XML file as the parser would.
It is a very fast, efficient way to read an XML file.
The main advantage of this method for reading an XML file is memory conservation.
A SAX parser reads XML sequentially instead of all at once.
It doesn't need to load the entire file into memory.

As the SAX parser iterates through the file, it emits events at either the start or the end of tags.
You can provide functions to handle these events.


\subsection*{ElementTree}
ElementTree is Python's unification of DOM and SAX into a single, high-level API for parsing and creating XML.
ElementTree provides a SAX-like interface for reading XML files via its \li{iterparse()} method.
This will have all the benefits of reading XML via SAX.
In addition to stream processing the XML, it will build the DOM tree as it iterates through each line of the XML input.
ElementTree provides a DOM-like interface for reading XML files via its \li{parse()} method.
This will create the tag tree that DOM creates.

We will demonstrate ElementTree using the following XML.
\lstinputlisting[style=FromFile,language=XML]{contacts.xml}

First, we will look at viewing an XML document as a tree similar to the DOM model described above.
\begin{lstlisting}
import xml.etree.ElementTree as et

f = et.parse('contacts.xml')

# manually traversing the tree
# we iterate through the element directly
# getchildren() is old and deprecated (not supported).
root = f.getroot()
children = list(root) # root has three children
person0 = children[0]
fields = list(person0) # the children elements of person0

# we can search the entire tree for specific elements
# searching for all tags equal to firstname
for n in root.iter('firstname'):
    print n.text
    
# we can also filter with multiple tags 
# notice we use a set lookup in the conditional inside the generator expression
fields = {'firstname', 'lastname', 'phone'}
fi = (x for x in root.iter() if x.tag in fields)
for n in fi:
    print n.text
    
# we can even modify the document tree inplace
# let's remove Thor
# refer to the documentation of ElementTree for adding elements
for n in root.findall("person"):
    if n.find("firstname").text == 'Thor':
        root.remove(n)

# verify that Thor is really gone
for n in root.iter('firstname'):
    print n.text
\end{lstlisting}

Next, we will look at ElementTree's \li{iterparse()} method.
This method is very similar to the SAX method for parsing XML.
There is one important difference.
ElementTree will still build the document tree in the background as it is parsing.
We can prevent this by clearing each element by calling its \li{clear()} method when are finished processing it.
\begin{lstlisting}
f = et.iterparse('contacts.xml') # this is an iterator
for event, tag in f:
    print "{}: {}".format(tag.tag, tag.text)
    tag.clear()
    
# we can get both start and end events
# however, start events are mostly useful for looking at attributes
# or to trigger some other action on element starts.
# The element is not guarenteed to be complete until the end event.
for event, tag in et.iterparse('contacts.xml', events=('start', 'end')):
    print "{} {}<{}>: {}".format(event, tag.tag, tag.attrib, tag.text)
\end{lstlisting}


\section*{TCP/IP}
The most common protocol that computers today use to communicate is TCP, Transmission Control Protocol.
It is used in everything from checking email, uploading files, and browsing web pages.
Being one of the core protocols of the internet protocol suite, it is often referred to as TCP/IP.
There are four layers to the TCP/IP protocol.
\begin{enumerate}
\item Network Interface: This is the level of networking hardware such as routers and switches.
\item Internet: This level contains a group of protocols that handle routing and movement of data on a network.
\item Transport: The critical protocols that define basic high level communication between two computers.
The two most common protocols in this layer are TCP and UDP.  TCP is by far the most widely used due to its reliability.
UDP, however, trades reliability for low latency.
\item Application: Software that utilize the transport protocols to move information between computers.
This layer includes protocols important for email, file transfers, and browsing the web.
\end{enumerate}

TCP/IP has its origins in the mid 1970s.
The TCP protocol dictates how computers connect to each, exchange bits of information called packets, and then close the connection.
TCP/IP is very reliable, ordered, and error-checked.

Python has support for communicating via TCP in the standard library.
A short demonstration will aide our discussion.
\lstinputlisting[style=FromFile]{tcp_server.py}

\lstinputlisting[style=FromFile]{tcp_client.py}

\section*{HTTP}
HTTP stands for Hypertext Transfer Protocol.
The protocol is centered around a request and response paradigm.
A client makes a request to a server and the server replies with response.
HTTP is an application layer networking protocol.
It usually relies on the underlying TCP protocol to provide networking capabilities.
There are several methods defined for HTTP, but the two most common are GET and POST.
GET is designed to request information from a server.
POST is designed to request that a server accept additional data.

Let's look at a GET request.
\begin{lstlisting}[language=HTML]
GET / HTTP/1.1
Host: acme.byu.edu
\end{lstlisting}

\begin{problem}
You are assigned to implement a message board client.

You client should accept a list of commands and performed the associated action.
\begin{table}[H]
\begin{tabular}{|l|l|}
\hline
Command & Action \\
\hline
/join & Join an existing channel \\
/leave & Leave a channel \\
/nick & Change the user's nickname \\
/quit & Quit the message board \\
/pull & Check for new messages and display them to the user\\
/push & Publish a new message on the message board \\
/create & Create a new channel on the server \\
\hline
\end{tabular}
\end{table}
\end{problem}
