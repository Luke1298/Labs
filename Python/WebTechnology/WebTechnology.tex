\lab{Python}{Web Technologies}{Web Technologies}
\objective{Learn about HTTP, XML, and WSGI.
You will learn how to communicate using HTTP, parse XML, and use WSGI for web applications.}
\label{lab:webtech}

Since the dawn of computing, the ability to make computers talk to each other has captivated the minds of many.


The web is part of everyday life.
Ever
It is important that 

\section*{TCP/IP}
The most common protocol that computers today use to communicate is TCP, Transmission Control Protocol.
It is used in everything from checking email, uploading files, and browsing web pages.
Being one of the core protocols of the internet protocol suite, it is often referred to as TCP/IP.
There are four layers to the TCP/IP protocol.
\begin{enumerate}
\item Network Interface: This is the level of networking hardware such as routers and switches.
\item Internet: This level contains a group of protocols that handle routing and movement of data on a network.
\item Transport: The critical protocols that define basic high level communication between two computers.
The two most common protocols in this layer are TCP and UDP.  TCP is by far the most widely used due to its reliability.
UDP, however, trades reliability for low latency.
\item Application: Software that utilize the transport protocols to move information between computers.
This layer includes protocols important for email, file transfers, and browsing the web.
\end{enumerate}

TCP/IP has its origins in the mid 1970s.
The TCP protocol dictates how computers connect to each, exchange bits of information called packets, and then close the connection.
TCP/IP is very reliable, ordered, and error-checked.

Python has support for communicating via TCP in the standard library.
A short demonstration will aide our discussion.
\lstinputlisting[style=FromFile]{tcp_server.py}
\lstinputlisting[style=FromFile]{tcp_client.py}


\section*{HTTP}
HTTP stands for Hypertext Transfer Protocol.
The protocol is centered around a request and response paradigm.
A client makes a request to a server and the server replies with response.
HTTP is an application layer networking protocol.
It usually relies on the underlying TCP protocol to provide networking capabilities.
There are several methods defined for HTTP, but the two most common are GET and POST.
GET is designed to request information from a server.
POST is designed to request that a server accept additional data.

Let's look at a GET request.
\begin{lstlisting}[language=HTML]
GET / HTTP/1.1
Host: acme.byu.edu
\end{lstlisting}

\section*{JSON}

\section*{XML}
ElementTree

\section*{Web Applications with WSGI}