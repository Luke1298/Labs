\lab{Python}{The Standard Library}{Standard Library}
\objective{Become familiar with the Python standard library}

Python has a very rich set of tools available by default.
There are about 80 built-in functions that are always present in any Python environment.
In addition to these 80 core functions, the standard library includes thousands of useful routines and classes that cover almost every imaginable need;
Python has a ``batteries included'' philosophy.
This philosophy makes complex tasks almost trivial to implement using the standard library.
There are many sophisticated and robust modules available therein.

\section*{File Objects in Python}
One of the built-in functionalities of Python is working with files.
Python has a useful file object which acts as an interface to all kinds of different streams.
File objects are created using the built-in \li{open} command.
The \li{open} command accepts up to three arguments, or parameters: filename, mode, and buffering.
The mode determines the kind of access to use when opening the file.
Possible mode strings are:
\begin{description}
\item \li{'r'} Opens a file for read-only access.
This is the default mode.
\item \li{'w'} Opens a file for write-only access.
This mode will create the file if it doesn't exist, or overwrite the file if it does exist.
\item \li{'a'} Opens a file for appending.
The file pointer is at the end of the file.
Creates a new file if none exists.
\end{description}
There are also two possible modifiers for the mode: \li{b} (binary) and \li{+} (read-write access).
Binary mode will open the file as-is, without any newline conversions.
The plus mode will open a file for reading and writing.
It is important to know that the modes \li{r+} and \li{w+} are not equivalent!
If your file cannot be opened for any reason, an exception is raised (usually an \li{IOException}).
Every file object has several attributes and methods (Table \ref{table:fileattribs}).
\begin{table}
\begin{tabular}{|l|l|}
\hline
Attribute & Description \\
\hline
\li{closed} & True if file object is closed. \\
\li{mode} & The access mode used to open the file object. \\
\li{name} & The name of the file. \\
\hline
\hline
Method & Description \\
\hline
\li{close()} & Flush any delayed writes and close the file object. \\
\li{flush()} & Flush any delayed writes.  File object remains open. \\
\li{read()} & Read the next string of the file. \\
\li{readline()} & Read a line of the file. \\
\li{readlines()} & Read lines of the file until end of file. \\
\li{seek(offset)} & Place the file pointer at a certain position within the file. \\
\li{tell()} & Return the current position of the file pointer. \\
\li{write()} & Write a string to the file. \\
\li{writelines()} & Write a sequence of strings to the file. \\
\hline
\end{tabular}
\caption{File object attributes and methods.}
\label{table:fileattribs}
\end{table}

\begin{info}
When opening a file in text mode, Python automatically converts all end-of-line characters to the newline character, \li{\\n}.
However, this causes strange issues on Windows machines since the newline character on Windows is \li{\\r\\n}.
To prevent this automatic conversion in text mode, we can open the file in binary mode or, in Python 2.7, use the universal newline mode.
\begin{lstlisting}
# Open a file, read-only, in binary mode.
# No newline conversion.
open(filename, 'rb')
# 'rU' indicates universal newline support.
# Recognizes any newline character.
open(filename, 'rU')
\end{lstlisting}
\end{info}


We can open a file as follows
\begin{lstlisting}
f = open('filename.txt','r')
# This will iterate through your entire file and print each line.
for line in f:
        print line
f.close()
\end{lstlisting}
However, it is preferable to open files in the following manner.
Notice that we don't have to explicitly close the file;
the moment we exit the \li{with} block, the file is automatically closed safely.
We generally prefer to use the method below because it guarantees that our files will be closed properly, even if an unexpected error occurs.
\begin{lstlisting}
with open('filename.txt', 'r') as f:
    for line in f:
        print line
\end{lstlisting}

To write to a file, we would do something similar to the following.
\begin{lstlisting}
num = 17
with open('output.txt', 'w'):
    f.write("My favorite number: ")
    # You can only write strings to a file.
    f.write(str(num))
\end{lstlisting}

Note that we can declare variables inside the \li{with} block that are accessible outside that block.
\begin{lstlisting}
# Read your file using a with statement.
with open('filename.txt', 'r') as f:
        read_data = f.read()
# read_data is accessible outside of the with-block.
print read_data
\end{lstlisting}

One of the reasons Python is so useful as a scientific computing platform is because it is not limited to scientific computing alone.
Python has a very large and comprehensive standard library that is available with almost every Python environment.
The standard library is comprised of over a hundred different modules that provide extra functionality to Python.
Python modules are simply Python source files where classes, functions, and variables are defined.
We can import any Python source file as a module and access the objects defined in it.
In this lab, we will look at some of the modules available in the Python standard library.

\section*{Namespaces and Importing Modules}
Before we can understand what a namespace is, we need to grasp the fundamental concept of names in Python.
What we normally call variables in other languages are \emph{names} in Python, but with a few minor differences.
Names act as labels for Python objects in memory.
An object can have one or more names.
\begin{lstlisting}
a = 5
b = [1, 2, 3, 4]
\end{lstlisting}
In the code above, there exist two Python objects in memory: an integer and a list.
We attach the name \li{a} to the integer object and the name \li{b} to the list object.
If we make the assignment \li{a = b}, the names \li{a} and \li{b} now both point to the same list object in memory.
Since functions in Python are simply objects, we can assign names for those too!
\begin{lstlisting}
def func():
    return 42
f = func
\end{lstlisting}
Whenever we want to call \li{func()}, we can call \li{f()} instead.

A \emph{namespace} is simply a collection of names.
Now, a \emph{module} in Python is simply a Python file with code.
Each module gets its own namespace, so every name in that namespace must be unique.
You can think of module as simply a collection of names assigned to various Python objects.
Since every module has its own namespace, these namespaces are completely isolated.
The main Python program gets a special namespace called \li{__main__}.
The name of a namespace is stored in the \li{__name__} attribute.
This is incredibly useful for checking to see if a namespace we are in has been imported, or is being run directly by the interpreter.
\begin{lstlisting}
if __name__ == "__main__":
    print "I am being run from the python interpreter."
elif __name__ != "__main__":
    print "I have been imported by another python module."
    print __name__
\end{lstlisting}

There are three ways to import names from a module.
\begin{description}
\item \li{import module [as alias]} This will import a module's namespace into the main namespace.
We can also use \li{as alias_name} to define an optional alias if the module name is too long.
The method names in the module are accessed by \li{module.name}.
\item \li{from module import name [as alias]} This will import a particular method name in the module directly into the main namespace.
This might replace a name in the main namespace if the imported name is not unique.
The names imported this way can be accessed directly by \li{name}.
\item \li{from module import *} This will import all names in the module's namespace directly into the main namespace.
This form is generally considered bad practice as it completely circumvents all the protections that namespaces provide.
However convenient this form might be, we strongly discourage to importing modules in this way as it can lead to many unknown name conflicts.
\end{description}

\begin{warn}
Be careful when importing modules using the \li{from module import item} syntax.
Consider the following scenario.
Suppose we defined the following two function in a file named \texttt{my\_functions.py}.
\begin{lstlisting}
def min(seq):
    return 0
def max(seq):
    return 1
\end{lstlisting}

We import the functions into our main program as follows
\begin{lstlisting}
from my_functions import min, max

# Calculate the median value of a sequence.
def calc_median(seq):
    return (min(seq)+max(seq))/2.
\end{lstlisting}

Suddenly our main program doesn't work as expected anymore!  What happened?
The functions \li{min()} and \li{max()} are defined in the core Python language.
However, when we imported the functions from our module, we ended up overwriting Python's built-in \li{min()} and \li{max()} functions.
Now, whenever we call \li{min()}, it will always return 0!
In this case, there is no way to revert back to the Python functions without restarting the interpreter.
This is a classic demonstration of name collisions.
\end{warn}

\section*{\texttt{sys} Module}
The \li{sys} module allows us to access information specific to the system running Python.
For example, we often import \li{sys} to get the list \li{sys.argv} which is a list of arguments passed to the current environment.
It is sometimes useful to access these values when writing programs that are run from command line.
Many programs are written to execute differently based on the various arguments and options specified at execution.
Here is an example.
\lstinputlisting[style=fromfile]{square.py}

When we run from the command line with a single argument, the code will print the name of the program along with the square of the number passed as the first argument.
If we execute the following script from the command line as follows:
\begin{lstlisting}[style=ShellInput]
python square.py 4
\end{lstlisting}
We should get the following output.
\begin{lstlisting}[style=ShellOutput]
square.py
16.0
\end{lstlisting}

\begin{info}
Parsing means extracting and analyzing an input by parts. \li{sys.argv} should only be used for the most simple command line argument parsing.
Python has another module in the standard library, \li{argparse}, which is many times more powerful for accepting and parsing complex command line arguments.  This module allows you to build user-friendly command line interfaces.  Using \li{argparse} is the recommended way to parse command line arguments.
\end{info}

\section*{\texttt{math} Module}
The \li{math} module and its companion, \li{cmath} (for complex numbers), are very valuable modules.
Common mathematical functions are defined in these modules, including \li{cos}, \li{sin}, \li{log}, and \li{sqrt}.
These functions wrap around the functionality of the C math library.
\begin{lstlisting}
import math
# Return the sine of 4 radians.
math.sin(4)
\end{lstlisting}

\begin{problem}
Import the \li{math} module and \li{cmath} module. Write a function that prints out both the complex and floating point results for $\sqrt{5}$.  What are the results for $\sqrt{-5}$?
\end{problem}

\section*{\texttt{random} Module}
The \li{random} module contains many helpful functions for obtaining random numbers.
These ``random'' numbers are not actually completely random; however, they are seemingly-random enough for most applications.
Some of the more commonly used methods of the \li{random} module are shown below.
\begin{lstlisting}
import random

# Get a random integer from the interval [5, 12].
random.randint(5, 12)

# Get a random integer from a range of integers [5, 12).
random.randrange(5, 12)

# Get a random element from a sequence.
random.choice("This is a sentence")

# Get a random sample of length n from a sequence.
random.sample(range(50), 10)

# Randomly shuffle a sequence in place.
random.shuffle(range(15))
\end{lstlisting}
\begin{comment}
The \li{random} module has functions that can sample from a variety of different statistical distributions such as: uniform, normal, beta, gamma, exponential, etc.
\end{comment}

\begin{problem}
Write a function that will generate a random bijective map between two lists of arbitrary items that are equal in length. Your
output should be a list of tuples of the form \li{(list1[i], list2[j])}, where \li{list1[i]} maps to \li{list2[j]}.
\label{prob:random_map}
\end{problem}

\section*{\texttt{csv} Module}
We commonly use comma separated value (CSV) files to exchange data between databases and tables.
Python has a very useful module for reading and writing data as comma separated values.
This \li{csv} module provides \li{reader} and \li{writer} objects.
There are also analogous \li{DictReader} and \li{DictWriter} objects that use dictionaries for handling data.

Using these reader and writer objects, we can define the format of our CSV file.
Contrary to what the name implies, CSV files can be delimited with any character.
A delimiter is the special character that separates fields in a line. Commas are typical, but tabs and spaces are two other popular delimiting characters.
\begin{lstlisting}
# Set the delimiter to tabs.
csv_reader = csv.reader(csv_file, delimiter='\t')
\end{lstlisting}

We can also define the characters that separate fields, terminate lines, escape special characters, and enclose strings.
The \li{csv} module refers to these special settings as \emph{dialects}.
The module comes with two dialects ready to use, excel (Windows Excel formatting) and excel-tab (Excel formatting, but tab-delimited).
\begin{lstlisting}
# Change the formatting dialect to CSV.
csv_reader = csv.reader(csv_file, dialect='excel')
# Modify the formatting characters.
csv_reader = csv.reader(csv_file, quotechar='*', lineterminator='\n')

\end{lstlisting}

To print the contents of a CSV file, \li{test.csv}, we would do the following.  A CSV reader will parse each line of a CSV file into a list of items.
\begin{lstlisting}
import csv

# Open test.csv as read-only.
with open('test.csv', 'r') as csv_file:
    # Create a csv reader object.
    csv_reader = csv.reader(csv_file)
    # Work with the reader object alone.
    for line in csv_reader:
        print line
\end{lstlisting}

Writing with a CSV \li{writer} object is very similar to writing with a regular file object.  Notice that we only need to pass a sequence of items to the CSV writer.  The CSV writer will take care of formatting things properly before writing it to the CSV file.
The following code demonstrates how this may be done.
\begin{lstlisting}
contents = [["Column 1", "Column 2", "Column 3"],
            [0,1,2], [3, 2, 1], [4,5,2], [68, 38, 99]]

# Open test_out.csv as a write-only file.
# Be careful! This will overwrite test_out.csv if it already exists.
with open('test_out.csv', 'w') as csv_file:
    # Create a csv writer object.
    csv_writer = csv.writer(csv_file)
    # Here, record is a loop variable that represents the rows.
    for record in contents:
        # Write rows using the writer object.
        csv_writer.writerow(record)
\end{lstlisting}

\section*{\texttt{timeit} Module}
This module is used to time the execution of small bits of Python code.
It is helpful to time lines of code in Python using the \li{timeit} module because it avoids a number of common pitfalls in measuring execution time; for example, it limits errors caused by startup and shutdown times by running the code repeatedly. In IPython, we usually time code using the built-in \li{\%timeit} function. Prefacing a line of code with \li{\%timeit} times that line; to time an entire cell of code, preface it with \li{\%\%timeit}. However, this will not work in all environments, notably in a command line setting, so it is important to know how to time code without the \li{\%timeit} function.

We will now define a function that times the execution of another function \li{f} and returns the minimum runtime for one call of the \li{f}. This will be useful in other labs when you are asked to time the execution of your solutions. Before we can do this, we need to know what a wrapper function is. A wrapper function is so named because it wraps a function and its arguments into a function object that can be called by the \li{timeit} function. In Python, this is done by declaring a \li{lambda} function, which is a temporary, anonymous function that takes no arguments.

\begin{lstlisting}
import timeit

# time_func takes as its arguments the name of a function f, a tuple of the
# arguments of f, a dictionary of the keyword arguments of f, and two 
# keyword arguments.
def time_func(f, args=(), kargs={}, repeat=3, number=100):
	# Wrap f into pfunc
	pfunc = lambda: f(*args, **kargs)
	# This is where we actually use the timeit module
	T = timeit.Timer(pfunc)

	# Time f several times, return the name of f and the minimum runtime
	try:
		#Note that repeat is a timeit module function
		_t = T.repeat(repeat=repeat, number=int(number))
		runtime = min(_t)/float(number)
		return f.__name__, runtime
	# Print an error statement if something goes wrong
	except:
		T.print_exc()
\end{lstlisting}

Remember that keyword arguments are arguments that are given a default value in the function declaration (after the non-keyword arguments), but can be overwritten. Therefore, when calling \li{time_func}, the dictionary of keyword arguments of \li{f} (where the names of the keyword arguments are the keys) is optional, and so are the keyword arguments of \li{time_func} itself.
\li{repeat} defines the number of time to test the code, while \li{number} defines the number of times the code is run per test. If your code is small and simple, you can set the value of \li{number} very high without causing problems, but for large, slower code, \li{number} will need to be smaller, or you may have to wait a long time to get your timing results.

For further information on the \li{timeit} module, see the documentation at \url{http://docs.python.org/2/library/timeit.html}. Resources found by a quick Google search may also be helpful.

\begin{comment}
\begin{problem}
For most timing situations, we rely on IPython's \li{\%timeit} function.
Since this is a special function only available inside IPython, however, we will write a more general timer that works anywhere with any function.
The solution to this problem will be useful in other labs where you will be asked to time the execution of your coded solutions.
Write a function that will time the execution of another function.
Your function should return the minimum runtime for one call of the function.

Because of the way that Python's \li{timeit} module works, we can either evaluate a string of Python code or use a \emph{callable} function with no arguments.
For this problem, we will use the the callable function approach.
This essentially means we have to wrap the function we are timing and all of its arguments into a function object that can be called by \li{timeit}.
This wrapping fixes the arguments of the function so we can call it without any arguments later.
This is done by declaring a Python \li{lambda}, or temporary, anonymous  function, which takes no arguments.  \li{pfunc} is a wrapper function for $f$ because its primary job is to call $f$ with little or no additional computation.
In our program, we use \li{pfunc} only to modify how we call $f$.
\begin{lstlisting}
pfunc = lambda: f(*args, **kwargs)
\end{lstlisting}
where \li{args} is a tuple of arguments and \li{kwargs} is a dictionary of keyword arguments to the function, $f$.
You can refresh your memory of functions and their arguments and keyword arguments in sections 4.7.1-4.7.4 of the official Python tutorial.
\end{problem}
\end{comment}


\section*{\texttt{collections} Module}
This module defines several specialized data structures to use in addition to the built-in Python data structures.

Named tuples are designed to help improve code readability in some cases.
Standard tuples in Python are accessed by index while named tuples allow access via index or by  fieldname (which defines the tuple elements and acts like a key).

A double-ended queue, or deque (pronounced ``deck''), can be thought of as a deck of cards.
Inserting and removing elements from either end is highly efficient.
Python's deque implementation only allows insertions at the left and right ends of the data structure (which is standard for deques).
This differs from a list, which allows insertions anywhere, but is very inefficient for all but right end insertions.

Counter objects are very efficient at counting items.  They behave like a Python dictionary.  Counts are allowed to be any integer, including 0 and negative values.

Ordered dictionaries are exactly like standard dictionaries except for one important difference:
ordered dictionaries remember the order in which key-value pairs were added to the data structure.
When iterating over an ordered dictionary, the items are returned in the order they were added from first to last.

Default dictionaries are a very convenient way to set a default value for all new keys in a dictionary.
While this can be done with standard dictionaries using the \li{setdefault()} method, using a default dictionary is simpler and faster.

For further information and examples of how to use these container objects, see section 8.3 of the official Python tutorial (\url{https://docs.python.org/2/library/collections.html}).

\begin{problem}
Write three functions that will rotate the elements of a deque and a list respectively.
\begin{enumerate}
\item Use a for-loop to rotate a deque.
\item Use the rotate method of a deque.
\item Use a for-loop to rotate a list.
\end{enumerate}

To rotate a sequence, remove elements from the right end one by one and insert them on the left end.
Compare the timings you obtain from a deque and a list of 10000 elements, using the timer function defined earlier in this lab.
\end{problem}

\begin{problem}
Write a function that takes in an integer maxInt and an integer n.
Make a list of random integers between 0 and maxInt of length $n$, then use a Counter object to count how times each number appears in the list.
Return a dictionary where the keys are the distinct integers in the list and the values are the frequency corresponding to each integer.
\end{problem}

\section*{\texttt{itertools} Module}
There is a very powerful library in the Python Standard Library that is built around the concept
of generators and iterators.  The functions in \li{itertools} are fast and memory efficient and are designed to be used as
building blocks in larger functions. 
This efficiency is especially important in Python where code is generally executed more slowly than in a compiled language like C or Fortran; an efficient way of doing things goes a very long way.

\subsection*{Generators}
Before discussing this module, let's briefly understand what a generator is, and the difference between generators and iterators.

Lists, tuples, sets and dictionaries are all examples of sequences in Python. It is frequently useful and necessary to visit all the elements of a sequence once; the process of doing so is called \emph{iteration}.
Each of the Python types we have mentioned define their own iterators, which you use every time you execute the statement \li{for <elem> in <sequence>}.

Python also has a special type of object called a \emph{generator}.
Similar to an iterator, a generator returns a sequence of values, but unlike an iterator, a generator computes the next value in the sequence and returns it instead of storing the entire sequence and returning one element at a time. It never has to store all the values in a sequence. When is this helpful?

\begin{itemize}
\item \emph{Iterating through only part of a sequence.}
It is inefficient to create an entire sequence if we know that we will not need all of it.
Representing the sequence as a generator can avoid excess memory use and computation.
\item \emph{Iterating through a sequence once.} Consider the statement
\li{sum([i for i in range(1000) if i\%2 == 0])}.
We are creating two sequences just to iterate through each them once and never use them again.
We can make this more efficient using generators:
\li{sum(i for i in xrange(1000) if i\%2 == 0)}
Notice that we have used syntax similar to a list comprehension.
The line \li{(i for i in xrange(1000) if i\%2 == 0)} will define a generator object similar to the list made by \li{[i for i in xrange(1000) if i\%2 == 0]}.
The solution using generators will often execute faster, and it will almost always be more memory efficient.  Consider using generators for any function that reduces a sequence to a single value.  Examples of such functions are \li{min()}, \li{max()}, and \li{sum()}.
\item \emph{Calculating large sequences.}  A sequence must be stored somewhere in computer memory.
If the sequence is large, we could exhaust all available memory.
\item \emph{Calculations involving infinite sequences.}  Pre-computed sequences are necessarily finite; we simply cannot do computations involving infinite sequences with normal iterators. We cannot create a list that stores all natural numbers, but we can create a generator that returns the next natural number every time it is called.
\end{itemize}

Now that we understand what generators are and why they are important, we can discuss how to create and use them. Python uses the \li{yield} keyword to define a generator. As an example, consider the following, where we re-implement Python's \li{range} function as an iterator and a generator.
\begin{lstlisting}
def range_iter(start, stop, step=1):
    i = start
    r = []
    while i < stop:
       r.append(i)
       i = i + step
    return r
\end{lstlisting}
\begin{lstlisting}
def range_gen(start, stop, step=1):
    i = start
    while i < stop:
        yield i
        i = i + step
\end{lstlisting}
These two functions look similar, but they behave very differently.
The first function when executed will build a list one element at a time until finished
at which point it will return the entire list.  On the author's computer, this function takes about 1.43ms
to build and return a list of 10000 elements.  The second function is only marginally better,
returning the 10000 elements in approximately 1.09ms, but this gap between the two functions grows much wider as the input size is increased.

The main difference in the two functions lies in the time the function takes to execute when
it is first called.  The first function requires 1.43ms to execute the first time.  The second function,
a mere .00000041ms!  The reason for this is that the first function calculates its results and returns them all at once. The second function only creates a \emph{generator} object.  

Generator objects have several methods.
The most important and most useful method is the \li{next()} method.  Every time this method is called, 
the generator resumes from its previous state and computes the next value in the sequence. After computing and yielding this value, the generator is suspended until \li{next} is called again.

When you are writing \li{for} loops, you should use a generator whenever it is reasonable to do so.
The \li{xrange()} function that is built into Python is a generator based version of \li{range()} and is generally faster when used in \li{for} loops; you should make a habit of using \li{xrange()} instead of \li{range()}. In a for loop, you can use \li{xrange()} the same way you would use \li{range}.

In general, the generators in \li{itertools} are used like this:
\begin{lstlisting}
import itertools
a = itertools.count(5) 
# Returns natural numbers, starting at 5, increasing by 1 each time it is called.
a.next()
\end{lstlisting}

Note that while \li{count()} takes an integer as input, most generators work over \emph{iterables}, or any Python object that can be iterated through; this includes strings, lists, tuples, and so on.

\subsection*{Useful Generators in \texttt{itertools}}
The \li{itertools} module contains three main types of generators: infinite generators, shortest sequence generators, and combinatoric generators.
Some common generators are summarized in Table \ref{table:populargens}. Note that \li{islice} is particularly useful for efficiently iterating through only part of a sequence. For the argument values of and further information about these generators, as well as recipes for other useful generators, see the documentation at \url{http://docs.python.org/2/library/itertools.html}
\begin{table}
\begin{tabular}{|l|l|}
\hline
Generator & Example \\
\hline
\li{count()} & Count to infinity \\
\li{cycle()} & Cycle through a sequence indefinitely\\
\li{repeat()} & Repeat the input element indefinitely (or up to $n$ times) \\
\hline
\li{chain()} & \li{chain('ABC', 'DEF') --> A B C D E F} \\
\li{compress()} & \li{compress('ABCDEF', [1,0,1,0,1,1]) --> A C E F} \\
\li{islice()} & \li{islice('ABCDEFG', 2, None) --> C D E F G} \\
\li{imap()} & \li{imap(pow, (2,3,10), (5,2,3)) --> 32 9 1000} \\
\li{izip()} & \li{izip('ABCD', 'xy') --> Ax By} \\
\hline
\li{product()} & Return the Cartesian product of two iterables\\
\li{permutations()} & Return permutations of a sequence\\
\li{combinations()} & Return combinations of a sequence\\
\hline
\end{tabular}

\caption{Popular generators in \li{itertools} module.}
\label{table:populargens}
\end{table}

\begin{problem}
You have been tasked to write a simulator that will simulate a Hunger Games tournament (\url{http://thehungergames.wikia.com/wiki/The_Hunger_Games_(event)}).
In this problem you will be using your cumulative knowledge of the standard library to write the simulator.

You solution should accept a single argument from the command line that is a filename.  All the output you generate in your program will be written to this file. The \li{sys} module section in this lab as well as documentation on the \li{sys} module will be  very helpful.

You are provided with two files: a text file and a CSV file.
The text file has a list of events, one per line.
(Note that on Windows or OSX, you might need to open the file with universal newlines support.)
The CSV file has two columns of names.
The first column lists the male tributes and the second column lists the female tributes.

First, create a named tuple that will contain information about each of the tributes such as their name, district, and gender.

You will need to randomly pair the male and female tributes (read from the CSV file) together to form 12 pairs.
Problem \ref{prob:random_map} might yield a useful approach, but there are several valid options.

To simulate the game, you will need to do at least the following.
For each day of the simulation, each (surviving) tribute will experience one event per day. Each event should be randomly chosen from the list of events you read from the text file.
To determine the survival of a given tribute, randomly generate a number between 0 and 1 (probably with a uniform or Gaussian distribution).
If the random number is greater than a certain threshold, then the tribute survives their event for that day.
You get to decide a survival threshold common to all tributes (the probability of surviving the event).

Your goal is to use the standard library as much as possible.
To pass off, you must use the standard library for the following parts:
\begin{itemize}
\item Accepting an arbitrary filename (sys)
\item Generating random numbers (random)
\item Pairing tributes (collections, itertools, random)
\item Loading the data (csv, file I/O)
\item Simulating the games (file I/O, random, itertools)
\end{itemize}

Stop the simulation when there is at most one surviving tribute and print the winner to standard output (it will appear in the console).  It is possible that there are no surviving tributes.  If there are no surviving tributes, indicate so in your outputs.

Keep in mind good coding practice as you solve this problem. Compartmentalize tasks into functions as much as possible to make debugging easier, use comments to document what your code is doing, and most importantly, plan out how you are going to approach the problem \emph{before} you start actually coding. You may find it helpful to do this by using comments to make an outline of what you need to do and how you are going to do it.

\end{problem}

\section*{Specifications}
We suggest that you submit your \li{solutions.py} file using the following format.
\lstinputlisting[style=fromfile]{solution_specs.py}




