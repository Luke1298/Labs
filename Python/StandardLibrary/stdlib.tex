\lab{Python}{The Standard Library}{Standard Library}
\objective{Become familiar with the Python standard library}

One of the reasons Python is so useful as a scientific computing platform is because it is not limited to only scientific computing.
Python has a very large and comprehensive standard library that is available with almost every Python environment.
The standard library is comprised of over a hundred different modules that provide extra functionality to Python.
In this lab, we will look at some of these modules.

Modules can be imported via the \li{import} statement. Methods provided by the module may then be called.
If we only want to one thing from a Module, we can use \li{from module import class}.
If we want a shorter name to call a Module when using methods, we can use the word \li{as}

\begin{lstlisting}
import random as rand
from collections import deque

m.sqrt(3) #uses the square root method from the math module.
\end{lstlisting}

\section*{\texttt{math} Module}
The \li{math} module and its companion, \li{cmath} (for complex numbers), are very useful modules.
Common mathematical functions are defined in these modules, including \li{cos}, \li{sin}, \li{log}, and \li{sqrt}.
These functions wrap around the functionality of the C math library.

\begin{lstlisting}
import math

math.sin(4)
\end{lstlisting}

\begin{problem}
Import the \li{math} module and \li{cmath} module. Write a function that prints out the results of \li{math.sqrt(5)} and \li{cmath.sqrt(5)}, and \li{cmath.sqrt(-5)}.
\end{problem}

\section*{\texttt{random} Module}
The \li{random} module contains many helpful functions for obtaining random numbers.
For example, it includes an implementation of the Mersenne Twister pseudorandom number generator.
It also contains functions that operate on sequences.  For example
\begin{enumerate}
\item \li{randint()} returns a random integer from a desired interval,
\item \li{randrange()} returns a random integer from a range of integers,
\item \li{choice()} returns a random integer from a list of integers,
\item \li{sample()} returns a list of specified length of randomly sampled numbers from a list,
\item \li{shuffle()} randomly shuffles a list of numbers.
\end{enumerate}

Here's some examples of how to use the random module.

\begin{lstlisting}
import random

# get a random integer between 5 and 12
random.randint(5,12)

#get a random sample of 5 people from the list. 
people = ['Kevin', 'Jean', 'Alex', 'Ian', 'Bob', 'John', 'Melissa', 'Christy', 'Dave']
random.sample(people,5)

\end{lstlisting}


\begin{problem}
Import the \li{random} module.  Write a function that takes in a list of integers and shuffles it. Then it takes a random integer between 0 and the length of the list minus one and outputs the number that is in the suffled list at that spot.
\end{problem}

The \li{random} module can sample from several distributions such as the uniform, normal, beta, gamma, exponential, and many others.
NumPy's \li{random} module has even more capability than the standard Python \li{random} module.
We will do more with NumPy in a later lab.

\section*{Reading and Writing into Text Files}
One useful way to store data is in a text file.
This allows us to use data from outside sources as well as write strings that can be accessed outside of the python environment.

To start, we need to use the \li{open()} method to access our file.
The method takes two strings, the filename and whether we want to read, write, or both.
We use \li{'r'} for reading, \li{'w'} for writing, and \li{'r+'} for reading and writing.
After we finish working with a file, we close the connection with the \li{close()} method.

\begin{lstlisting}
f = open('filename.txt', 'r')
#f is now how we access our file.
# We can read one line at a time by using
f.readline()
#We can read the entire file by using
f.read()

f.close()
\end{lstlisting}

The methods \li{read()} and \li{readline()} return strings.
We can also iterate through our file and look at each line.

\begin{lstlisting}
f = open('filename.txt','r')
#This will iterate through our entire file and print each line.
for line in f:
	print line
f.close()
\end{lstlisting}

Writing to a file is similar, we open the file and then use methods. 
However, the \li{write()} method will only accept strings, so everything we write needs to be a string.

\begin{lstlisting}
num = 17
f = open('output.txt','w')
f.write('My favorite number: ')
s = str(num) #convert num to a string
f.write(s)
f.close()
\end{lstlisting}

Working with files can often cause errors if we don't know exactly what is in the file.
To avoid errors, we can use \li{try} and \li{except} blocks, or use Python's \li{with} statement.
This will close the file when we are done without us having to call the \li{close()} method even if an exception is raised while working.

\begin{lstlisting}
#read our file using a with statement
with open('filename.txt', 'r') as f:
	read_data = f.read()
\end{lstlisting}



\section*{\texttt{csv} Module}
Comma separated value (CSV) files are commonly used for exchanging data between databases and tables. 
Python has a very useful module for reading and writing data as comma separated values.
The \li{csv} module provides \li{reader} and \li{writer} objects.
There are also analogous \li{DictReader} and \li{DictWriter} objects that use dictionaries for handling data.

For example, download \li{test.csv} and run the following code to print its contents.
\begin{lstlisting}
import csv

#print the contents of a csv_file
with open(`test.csv', `r') as csv_file:
    csv_reader = csv.reader(csv_file)
    for line in csv_reader:
        print line
\end{lstlisting}

%Writing with a CSV \li{writer} object is very similar to writing with a regular file object.
We can also write to a \li{csv} file using a CSV \li{writer} object.  The following code demonstrates how this may be done.

\begin{lstlisting}
contents = [["Column 1", "Column 2", "Column 3"],
            [0,1,2], [3, 2, 1], [4,5,2], [68, 38, 99]]
with open(`test_out.csv', `w') as csv_file:
    csv_writer = csv.writer(csv_file)
    for record in contents:
        csv_writer.writerow(record)
\end{lstlisting}

\begin{problem}
Practice reading and writing a CSV file.
Read \texttt{test.csv} and write it followed by your name and student id number to \texttt{test\_out.csv}.
When writing the CSV file, use a ; as your delimiter.
Read the Python documentation for instructions on how to do this.
\end{problem}

\section*{\texttt{sys} Module}
The \li{sys} module allows us to access information specific to the system running Python.
For example, we often import \li{sys} to get the list \li{sys.argv} which is a list of arguments passed to the current environment.
It is sometimes useful to access these values when writing programs that are run from command line.
Many programs are written to execute differently based on the various arguments and options specified at execution.
Here is an example.
\begin{lstlisting}[title = echo.py]
def echo(word):
	print word
	
if __name__=='__main__':
	import sys
	echo(sys.argv[1])
\end{lstlisting}
When we run from the command line with a single argument, inside the main function, \li{sys.argv} returns the arguments thus letting the echo function print out the given argument.
If we execute the following script from the command line as follows:
\begin{verbatim}
python echo.py ACME
\end{verbatim}
We should get
\begin{verbatim}
ACME
\end{verbatim}



\begin{comment}
\section*{\texttt{pickle} Module}
The \li{pickle} module saves a Python object to a file.
It can also take the file and turn it back into a Python object.
Pickle can be used to store any built-in Python data type such as lists, tuple, integers, etc.
Not all Python objects can be pickled.
Here is an example of saving an object to a file.
\begin{lstlisting}
import pickle
a = range(10)
pickle.dump(a, open(`out.pkl', `w'))
\end{lstlisting}
When given an object, the \li{pickle} module outputs a small program that will rebuild the object later.
When unpickling a file, pickle executes this file in the interpreter.
Because of this, pickle is meant to be used as a temporary storage and data persistence mechanism.
It is not designed to be used for long term storage of data.
The pickle documentation includes this warning
\begin{quote}
\textbf{The pickle module is not intended to be secure against erroneous or maliciously constructed data.
Never unpickle data received from an untrusted or unauthenticated source.}
\end{quote}
To unpickle an object, use \li{pickle.load()}.
It accepts a file handle and returns the Python object it creates.

\begin{problem}
Create a list of numbers and strings.
Pickle the object to a file.
Inspect the contents of the file you created when pickling.
Unpickle the object.
\end{problem}

The \li{pickle} module only allows you to store one object per file.
If you want to store many objects, you should consider the \li{shelve} module.
This stores objects in a dictionary-like data structure with keys and values.
The values are pickled objects.
Since \li{shelve} relies on \li{pickle}, the same warning against untrusted sources applies to \li{shelve} as well.
\end{comment}

\section*{\texttt{timeit} Module}
This module is used to time the execution of small bits of Python code.
It is usually good to time lines of code in Python by using this module because it avoids a number of common pitfalls in measuring execution time.
IPython's \li{\%timeit} magic function is a wrapper around this module.

\begin{problem}
For most timing situations, we rely on IPython's \li{\%timeit} magic function.
One major drawback is that it only works in IPython.
The solution to this problem will be useful in other labs where you will be asked to time the execution of your coded solutions.
Write a function that will time the execution of another function.
You will need to use the \li{timeit} module.
Your function should accept as arguments, a function, $f$, and any arguments that should be passed to $f$.
Your function should return the minimum runtime.

Because of the way that Python's \li{timeit} module works, we must use a \emph{callable} function.
This essentially means we have to wrap the function we are timing and all of its arguments into a function object that can be called by \li{timeit}.
This is done by declaring a Python \li{lambda} function which takes no arguments.
\begin{lstlisting}
pfunc = lambda: f(*args, **kwargs)
\end{lstlisting}
where \li{args} is a tuple and \li{kwargs} is a dictionary.
This syntax is explained in chapter 4 of the Official Python Tutorial.
\end{problem}

\section*{\texttt{collections} Module}
This module defines several specialized data structures to use in addition to the built-in Python data structures. It includes namedTuple, deque, Counter, OrderedDict, and defaultDict.

Named tuples are designed to help improve code readability in some cases.
Standard tuples in Python are accessed by index while named tuples allow access via index or by a field name.

A deque is similar to a list, but instead of inserting and removing anywhere, it has very quick insert and removal from each end of the structure.
 
Counters are a subclass of dictionaries, but are expecially useful for counting things. They remain ordered with the item with the highest count at the beginning of the dictionary.
 
Ordered dictionaries are exactly like standard dictionaries except for one important difference.
Ordered dictionaries remember the order in which key-value pairs were added to the data structure.
This data structure is useful if we want to iterate over key-value pairs of a dictionary in a specific order.

Default dictionaries are a very convenient way to set a default value for all new keys in a dictionary.
While this can be done with standard dictionaries using the \li{setdefault()} method, using a default dictionary is simpler and faster.

\begin{problem}
Write a function that will read in a text file and then use a Counter to find the most common word in the file. Use it to find the most common word in \li{words.txt}.

\end{problem}


\begin{comment}
\begin{problem}
Write a function that takes in an integer maxInt and a integer n. Make a list of random integers between  0 and maxInt of length n, then use the Counter class to count how many there are of each one. Return a dictionary where the keys are the integers in the list and values are the frequency of the integer.
\end{problem}


 are named tuples and deques.

Named tuples are designed to help improve code readability in some cases.
Standard tuples in Python are accessed by index.
Named tuples allow access via index or by a field name.
Compare the following:
\begin{lstlisting}
from collections import namedtuple
pt = (32.1, 63.2)

Pt = namedtuple(`Point', `x y')
npt = Pt(32.1, 63.2)
\end{lstlisting}
The tuple \li{pt} is a standard tuple, which we can surmise represents the coordinates of a pt in 2D space.
We must assume that \li{pt[0]} is the x-coordinate and \li{pt[1]} is the y-coordinate.
When declaring a named tuple, we clearly defined what each index represents.
We know for certain that \li{npt.x} is the x-coordinate and that \li{npt.y} is the y-coordinate.


\begin{problem}
A double-ended queue, or deque, can be thought of as a deck of cards.
Inserting and removing elements from either end is very efficient.
Python's deque implementation only allows insertions at the left and right ends of the data structure.
This differs from a list, which allows insertions anywhere, but is very inefficient for all but right end insertions.
Write two functions that will rotate the elements of a deque and a list respectively.
To rotate, remove elements from the right end one by one and insert them on the left end.
Compare the timings you obtain from a deque and a list of 10000 elements.
\end{problem}
\end{comment}
