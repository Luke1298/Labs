\lab{The Standard Library}{Standard Library}
\objective{Become familiar with the Python standard library}
\label{lab:Standard Library}

\section*{Introduction}

Python is designed to make it easy to implement complex tasks with little code.
To that end, any python distribution includes several built-in functions for accomplishing common tasks.
In addition, Python is designed to import and reuse Python code written by others.
A Python file that can be imported is called a module.
%In other words, the Python programming language is designed to be extendable.
For example, it is possible to import a Python module that produces graphs and figures.
%Also included with any python distribution is a collection of modules for accomplishing a variety of tasks.
Any Python distribution includes a collection of modules for accomplishing a variety of tasks.
These modules are called the Python standard library.

In this lab, Python built-in functions and standard library are described and explored.
Having a solid foundation in this material will lead to more efficient coding in future labs.

\section*{Python Built-in Functions}

As mentioned, every Python installation will come with several built-in functions.
A listing of all Python built-in functions can be found \href{https://docs.python.org/2/library/functions.html}{here}.
Built-in functions may be used at any time and Python will recognize them.
Examples of some built-in functions that will be useful for future labs follow.

\begin{lstlisting}
# abs() returns the absolute value of a number
>>>abs(-5)
5

# min() returns the smallest element of a list or string
>>>min([1,2,3])
1
>>>min('xyz')
'x'

# max() returns the largest element of a list or string
>>>max([1,2,3])
3
>>>max('xyz')
'z'

# len() returns the length of a list or string
>>>len([1,2,3])
3
>>>len('abc')
3

# sum() returns the sum of a list
>>>sum([1,2,3])
6

# range(n) returns a list of numbers from 0 to n-1
>>>range(3)
[0,1,2]
\end{lstlisting}

\begin{problem}
Write a function that accepts a list of numbers as input and returns a list with the minimum, maximum, and average of the original list.
Remember to use floating point division to calculate the average.
\end{problem}

\section*{Names, Namespaces, and Modules}

\subsection*{Names and Namespaces}
Any object created in Python resides in memory.
These objects may be data, data structures, methods, or any other sort of Python object.
A Python name is a reference to a python object.

A namespace is a dictionary that maps names to python objects.
For example, in the following code block we create several names.

\begin{lstlisting}
# The number 13 is the object, x is the name.
>>>x = 13

# The number 4 is the object, number_of_students is the name.
>>> number_of_sudents = 4

# This is more complicated.  The set of python statements form an object.  
# The name is add_numbers.
>>> def add_numbers(a,b):
>>>   return a+b

# The list is the object, and students is the name.
>>> students = ["John", "Paul", "George", "Ringo"]
\end{lstlisting}

The equal sign assigns an object to a name.
In the previous code the equal sign assigned 13 to the name \li{x}.
%If the object on the right hand side of the equal sign is a name, then the name on the left hand side refers to the same object as the name on the right hand side.
If a name is assigned to another name, that new name refers to the same object that the orignal name refers to.
For example, if as above the name \li{students} is assigned a list of names, and then the name \li{band\_members} is assigned \li{students}, then both \li{band\_members} and \li{students} refer to the same object.

\begin{lstlisting}
>>> students = ["John", "Paul", "George", "Ringo"]
>>> band_members = students
#band_members now referes to the same object as students.
>>> band_members
['John', 'Paul', 'George', 'Ringo']
\end{lstlisting}

Python objects may be mutable or immutable.
An immutable object cannot be altered once created.
A mutable object's value may be changed.
For example, a list is mutable while a tuple is immutable.
\begin{lstlisting}
>>> example_tuple = (1,2)
>>> example_tuple[0] = 3
Traceback (most recent call last):
  File "<stdin>", line 1, in <module>
TypeError: 'tuple' object does not support item assignment

>>> example_list = [1,2]
>>> example_list[0] = 3
>>> example_list
[3,2]
\end{lstlisting}

If two names refer to the same mutable object, then any changes to the object is reflected in both names.
To avoid this, it is possible to create a copy of an object.
Changes made to the copy will not change the original object.
The way to make a copy of an object varies with object types.
To make a copy of a list, the \li{list()} built-in method may be used (See the following code block).

\begin{lstlisting}
>>> students = ["John", "Paul", "George", "Pete"]
>>> band_members = students

# Replace 'Pete' with 'Ringo' in band_members
>>> band_members[3] = "Ringo"
>>> band_members
"John", "Paul", "George", "Ringo"

# Since students and band_members refer to the same object, 
# 'Ringo' also replaces 'Pete' in students
>>>students
"John", "Paul", "George", "Ringo"

# Make a copy this time
>>> students = ["John", "Paul", "George", "Pete"]
>>> band_members = list(students)
>>> band_members[3] = "Ringo"
>>> band_members
'John', 'Paul', 'George', 'Ringo'
>>> students
'John', 'Paul', 'George', 'Pete'
\end{lstlisting}

\begin{problem}
Python has several methods that seem to change immutable objects.
These methods actually work by making copies of objects.
It is possible to determine which object types are mutable and immutable by using the equal sign and 'changing' the objects.
For example, if two names refer to the same list then changes to the entries of one name show up in the second.
\begin{lstlisting}
>>>list_1 = [1,2,3]
>>>list_2 = list_1
>>>list_1.append(4)
>>>list_1
[1,2,3,4]
>>>list_2
[1,2,3,4]
\end{lstlisting}

Performing a similar operation on a tuple yields a different result.
\begin{lstlisting}
>>>tuple1 = (1,2,3)
>>>tuple2 = tuple1
>>>tuple1 = tuple1 + (4,)
>>>tuple1
(1,2,3,4)
>>>tuple2
(1,2,3)
\end{lstlisting}

Using this technique, determine which python object types are mutable and which are immutable.
Recall that the python object types are numbers, strings, lists, tuples, and dictionaries.
Use the following operations to modify each type.
\begin{center}
\begin{tabular}{|c|c|}
\hline
numbers & num += 1 \\
\hline
strings & str1 += `a' \\
\hline
lists & list1.append(1) \\
\hline
tuples & tup1 += (1,) \\
\hline
dictionaries & dic1[1] = `a' \\
\hline
\end{tabular}
\end{center}
\end{problem}

\subsection*{Modules}

In general, a module is a file that contains Python statements.
The Python \li{import} operation loads the statements from a Python file.
When writing scripts and modules, all \li{import} statements occur at the top of the file.

The \li{import} statement first tries to find the python file specified.
Once found, every statement in the python file is executed.
Once imported, a module's names are available to the user.

For example, download the file "test\_module.py" and save it to a directory.
In a Python interpreter, navigate to a directory where the file is not saved.
If the interpreter then tries to import \li{test\_module}, it throws an error:

\begin{lstlisting}
>>> import test_module
Traceback (most recent call last):
  File "<stdin>", line 1, in <module>
ImportError: No module named test_module
\end{lstlisting}

If the interpreter is running a directory that contains the module no error is thrown and the module names become available:

\begin{lstlisting}
>>> import test_module
>>> test_module.add_numbers(1,1)
2
>>> test_module.print_students()
John
Paul
George
Ringo
>>> test_module.students
['John', 'Paul', 'George', 'Ringo']
\end{lstlisting}

If the module is changed after it is imported, then to access those changes the \li{reload} command is used.
The import command will not update the module.
Continuing the example, if a method called \li{subtract\_numbers} is added to \li{test\_module}, then \li{reload} is used.

\begin{lstlisting}
'''
Add the following to test_module.py and save it.

def subtract_numbers(a,b):
	return a - b
'''
>>> import test_module
>>> test_module.subtract_numbers(2,1)
Traceback (most recent call last):
  File "<stdin>", line 1, in <module>
AttributeError: 'module' object has no attribute 'subtract_numbers'

#Using reload instead of import
>>> reload(test_module)
>>> test_module.subtract_numbers(2,1)
1
\end{lstlisting}

\begin{problem}

Create a module called \li{calculator.py}.
Include methods that add and multiply two numbers and return the results.
Also include a method that returns the square root of a number (Hint:  Read the first paragraphs of the next section).
In the lab specification file, import the \li{calculator} module.
Using the functions defined in the module, write a new function that calculates the length of the hypotenuse of a triangle given the lengths of the other two side.
\end{problem}
%finish the module.  Use the addition method?

\section*{Python Standard Library}

Any Python distribution includes a set of modules that are available for use.
This set of modules is called the Python standard library.
To access a module in the standard library use the \li{import} command with the module name.

A summary of the documentation for these modules, called the docstring, can be read using the \li{\_\_doc\_\_} name.
For example, there is a \li{math} module in the standard library that includes common mathematical functions.
To read the docstring, do the following.
\begin{lstlisting}
>>> import math
>>> math.__doc__
'This module is always available.  It provides access to the\nmathematical functions defined by the C standard.'
\end{lstlisting}
Functions also have docstrings.
\begin{lstlisting}
>>> math.cos.__doc__
'cos(x)\n\nReturn the cosine of x (measured in radians).'
\end{lstlisting}

More extensive documentation is available using the \li{help} built-in function.
Many Python interpreters provide a similar function by using a question mark.
For example, in IPython, the question mark yields:

\begin{lstlisting}
In [0]: import math

In [1]: math?
Type:        module
String form: <module 'math' from '/Users/ACME/anaconda/lib/python2.7/lib-dynload/math.so'>
File:        /Users/ACME/anaconda/lib/python2.7/lib-dynload/math.so
Docstring:
This module is always available.  It provides access to the
mathematical functions defined by the C standard.
\end{lstlisting}

Many interpreters also provide shortcuts for exploring modules.
In IPython if the tab button is pressed after a module name followed by a period, the available functions are displayed.

\begin{lstlisting}
In [1]: import math

# typing math. and then pressing tab gives the available functions.
In [2]: math.
math.acos       math.atanh      math.e          math.factorial  
math.hypot      math.log10      math.sin        math.acosh      
math.ceil       math.erf        math.floor      math.isinf      
math.log1p      math.sinh       math.asin       math.copysign   
math.erfc       math.fmod       math.isnan      math.modf       
math.sqrt       math.asinh      math.cos        math.exp        
math.frexp      math.ldexp      math.pi         math.tan        
math.atan       math.cosh       math.expm1      math.fsum       
math.lgamma     math.pow        math.tanh       math.atan2      
math.degrees    math.fabs       math.gamma      math.log        
math.radians    math.trunc

In [2]: math.sqrt?
Type:        builtin_function_or_method
String form: <built-in function sqrt>
Docstring:
sqrt(x)

Return the square root of x.
\end{lstlisting}

\subsection*{The \li{sys} module}

The \li{sys} module includes methods for interacting with the python interpreter.
The module has a name \li{argv} that is a list of arguments passed to the interpreter when it runs Python scripts.
Consider the following script.

\begin{lstlisting}
# Save this as test_script.py
import sys

print(sys.argv)
\end{lstlisting}

If this script is run from the command line with additional arguments it will print them out.

\begin{lstlisting}
$ python test_script.py I am the walrus
['test_script', 'I', 'am', 'the', 'walrus']
\end{lstlisting}

This method can be used to point a python script to a file to be analyzed.
It can also be used to control a scripts behavior.
For example, a script's might have two methods and let the user decide which to use.
\begin{lstlisting}
import sys

arg = 'a thing'

def method1(arg):
	#Do something

def method2(arg):
	#Do something else

if sys.argv[1] == 1:
	method1(arg)
elif sys.argv[1] == 2:
	method2(arg)
\end{lstlisting}

\subsection*{The \li{time} module}

The \li{time} module includes functions for dealing with time.
In particular, functions in \li{time} access the computer's system clock.
This is useful for precisely measuring how long it takes for code to run.

The \li{time} module includes a function also called \li{time} that measures the number of seconds from a fixed starting point.
For most machines, this starting point will be January 1, 1970.

\begin{lstlisting}
>>> import time

# This will give the number of seconds since January 1, 1970 at the time of execution.
# Note that if this command will give a new time each time it is run.
>>> time.time()
1432229561.022016
\end{lstlisting}

In order to measure how long it takes to execute some Python code, a mesurement is taken right before and right after it is run.
Subtracting the first measurement from the second gives the amount of time in seconds that have passed.

\begin{lstlisting}
def time_for_loop():
	#This measures how long it takes a for loop to go through 10000 iterations using range.
	time1 = time.time()
	for i in range(10000):
		pass
	time2 = time.time()
	duration = time2 - time2
	print duration
\end{lstlisting}

The standard library also has a module called \li{timeit}.
This library is built to time python code.
In practice, \li{timeit} is the correct module for timing code.
In IPython, it can be used like a built-in function any time with the \li{\%timeit} command.

\begin{lstlisting}
# Put any code or function name after %timeit to time how long it takes to run.
# %timeit will run the code as many times as possible and return the best time.

#This example calculates how it takes to got through a for loop of length 100
In [0]: %timeit for i in xrange(100): pass 
100000 loops, best of 3: 2.88 µs per loop 
\end{lstlisting}

\begin{problem}

Download the files \li{matrix\_multiply.py} and \li{matrices.npz}.
The file \li{matrix\_multiply.py} is a module that has three methods for multiplying two matrices together, called \li{method1}, \li{method2}, and \li{method3}.
It also has a \li{load\_matrices} method that returns two matrices from \li{matrices.npz}
Write python commands that will do the following:
\begin{enumerate}
\item Import the \li{matrix\_multiply} module.
\item If no command line argument is given, print "No Input."  If anything other than \li{matrices.npz} is given, print "Incorrect Input."
\item If \li{matrices.npz} is given as a command line argument, use the \li{load\_matrices} function to load two matrices called \li{A} and \li{B}.
\item Finally, time how long each method takes to multiply the two matrices together.  Return a list of each duration.
\end{enumerate}
\end{problem}

%Introduce some modules

%Exercise that ties things together.


