\lab{Python}{Scrapy}{Scrapping}
\objective{Teach students how to scrape websites}

\section*{Scrapying Data}
Scrapping data from a website is a way to extract structured or unstructured data from the HTML of a page.

Most websites have APIs that direct people how to take data from the webpage. Often you use the API. You use a scrapying tool if you want to extract data from a website that does not have an API or the data you want is not a part of the API. (Like extracting e-mail addresses of users)

There is a nice python package called scrapy that is an application framework for scrapying. . The offical documentation can be found at \url{http://doc.scrapy.org/en/latest/}. 

Another package is BeautifulSoup the Documentation can be found at \url{http://www.crummy.com/software/BeautifulSoup/bs4/doc/}

\begin{info}
You will making and editing python files in this lab, so you will not be using ipython.
\end{info}

\begin{warn}
This Lab was written using version 0.22 of scrapy. If you are using a different version, the code will not be exactly the same.
\end{warn}

\section*{Installation}

To learn how to install Scrapy go to the website \url{http://doc.scrapy.org/en/latest/intro/install.html}

\section*{Setting up}
A good tutorial is found on scrapy's website
\url{http://doc.scrapy.org/en/latest/intro/tutorial.html}

\begin{problem}
Work thourgh the tutorial from the url above. You should turn in items.json as specified in the tutorial.
\end{problem}

When you run
\begin{lstlisting}
scrapy crawl dmoz -o items.json -t json
\end{lstlisting}
which generates a file \li{items.json} where all the scraped items are serialized in JSON. You can also store them in csv or xml by changing all the json to csv or xml.
For example
\begin{lstlisting}
scrapy crawl dmoz -o items.csv -t csv
\end{lstlisting}
generates a file \li{items.csv} where all the scraped items are stored in a csv file.

\begin{comment}
To begin a project go to the directory where you want to store the code and run
\begin{lstlisting}
scrapy startproject <project name>
\end{lstlisting}
This will create several files. \li{scrapy.cfg} is a configuration file. It will create folder called \li{<project name>} , that contains files \li{items.py}, \li{pipelines.py}, \li{settings.py} and a directory called \li{spiders}. We will use all this files.

\section*{Items}
Items are like python dictionaries with some addtional functionality. Go to the items.py found in the  \li{<project name>} directory. items are labels for the data that you be storing.

\begin{lstlisting}
from scrapy.item import Item, Field

class <project name>Item(Item):
    <item1> = Field()
    <item2> = Field()
    <item3> = Field()
\end{lstlisting}

\section*{Spiders}
Spiders are classes used to scrape data from a group of websites. You will define calsses that define the initail list of URLs to download, how to follow the links, and how to parse the contents of the pages into your items objects. There are two main spiders. \li{BaseSpider} and \li{CrawlSpider}. A BaseSpider is used for one website and a CrawlSpider used for crawling mulitple websites.

We will start with a BaseSpider. Go in your spider directory and create a new file (the name of the file does not matter). Subcalss BaseSpider as shown below. A BaseSpider has three things. A name that is a unique identifier, \li{start_urls} which is a list of URLS from which the spider will begin to crawl from and parse(), a method which will be called with the Response object at the beginning of each url.

the following is an example.
\begin{lstlisting}
from scrapy.spider import BaseSpider

class DmozSpider(BaseSpider):
    name = "example"
    allowed_domains = ["example.org"]
    start_urls = [
        "http://www.example.org/Computers/",
        "http://www.example.org/books/"
    ]

    def parse(self, response):
        hxs = HtmlXPathSelector(response)
        sites = hxs.select('//td')
        for site in sites:
            title = site.select('a/text()').extract()
            link = site.select('a/@href').extract()
            desc = site.select('text()').extract()
            print title, link, desc
\end{lstlisting}

Next we are going to explain what is in the parse method.

\section*{Parsing}
Scrapy uses a mechanism called selectors to extract data from web pages.
They have thhree methods.

the \li{select(path)} method returns a list of selectors, each of them representing nodes by the expression given as the arguement.This involves knowing a little HTML. HTML has elements whose name is enclosed in <>. So in the code above \li{select('//td')} selects  all the \li{<td>} elements. \li{select('//a/text()')} selects the text inside the \li{<a>} elements. the \li{//div[@class="mine"]} selects all the \li{<div>} elements with the attirbute \li{class="mine"}.  

The \li{extract()} methoed return a string with the data selected by the selector.

The \li{re(string)} returns a list of strings extracted by applying the regular espression given as the argument.

You can test this in a shell by typing in
\begin{lstlisting}
scrapy shell <URL of website>
\end{lstlisting}

try \li{scrapy shell http://www.dmoz.org/} and \li{hxs.select('//title')}. You will get feedback as if you were using the the \li{hxs.select()} in your code. You can try other methods to see what they do.

\section*{Storing the data}
To store the data we use our items. 

Say we defined our items like so
\begin{lstlisting}
from scrapy.item import Item, Field

from scrapy.item import Item, Field

class SportscrapeItem(Item):
    names = Field()
    numbers = Field()
\end{lstlisting}

And we definded our spider like so.
\begin{lstlisting}
from scrapy.spider import BaseSpider
from scrapy.selector import HtmlXPathSelector

from tutorial.items import DmozItem

class DmozSpider(BaseSpider):
   name = "dmoz"
   allowed_domains = ["dmoz.org"]
   start_urls = [
       "http://www.dmoz.org/Computers/Programming/Languages/Python/Books/",
       "http://www.dmoz.org/Computers/Programming/Languages/Python/Resources/"
   ]

   def parse(self, response):
       hxs = HtmlXPathSelector(response)
       sites = hxs.select('//ul/li')
       items = []
       for site in sites:
           item = DmozItem()
           item['title'] = site.select('a/text()').extract()
           item['link'] = site.select('a/@href').extract()
           item['desc'] = site.select('text()').extract()
           items.append(item)
       return items
\end{lstlisting}

Then I would go to the top directory and run
\begin{lstlisting}
scrapy crawl dmoz -o items.json -t json
\end{lstlisting}
which will generate a file \li{items.json} where all the scraped items are serialized in JSON. You can also store them in csv or xml by changing all the json to csv or xml.

\end{comment}

\section*{Crawling the Web}
This follows the section \url{http://doc.scrapy.org/en/latest/topics/spiders.html} subsection CrawlSpider.

The spider you wrote from the tutorial will only scrape one page. In order to scrape more pages you will need to inherit from \li{CrawlSpider}. The main difference is you set a list of rules for how you spider clicks on links and you do not overwrite the parse() method.

\begin{warn}
When using a \li{CrawlSpider} do not overwrite the \li{parse} method.
\end{warn}

For example if I were scrapying the ESPN website for basketball statistics (do not do because it is illegal to scrape data the ESPN website), this code would start of the mainpage and click are the links as specified in the rules. So below I made a\li{rule} list and for each rule I cread a \li{Rule} object. The first varible a takes in is a link extractor. the most common one is the li{SgmlLinkExtractor}. Two of its many attributes it has is \li{allow} and \li{deny}. \li{allow} takes in a list of strings and will click on those links has those words in the url. \li{deny} also takes in a list of strings and will not click on links with those words in the url. \li{deny} has higher precedence than \li{allow}. For a \li{Rule} object you set \li{callback} as a string of the method that parses the pages that you click on with the link extractor.  And than you set \li{follow=True} which means that it will click on those links.

The following code will go to the espn nba statistics website and click on the links that have \li{'statistics/team/_/stat'} in the name but not \li{'sort'} and scrape those webpages.

To do change
\begin{lstlisting}
from scrapy.contrib.spiders import CrawlSpider, Rule
from scrapy.contrib.linkextractors.sgml import SgmlLinkExtractor
from scrapy.selector import Selector

from sportscrape.items import SportscrapeItem

class EspnSpider(CrawlSpider):
   name = "espn"
   allowed_domains = ["espn.go.com"]
   start_urls = [
       "http://espn.go.com/nba/statistics"
   ]
   rules = (
        Rule(SgmlLinkExtractor(allow=('statistics/team/_/stat',),deny=('sort',)),callback='parse_page',follow=True),
    )

   def parse_page(self, response):
       sel = Selector(response)
       sites = sel.xpath('//td')
       items = []
       for site in sites:
           item = SportscrapeItem()
           item['numbers'] = site.xpath('text()').extract()
           item['names'] = site.xpath('a/text()').extract()
           items.append(item)
       return items
\end{lstlisting}

\section*{Pipelining}

The documentation can be found \url{http://doc.scrapy.org/en/latest/topics/item-pipeline.html}
Now you have a spider that crawls the web and dumps all the data into csv or json file. For some applications that will be enough. 

In addition you can cleanse the data or put it into a database via pipelining. In your directory \li{<project name>} you will have a file called \li{pipelines.py}. In this file you can create a class that has three methods \li{process_item(item, spider)}, \li{open_spider(spider)} and \li{close_spider(spider)}. The methods \li{open_spider} and \li{closed_spider} all called when the project is started and ended respectively. The \li{process_item} is called when the \li{parse} method of the spider returns. 

\subsection*{Cleansing}

The following code drops an item if it does not have anything in the data part of the object and returns it otherwise.

\begin{lstlisting}
from scrapy.exceptions import DropItem

class DataPipeline(object):

    def process_item(self, item, spider):
        if item['data']:
            return item
        else:
            raise DropItem("Missing data in %s" % item)
\end{lstlisting}

\subsection*{Adding to a database}
The following code puts the items where both its attributes are not null into a database. 
\begin{lstlisting}
import sqlite3 as sql
class SportscrapePipeline(object):
	
	def open_spider(self,spider):
		self.db = sql.connect("test1")
		self.cur = self.db.cursor()
		self.cur.execute('DROP TABLE IF EXISTS vals')
		self.cur.execute('CREATE TABLE vals (name TEXT, numbers TEXT);')
		self.statement = "INSERT INTO vals VALUES(?, ?);"
		return

	def close_spider(self, spider):
		self.db.commit() #save changes made in the transaction
		self.db.close()
		return

	def process_item(self, item, spider):
		#data=(item['names'][0],item['numbers'][0])
		if item['names'] and item['numbers']:
			data=(item['names'][0],item['numbers'][0])
			self.cur.execute(self.statement, data)
		elif item['names'] and  not item['numbers']:
			data=(item['names'][0],"null")
			self.cur.execute(self.statement, data)
		elif not item['names'] and item['numbers']:
			data=("null",item['numbers'][0])
			self.cur.execute(self.statement, data)
\end{lstlisting}

\subsection*{Activating your Pipeline}
Once you have written your pipeline code to activate it you go to your \li{settings.py} file and add the line
\begin{lstlisting}
ITEM_PIPELINES = '<directory>.pipelines.<name of class>': 300
\end{lstlisting}
where \li{<directory>} is the directory, \li{pipelines} stands for \li{pipelines.py} file, and \li{<name of class>} is what you named your pipeline class. In the example above it would be \li{SportscrapePipeline}. The \li{300} is a precedence operator that takes effect if you have multiple pipelines.

\begin{problem}
Write a webscraper that starts \url{http://math.byu.edu/peopleresearch/faculty/} clicks on links  in order to store each the professors name, email, office number, phone number (not fax number), and email address. If data is missing on the webpage put null. Store it in a csv file. Turn in the csv file. Call it \li{ProfessorData.csv}.

Hint: Use regular expressions.
\end{problem}

\begin{problem}
Create a pipeline to store the data from the last problem in a database. Turn in the database file. Call it \li{ProfessorData}.
\end{problem}

For this lab you turn in \li{items.json}, li{ProfessorData.csv}, li{ProfessorData} and your \{spider.py} file from the second and third problem into your google drive.