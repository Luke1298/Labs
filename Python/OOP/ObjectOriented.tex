\lab{Object-Oriented Programming}{Object Oriented Programming}
\label{lab:OOP}
\objective{
Python is a class-based language.
A \emph{class} is a blueprint for an object that binds together specified variables and routines.
Creating and using custom classes is often a good way to clean and speed up code.
In this lab we learn how to define and use Python classes.
In subsequents labs, we will create customized classes often for use in algorithms.}

% INTRODUCTION ----------------------------------------------------------------

\section*{Python Classes}

A Python \emph{class} is a code block that defines a custom object and determines its behavior.
The \li{class} command defines and names a new class.
Other statements follow, indented below the class name, to determine the behavior of objects instantiated by the class.

A class needs a method called a \emph{constructor} that is called whenever the class instantiates a new object.
The constructor specifies the initial state of the object.
In Python, a class's constructor is always named \li{__init__()}.
For example, the following code defines a class for storing information about backpacks.

\begin{lstlisting}
class Backpack(object):
    """A Backpack object class. Has a color and a list of contents.
    
    Attributes:
        color (str): the color of the backpack.
        contents (list): the contents of the backpack.
    """
    def __init__(self, color):          # This function is the constructor.
        """Set the color and initialize an empty contents list.
        
        Inputs:
            color (str): the color of the backpack.
        
        Returns:
            A Backpack object wth no contents.
        """ 
        self.color = color              # Initialize some attributes.
        self.contents = []
\end{lstlisting}

This \li{Backpack} class has two \emph{attributes}: \li{color} and \li{contents}.
Attributes are variables stored within the class object.
In the body of the class definition, attributes are accessed via the name \li{self}.
This name refers to the object internally once it has been created.

\subsection*{Instantiation}

The \li{class} code block above only defines a blueprint for backpack objects.
To create a backpack object, we ``call'' the class like a function.
An object created by a class is called an \emph{instance} of the class.
It is also said that a class \emph{instantiates} objects.

Classes may be imported in the same way as modules.
In the following code, we import the \li{Backpack} class and instantiate a \li{Backpack} object.

\begin{lstlisting}
# Import the Backpack class and instantiate an object called 'my_backpack'.
>>> from Packs import Backpack
>>> my_backpack = Backpack("blue")

# Access the attributes of the object with a period.
>>> my_backpack.color
<<'blue'>>
>>> my_backpack.contents
[]

# The object's attributes can be modified dynamically.
>>> my_backpack.color = "green"
>>> print(my_backpack.color)
green
\end{lstlisting}

\begin{info}
Many programming languages distinguish between \emph{public} and \emph{private} attributes.
In Python, all attributes are automatically public.
However, an attribute can be hidden from the user in IPython by starting the name with an underscore. %Example?
\end{info}

\subsection*{Methods}

In addition to storing variables as attributes, classes can have functions attached to them.
A function that belong to a specific class is called a \emph{method}.
In the following code, we define two simple methods in the \li{Backpack} class.

\begin{lstlisting}
class Backpack(object):
    # ...
    def put(self, item):
        """Add 'item' to the backpack's list of contents."""
        self.contents.append(item)
    
    def take(self, item):
        """Remove 'item' from the backpack's list of contents."""
        self.contents.remove(item)
\end{lstlisting}

The first argument of each method must be \li{self}, to give the method access to the attributes and other methods of the class.
The \li{self} argument is only included in the declaration of the class methods, \textbf{not} when calling the methods.

\begin{lstlisting}
# Add some items to the backpack object.
>>> my_backpack.put("notebook")
>>> my_backpack.put("pencils")
>>> my_backpack.contents
<<['notebook', 'pencils']>>

# Remove an item from the backpack.
>>> my_backpack.take("pencils")
>>> my_backpack.contents
<<['notebook']>>
\end{lstlisting}

\begin{comment}
IPython's object introspection feature reveals details on the object.

\begin{lstlisting}
In [1]: import Packs

In [2]: b = Packs.Backpack("green")

In [3]: b.  # press 'tab'
b.color     b.contents  b.put       b.take      

In [4]: b.put?
<<Signature: b.put(item)
Docstring: Add 'item' to the backpack's content list.
File:      ~/Downloads/Packs.py
Type:      instancemethod>>
\end{lstlisting}
\end{comment}

% Problem 1: Expand the Backpack class.
% Should we specify what the attributes are called?
\begin{problem}
Expand the \li{Backpack} class to match the following specifications.
\begin{enumerate}
\item Modify the constructor so that it accepts a color, a name, and a maximum size, in that order.
Make \li{max_size} a default argument with default value $5$.
Store each input as a class attribute.
%the new attributes as \li{self.name} and \li{self.max_size}.

\item Modify the \li{put()} method to ensure that the backpack does not go over capacity.
If the user tries to put in more than \li{max\_size} items, print ``I'm Full!'' and do not add the item to the contents list.

\item Add a new method to the backpack called \li{dump()} that empties the contents of the backpack.
This method should receive no arguments.

(Hint: this method can be implemented in a single line.)
\end{enumerate}

To ensure that your class works properly, consider writing a test function outside outside of the \li{Backpack} class that instantiates and analyzes a \li{Backpack} object.
Your function may look something like this:
\begin{lstlisting}
def test_backpack():
    backpack_1 = Backpack("black", "Barry")     # Instantiate the object.
    if backpack_1.max_size != 5:                # Test an attribute.
        print("Wrong default max_size!")
    for item in ["pencil", "pen", "paper", "computer"]:
        backpack_1.put(item)                    # Test a method.
    print(backpack.contents)
\end{lstlisting}
\end{problem}

% INHERITANCE -----------------------------------------------------------------

\section*{Inheritance}

\emph{Inheritance} is an object-oriented programming tool for code reuse and organization.
To create a new class that is similar to one that already exists, it is often better to ``inherit'' the already existing methods and attributes rather than create a new class from scratch.
This is done by including the existing class as an argument in the class definition (where the word \li{object} is in the definition of the \li{Backpack} class).
This creates a \emph{class hierarchy}: a class that inherits from another class is called a \emph{subclass}, and the class that a subclass inherits from is called a \emph{superclass}.

For example, since a knapsack is a backpack (but not all backpacks are knapsacks), we create a special \li{Knapsack} subclass that inherits the structure and behaviors of the \li{Backpack} class, and adds some extra functionality.

\begin{lstlisting}
# Inherit from the Backpack class in the class definition.
class Knapsack(Backpack):
    """A Knapsack object class. Inherits from the Backpack class.
    A knapsack is smaller than a backpack and can be tied closed.
    
    Attributes:
        color (str): the color of the knapsack.
        name (str): the name of the knapsack.
        max_size (int): the maximum number of items that can fit
            in the knapsack.
        contents (list): the contents of the backpack.
        closed (bool): whether or not the knapsack is tied shut.
    """
    def __init__(self, color, name, max_size=3):
        """Use the Backpack constructor to initialize the name and
        max_size attributes. A knapsack only holds 3 item by default
        instead of 5.

        Inputs:
            color (str): the color of the knapsack.
            name (str): the name of the knapsack.
            max_size (int, opt): the maximum number of items that can be
                stored in the knapsack. Defaults to 3.
        
        Returns:
            A knapsack object with no contents.
        """
        Backpack.__init__(self, color, name, max_size)
        self.closed = True
\end{lstlisting}

A subclass may have new attributes and methods that are unavailable to the superclass, such as \li{self.closed} in the \li{Knapsack} class.
If methods in the new class need to be changed, they are overwritten as is the case of the constructor in the \li{Knapsack} class.
New methods can be included normally.
As an example, we modify the \li{put()} and \li{take()} methods in \li{Knapsack} to check if the knapsack is shut.

\begin{lstlisting}
class Knapsack(Backpack):
    # ...
    def put(self, item):
        """If the knapsack is untied, use the Backpack.put() method."""
        if self.closed:
            print "I'm closed!"
        else:
            Backpack.put(self, item)
    
    def take(self, item):
        """If the knapsack is untied, use the Backpack.take() method."""
        if self.closed:
            print "I'm closed!"
        else:
            Backpack.take(self, item)
\end{lstlisting}

Since \li{Knapsack} inherits from \li{Backpack}, a knapsack object is a backpack object.
All methods defined in the \li{Backpack} class are available to instances of the \li{Knapsack} class.
Thus, in this example, the \li{dump()} method is available even though it is not defined explicitly in the \li{Knapsack} class.

\begin{lstlisting}
>>> from Packs import Knapsack
>>> my_knapsack = Knapsack("brown", "Brady")
>>> isinstance(my_knapsack, Backpack)       # A Knapsack is a Backpack.
True

# The put and take method now require the knapsack to be open.
>>> my_knapsack.put('compass')
<<I'm closed!>>

# Open the knapsack and put in some items.
>>> my_knapsack.closed = False
>>> my_knapsack.put("compass")
>>> my_knapsack.put("pocket knife")
>>> my_knapsack.contents
<<['compass', 'pocket knife']>>

# The dump method is inherited from the Backpack class, and
# can be used even though it is not defined in the Knapsack class.
>>> my_knapsack.dump()
>>> my_knapsack.contents
[]
\end{lstlisting}

% Problem 2: create an inheritance class.
\begin{problem}
Create a \li{Jetpack} class that inherits from the \li{Backpack} class.
\begin{enumerate}
\item Overwrite the constructor so that in addition to a color, name, and maximum size, it also accepts an amount of fuel.
Store the fuel as a class attribute.
Also change the default value of \li{max_size} to $2$, and set the default value of fuel to $10$.
%Add an optional argument for fuel with a default value of $10$ to keep track of how much fuel is left in the jetpack.

\item Add a \li{fly()} method that accepts an amount of fuel to be burned and decrements the fuel attribute by that amount.
If the user tries to burn more fuel than remains, print ``Not enough fuel!" and do not decrement the fuel.

\item Overwrite the \li{dump()} method so that both the contents and the fuel tank are emptied.
\end{enumerate}
\end{problem}

% MAGIC METHODS ---------------------------------------------------------------

\section*{Magic Methods}

In Python, a \emph{magic method} is a special method used to make an object behave like a built-in data type.\footnote{A complete list of magic methods can be found at \url{https://docs.python.org/2/reference/datamodel.html}}%#special-method-names}.}
Magic methods begins and ends with two underscores, like \li{__init__()}.
Every Python object is automatically endowed with several magic methods, but they are hidden because they begin with an underscore (this is a way of hiding attributes or methods from the user; for example, try hiding the \li{closed} attribute in the \li{Knapsack} class by changing it to \li{_closed}).

We can see details on the \li{Backpack} object, including its magic methods, using object instrospection:

\begin{lstlisting}
In [1]: import Packs

In [2]: b = Packs.Backpack("green", "Oscar")

In [3]: b.      # Press 'tab' to see standard methods and attributes.
b.color     b.contents  b.put       b.take

In [4]: b.put?
<<Signature: b.put(item)
Docstring: Add 'item' to the backpack's content list.
File:      ~/Downloads/Packs.py
Type:      instancemethod>>

In [5]: b.__	# Press 'tab' to see magic methods.
b.__add__           b.__getattribute__  b.__reduce_ex__
b.__class__         b.__hash__          b.__repr__
b.__delattr__       b.__init__          b.__setattr__
b.__dict__          b.__lt__            b.__sizeof__
b.__doc__           b.__module__        b.__str__
b.__eq__            b.__new__           b.__subclasshook__
b.__format__        b.__reduce__        b.__weakref__

In [6]: b?
<<Type:           Backpack
File:           ~/Downloads/Packs.py
Docstring:
A Backpack object class. Has a color and a list of contents.

Attributes:
    color (str): the color of the backpack.
    contents (list): the contents of the backpack.
Init docstring:
Set the color and initialize an empty contents list.

Inputs:
    color (str): the color of the backpack.

Returns:
    A backpack object wth no contents.>>
\end{lstlisting}

\begin{info}
In all of the preceding examples, the comments enclosed by sets of three double quotes are the object's \emph{docstring}, stored as the \li{__doc__} attribute.
A good docstring typically includes a summary of the class or function, information about the inputs and returns, and other notes.
Writing detailed docstrings is critical so that others can utilize your code correctly.
\end{info}

Now, suppose we wanted to add two backpacks together.
How should addition be defined for backpacks?
A simple option is to add the number of contents.
Then if backpack $A$ has $3$ items and backpack $B$ has $5$ items, $A + B$ should return $8$.
%Magic methods can also potentially alter the objects themselves.
%In the following example, addition copies over the contents from the backpack on the right-hand side of the $+$ operator into the contents of the backpack on the left-hand side of the $+$.

\begin{lstlisting}
class Backpack(object):
    # ...
    def __add__(self, other):
        """Add the number of contents of each Backpack."""
        return len(self.contents) + len(other.contents)
\end{lstlisting}

Using the $+$ binary operator on two \li{Backpack} objects calls the \li{__add__()} method.
The object on the left side of the $+$ is passed in as \li{self} and the object on the right side of the $+$ is passed in as \li{other}.

\begin{lstlisting}
>>> from Packs import Backpack
>>> backpack1 = Backpack("red", "Rose")
>>> backpack2 = Backpack("cyan", "Carly")

# Put some items in the backpacks.
>>> backpack1.put("textbook")
>>> backpack2.put("water bottle")
>>> backpack2.put("snacks")

# Now add the backpacks like numbers.
>>> backpack1 + backpack2
3
\end{lstlisting}

Subtraction, multiplication, and division may be similary defined with the magic methods \li{__sub__()}, \li{__mul__()}, and \li{__div__()}.
What each of these operations do, or should do, is up to the programmer and should be carefully documented.

\subsection*{Comparisons}

% DO NOT USE __cmp__ ! cmp has been removed as of Python 3.0.
Magic methods also allow for object comparisons.
The \li{__lt__()} and \li{__gt__()} methods correspond to $<$ and $>$, respectively.
Suppose we decide that one backpack should be considered ``less'' than another if it has fewer items in contents.

\begin{lstlisting}
class Backpack(object)
    # ...
    def __lt__(self, other):
        """Compare two backpacks. If 'self' has fewer contents
        than 'other', return True. Otherwise, return False.
        """
        return len(self.contents) < len(other.contents)
\end{lstlisting}

% Since $<$ is a logical operator, \li{__lt__()} should return a boolean value.
% The comparison in the above line of code evaluates to \li{True} or \li{False}, so a boolean value will be returned when $<$ is used with backpacks.

Since $<$ is a boolean binary operator, the \li{__lt__()} method should returns either \li{True} or \li{False}, as should other comparison operators.

Now using the $<$ binary operator on two \li{Backpack} objects calls the \li{__lt__()} function.
Again, the object on the left side of the operator is passed in as \li{self}, and the object on the right side is passed in as \li{other}.

Other comparison operators have corresponding magic methods as well.
The \li{__le__()}, \li{__ge__()}, \li{__eq__()}, and \li{__ne__()} methods correspond to $<=$, $>=$, $==$, and $!=$, respectively.

\begin{lstlisting}
>>> from Packs import Backpack
>>> backpack1 = Backpack("magenta", "Maggy")
>>> backpack2 = Backpack("yellow", "Lemony")

>>> backpack1.put('book')
>>> backpack2.put('water bottle')
>>> backpack1 < backpack2
False

>>> backpack2.put('pencils')
>>> backpack1 < backpack2
True
\end{lstlisting}

% Problem 3: __str__ and __eq__(?) magic methods
\begin{problem}
Endow the \li{Backpack} class with two additional magic methods:
\begin{enumerate}
\item The \li{__eq__()} magic method is used to determine if two objects are equal, and is invoked by the \li{==} operator.
Implement the \li{__eq__()} magic method for the \li{Backpack} class so that two \li{Backpack} objects are equal if and only if they have the same name, color, and contents.
The two contents lists do not have to have their items in the same order to be considered equal.

% TODO: for Python 3, call the 'print statement' the 'print function'.
\item The \li{__str__()} magic method is used to produce the string representation of an object.
This method is invoked when an object is cast as a string with the \li{str} function, or when using the \li{print} statement.
Implement the \li{__str__()} method in the \li{Backpack} class so that printing a \li{Backpack} object yields the following output:
\begin{lstlisting}
<<Name:		<name>
Color:		<color>
Size:       <number of items in contents>
Max Size:   <max_size>
Contents:   [<item1>, <item2>, ...]>>
\end{lstlisting}
(Hint: Use the `\textbackslash{t}' tab and `\textbackslash{n}' newline characters to help align output.)
\end{enumerate}
\end{problem}

\begin{warn}
Comparison operators are not automatically related.
For exampele, for two backpacks $A$ and $B$, if $A==B$ is \li{True}, it does not automatically imply that $A!=B$ is \li{False}.
Accordingly, when defining \li{__eq__()}, one should also define \li{__ne__()} so that the operators will behave as expected.
\end{warn}

% Problem 4: ComplexNumber class
\begin{problem}
Create your own \li{ComplexNumber} class that supports the basic operations of complex numbers.
\begin{enumerate}
\item Complex numbers have a real and an imaginary part. The constructor should therefore accept two numbers. Store the first as \li{self.real} and the second as \li{self.imag}.
\item Implement a \li{conjugate} method that returns the object's complex conjugate (as a new \li{ComplexNumber} object). Recall that $\overline{a + bi} = a - bi$.
\item Add the following magic methods:
\begin{enumerate}
\item The \li{__abs__()} magic method determines the output of the built-in \li{abs} function (absolute value). Implement \li{__abs__()} so that it returns the magnitude of the complex number. Recall that $|a+bi| = \sqrt{a^2+b^2}$.
\item Implement \li{__lt__()}, \li{__gt__()}, and \li{__eq__()} so that \li{ComplexNumber} objects can be compared by their magnitudes. That is, $(a+bi) < (c+di)$ if and only if $|a+bi| < |c+di|$, and so on.
\item Implement \li{__add__()}, \li{__sub__()}, \li{__mul__()}, and \li{__div__()} appropriately.
Each of these should return a new \li{ComplexNumber} object.

(Hint: use the complex conjugate to implement division).
\end{enumerate}
\end{enumerate}
Compare your class to Python's built-in \li{complex} class.
How can your class be improved to act more like complex numbers?
\end{problem}

% TODO: It might be nice to add in a few short optional sections.
\begin{comment} % OPTIONAL vvvvvvvvvvvvvvvvvvvvvvvvvvvvvvvvvvvvvvvvvvvvvvvvvvvv
\section*{Advanced Topics (Optional)}

\subsection*{Static Attributes}

Static attributes are shared between all instances of the class.
To make an attribute static, declare it inside of the \li{class} block but outside of any of the class's functions.
Do not use \li{self}.
The attribute is accessed with the class name instead of with \li{self}.

\subsection*{Static Methods}

Individual methods can also be static.
A static method can have no dependence on individual instances, so there can be no references to \li{self} inside the body of the method.
Accordingly, \li{self} is \textbf{not} listed as an argument like it usually is.
In addition, static methods need the tag \li{@staticmethod} above them.

% Add a static method / attribute to the Backpack class.
\begin{problem}
Add a static attribute to the \li{Backpack} class to serve as a counter for an ID.
In the constructor for the \li{Backpack} class, add an instance variable called \li{self.ID}.
Set this ID based on the static ID variable, then increment the static ID so that the next \li{Backpack} object will have a unique ID.
\end{problem}

\subsection*{Other Magic Methods}

\begin{enumerate}
\item The \li{__hash__()} magic method brands an object with a (hopefully) unique ID for quick comparison and indexing.
Hash functions are important in many fast data structures and are discussed in Chapter 3 of the Volume II text.
\item The \li{__getitem__()} magic method tells an object how to be indexed by brackets, including slicing behavior.

% Implement hash and getitem in the Backpack class.
\begin{problem}
Implement both of the aforementioned methods for the \li{Backpack} class.
\end{problem}

\end{comment} % OPTIONAL ^^^^^^^^^^^^^^^^^^^^^^^^^^^^^^^^^^^^^^^^^^^^^^^^^^^^^^
