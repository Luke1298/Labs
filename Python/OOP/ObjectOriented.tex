\lab{Object-Oriented Programming}{Object Oriented Programming}
\label{lab:OOP}
\objective{
Python is a class-based language.
A class is a blueprint for an object, defining attributes and methods.
In this lab we learn how to define Python classes to create customized objects for use in algorithms.}

\section*{Python Classes}

A Python class is code that defines a custom object and determines its behavior.
A class is defined with the \li{class} command to name the class.
Other statements follow, indented below the class name, to determine the behavior of objects instantiated by the class.

A class needs a method called a constructor that is called when the class instantiates a new object.
The constructor specifies the initial state of the object.
In Python, a class's constructor is always named \li{__init__}.
For example, the following is a class that created backpack objects.

\begin{lstlisting}
# Backpack.py

class Backpack(object):
    """A Backpack object class. Has a color and a list of contents.
    
    Attributes:
        color (str): the color of the backpack.
        contents (list): the contents of the backpack.
    """
    
    def __init__(self, color = 'black'):
        """Constructor for a backpack object.
		Set the color and initialize the contents list.
        
        Inputs:
            color (str, opt): the color of the backpack. Defaults to 'black'.
        
        Returns:
            A backpack object wth no contents.
        """
        
        # Assign the backpack a color.
        self.color = color
        
        # Create a list to store the contents of the backpack.
        self.contents = []
    
    def put(self, item):
        """Add 'item' to the backpack's content list."""
        self.contents.append(item)
    
    def take(self, item):
        """Remove 'item' from the backpack's content list."""
        self.contents.remove(item)
\end{lstlisting}

Note that the first argument of each method is called \li{self}.
This name refers to the object internally once it has been created.
The \li{self} argument is only included in the declaration of the class methods, but not when calling the methods.
In the above example, \li{self.color} refers to one of the backpack's internal attributes.
The argument \li{color} is an input for the constructor, and holds the value that is then stored in the backpack as \li{self.color}.

Classes may be imported in the same way as modules.
If the above module is saved as \li{Backpack.py}, it is possible to import the \li{Backpack} class from it.
Once imported, the \li{Backpack} class can be called to construct backpack objects.

Calling a class constructs an object.
Once constructed, the user has access to an object's methods and attributes.
An object created by a class is called an \emph{instance} of the class.
It is also said that a class \emph{instantiates} objects.
\begin{lstlisting}
# From the file Backpack.py import the Backpack class. Then create a backpack.
>>> from Backpack import Backpack
>>> my_backpack = Backpack('green')

# Add an item to the backpack.
>>> my_backpack.put('notebook')
>>> my_backpack.put('pencils')

# Access attributes of the object.
>>> my_backpack.color
'green'

>>> my_backpack.contents
['notebook', 'pencils']

# Remove an item from the backpack.
>>> my_backpack.take('pencils')
>>> my_backpack.contents
['notebook']
\end{lstlisting}

% Problem 1: expand the Backpack class.
\begin{problem}
Expand the \li{Backpack} class.
Add attributes called \li{name} and \li{max_size}.
The \li{name} attribute is a label for the backpack, and the \li{max_size} attribute is the total capacity of the backpack.
Modify the constructor method to accept a \li{name} and a \li{max_size} in addition to \li{color}.
Set the default values of \li{name} to `backpack' and \li{max_size} to 5.

Once the new attributes are set, modify the \li{put} method to ensure that the backpack does not go over capacity.
If the user tries to put in more than \li{max\_size} items, print ``Backpack Full.''

Finally, add a new method to the backpack called \li{dump} that empties the contents of the backpack.
\end{problem}

\section*{Inheritance}

In object-oriented programming, inheritance is a tool for code reuse and organization.
To create a new class similar to one that already exists, it is sometimes easier to ``inherit'' the already existing methods.
This is done by including the existing class as an argument in the class definition.
This creates a \emph{class heierarchy}.
For example, since a knapsack is a backpack (but not all backpacks are knapsacks), we can create a special \li{Knapsack} class that automatically inherits the attributes and methods of the \li{Backpack} class, and adds extra functionality.
A class that inherits from another class is called a \emph{subclass}.
The class that a subclass inherits from is called a \emph{superclass}.
We define a \li{Knapsack} subclass that inherits from the \li{Backpack} superclass using the following code:

\begin{lstlisting}
# Backpack.py

# Inherit the Backpack class in the Knapsack definition
class Knapsack(Backpack):
    """A Knapsack object class. Inherits from the Backpack class.
    A knapsack is smaller than a backpack and can be tied closed.
    
    Attributes:
        color (str): the color of the knapsack.
        name (str): the name of the knapsack.
        max_size (int): the maximum number of items that can fit
            in the knapsack.
        contents (list): the contents of the backpack.
        closed (bool): whether or not the knapsack is tied shut.
    """
    
    def __init__(self, color='brown', name='knapsack', max_size=3):
        """Constructor for a knapsack object. A knapsack only holds 3 item by
        default instead of 5. Use the Backpack constructor to initialize the
        name and max_size attributes.
        
        Inputs:
            color (str, opt): the color of the knapsack. Defaults to 'brown'.
            name (str, opt): the name of the knapsack. Defaults to 'knapsack'.
            max_size (int, opt): the maximum number of items that can be
                stored in the knapsack. Defaults to 3.
        
        Returns:
            A knapsack object with no contents.
        """
        
        Backpack.__init__(self, color, name, max_size)
        self.closed = True
\end{lstlisting}

A subclass may have new attributes and methods that are unavailable to the superclass.
If methods in the new class need to be changed, they are overwritten as is the case of the constructor in the \li{Knapsack} class.
New methods are included normally.
For example, the \li{put} and \li{take} methods in \li{knapsack} are modified to check if the knapsack is shut. 
\li{tie} and \li{untie} methods are added as well.

\begin{lstlisting}
# Knapsack class
    def put(self, item):
        """If the knapsack is untied, use the Backpack put() method."""
        if self.closed:
            print "Knapsack closed!"
        else:
            Backpack.put(self, item)
    
    def take(self, item):
        """If the knapsack is untied, use the Backpack take() method."""
        if self.closed:
            print "Knapsack closed!"
        else:
            Backpack.take(self, item)
    
    def untie(self):
        """Untie the knapsack."""
        self.closed = False
    
    def tie(self):
        """Tie the knapsack."""
        self.closed = True
\end{lstlisting}

Since \li{Knapsack} inherits from \li{Backpack}, a knapsack object is a backpack object.
All methods defined in the \li{Backpack} class are available to instances of the \li{Knapsack} class.
Thus, in this example, the \li{dump} method is available even though it is not defined explicitly in the \li{knapsack} class.

\begin{lstlisting}
# Import the Knapsack class from Backpay.py
>>> from Backpack import Knapsack
>>> my_knapsack = Knapsack()

# The put and take method now require the knapsack to be open.
>>> my_knapsack.put('compass')
Knapsack closed!

# Open the knapsack and put in some items.
>>> my_knapsack.untie()
>>> my_knapsack.put('compass')
>>> my_knapsack.put('pocket knife')
>>> my_knapsack.contents
['compass', 'pocket knife']

# The dump method is inherited from the backpack class, and
# can be used even though it is not defined in the knapsack class.
>>> my_knapsack.dump()
>>> my_knapsack.contents
[]
\end{lstlisting}

% Problem 2: create an inheritance class.
\begin{problem}
Create a \li{Jetpack} class that inherits from the \li{Backpack} class.
Overwrite the constructor so that the \li{color} attribute defaults to `silver', the \li{name} defaults to `jetpack', and the \li{max_size} defaults to 2.
Also in the constructor initialize a \li{fuel} attribute that keeps track of how much fuel is left in the jetpack.
Set the default amount of \li{fuel} to 10.

Add a \li{fly} method that accepts an amount of fuel to be burned, and decrement \li{fuel} by that amount.
If the user tries to burn more fuel than remains, print ``Not enough fuel!"
Finally, overload the \li{dump} method so that both the contents and the fuel tank are emptied.
\end{problem}

\section*{Magic Methods}

Python \emph{magic methods} can be used to make objects behave like built-in data types.
A magic method begins and ends with two underscores, like \li{__init__}.
Every Python object is automatically endowed with several magic methods, but they are hidden because they begin with an underscore (this is a way of hiding attributes or methods from the user; for example, try hiding the \li{closed} attribute in the \li{Knapsack} class by changing it to \li{_closed}).
A complete list of magic methods can be found \href{https://docs.python.org/2/reference/datamodel.html#special-method-names}{here}.

You can see all of the available magic methods for a \li{Backpack} object in IPython with the following code:

\begin{lstlisting}
In [1]: from Backpack import Backpack
In [2]: b = Backpack()
In [3]: b.__	# Press 'tab'
b.__add__     b.__eq__      b.__lt__      b.__str__     
b.__doc__     b.__init__    b.__module__  
\end{lstlisting}

\begin{info}
In all of the preceding examples, the comments enclosed by sets of three double quotes are the object's \emph{docstring}, stored as the \li{__doc__} attribute.
It is important to write good docstrings so that others can utilize your code correctly.
\end{info}

Now, suppose we wanted to add two backpacks together.
How should addition be defined for backpacks?
A simple option is to add the number of contents.
Then if backpack $A$ has $3$ items and backpack $B$ has $5$ items, $A + B$ return $8$.
Magic methods can also potentially alter the objects themselves.
In the following example, addition copies over the contents from the backpack on the right-hand side of the $+$ operator into the contents of the backpack on the left-hand side of the $+$.

\begin{lstlisting}
# Backpack class
    def __add__(self, other):
        """Add the contents of 'other' to the contents of 'self'.
        Note that the contents of 'other' are unchanged.
        
        Inputs:
            self (Backpack): the backpack on the left-hand side
                of the '+' addition operator.
            other (Backpack): The backpack on the right-hand side
                of the '+' addition operator.
        """
        self.contents = self.contents + other.contents
\end{lstlisting}

To demonstrate the addition method, create two instances of the \li{Backpack} class and add them together.

\begin{lstlisting}
>>> from Backpack import Backpack
>>> backpack1 = Backpack()
>>> backpack2 = Backpack()

# Put some items in the backpacks
>>> backpack1.put('textbook')
>>> backpack2.put('water bottle')

# Now add the backpacks like numbers
>>> backpack1 + backpack2
>>> backpack1.contents
['textbook', 'water bottle']
>>> backpack2.contents
['water bottle']
\end{lstlisting}

Subtraction, division, or multiplication may be similary defined with the magic methods \li{__sub__}, \li{__div__}, and \li{__mul__}.
What each of these operations do, or should do, is up to the programmer and should be carefully documented.

Magic methods also allow for comparisons.
The methods \li{__lt__}, \li{__le__}, \li{__gt__}, and \li{__ge__} methods correspond to $<$, $<=$, $>$, and $>=$ respectively.
For example, one backpack might be less than another if it has fewer items in contents.

\begin{lstlisting}
# Backpack class
    def __lt__(self, other):
        """Compare two backpacks. If 'self' has fewer contents than 'other',
        return True. Otherwise, return False.
        
        Inputs:
            self (Backpack): the backpack on the left-hand side
                of the '<' comparison operator.
            other (Backpack): The backpack on the right-hand side
                of the '<' comparison operator.
        """
        return len(self.contents) < len(other.contents)
\end{lstlisting}

Since $<$ is a logical operator, it makes sense for \li{__lt__} to return a boolean value.
The comparison in the above line of code evaluates to \li{True} or \li{False}, so a boolean value will be returned when $<$ is used with backpacks.

Again to test this, create two backpacks and use the $<$ operator.
\begin{lstlisting}
>>> from Backpack import Backpack
>>> backpack1 = Backpack()
>>> backpack2 = Backpack()

>>> backpack1.put('book')
>>> backpack2.put('water bottle')
>>> backpack1 < backpack2
False

>>> backpack2.put('pencils')
>>> backpack1 < backpack2
True
\end{lstlisting}

\begin{info}
The \li{__hash__} magic method brands an object with a nearly unique ID for quick comparison and indexing.
Hash functions are important in many fast data structures and will be discussed in Chapter 3 of the Volume II text.
\end{info}

% Problem 3: __str__ and __eq__ magic methods
\begin{problem}
Endow the \li{Backpack} class with two additional magic methods:
\begin{enumerate}
\item The \li{__str__} magic method is used to produce the string representation of an object.
When an object is used as an argument for the \li{print} function, the object's \li{__str__} method is called and the results are displayed.
Implement the \li{__str__} method in the \li{Backpack} class so that printing a \li{Backpack} object yields the following output:
\begin{lstlisting}
 Name:		<name>
 Color:		<color>
 Size:		<number of items in contents>
 Max Size:	<max_size>
 Contents:
 			<item1>
 			<item2> ...
\end{lstlisting}
If the backpack is empty, the contents line should read:
\begin{lstlisting}
Contents:	Empty
\end{lstlisting}
(Hint: Use the `\textbackslash{t}' tab character to help align output.)

\item The \li{__eq__} magic method is used to determine if two objects are equal. Add the \li{__eq__} magic method to Backpack.py so that two backpack objects are equal if and only if they have the same name, color, and contents. Note that the contents do not need to be in the same order for the contents to be the same.
\end{enumerate}
\end{problem}

% Problem 4: ComplexNumber class
\begin{problem}
Create your own \li{ComplexNumber} class that supports the basic operations of complex numbers.
The constructor should accept two numbers:
store the first as \li{self.real} and the second as \li{self.imag}.

Implement a \li{conjugate} method that returns a \li{ComplexNumber} object that is the object's complex conjugate, and a \li{norm} method that returns the magnitude of the complex number (as a \li{float}).
Neither of these methods should receive any arguments.

Also implement the following magic methods appropriately:
\begin{enumerate}
\item \li{__add__(self, other)}
\item \li{__sub__(self, other)}
\item \li{__mul__(self, other)}
\item \li{__div__(self, other)} (Use the complex conjugate to implement division)
\end{enumerate}
Each of these should return a new \li{ComplexNumber} object.

\end{problem}
