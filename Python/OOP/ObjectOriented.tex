\lab{Object-Oriented Programming}{Object Oriented Programming}
\label{lab:OOP}
\objective{Teach object-oriented programming in Python.}

\section*{Introduction}

Writing readable code is an important skill for computer programmers.
Well-written code is easy to understand and modify.
An important part of readable code is how it is organized.
Object-oriented programming is one way to accomplish good code organization.

Python is a class-based language.
A class is a sort of blueprint for an object.
Calling a class constructs an object.
Once constructed, the user has access to an object's methods and attributes.
An object created by a class is called an \emph{instance} of the class.
It is also said that a class \emph{instantiates} objects.

\section*{Python Classes}

A Python class is code that determines an object's behavior.
A class is defined using the the \li{class} command to name the class.
Other statements follow the \li{class} command to determine the behavior of objects instantiated by the class.
Finally, a class needs a method called a constructor that is called when the class instantiates a new object.
The constructor specifies the initial state of the object.
In Python, a class's constructor is always named \li{__init__}.
For example, the following is a class that created backpack objects.

\begin{lstlisting}
class Backpack(object):
    """Backpack object. Has a color, list of contents, name, and max size.
    
    Attributes:
        color (str): the color of the backpack. Defaults to 'black'.
        contents (list): the contents of the backpack.
        name (str): the name of the backpack. Defaults to 'backpack'.
        max_size (int): the maximum numer of items that can fit in the
            backpack. Defaults to 5.
    """
	
	def __init__(self, color = 'black'):
		'''Constructor for a backpack object.

		Inputs:
			color (String) - The color of the backpack

		Returns:
			A backpack object with no contents.
		'''

		self.color = color
	
		# Create a list that will store the contents of the backpack.
		self.contents = []

	def put(self, item):
		'''Add an item to the backpacks content list.'''
		self.contents.append(item)

	def take(self, item):
		'''Remove an item from the backpack's content list.'''
		self.contents.remove(item)

\end{lstlisting}

Note that the first argument of each method is called \li{self}.
This name refers to the object internally once it has been created.
There is no need to include the \li{self} argument when calling the methods defined by the class.

Classes may be imported in the same way was modules.
If the above module is saved as \li{Backpack.py}, it is possible to import the \li{Backpack} class from it.
Once imported, the \li{Backpack} class can be called to construct backpack objects.
\begin{lstlisting}
# From Backpack.py import the Backpack class.  Then create a backpack.
>>> from Backpack import Backpack
>>> my_backpack = Backpack('green')

# Add an item to the backpack.
>>> my_backpack.put('notebook')
>>> my_backpack.put('pencils')

# Access attributes of the object.
>>> my_backpack.color
'green'

>>> my_backpack.contents
['notebook', 'pencils']

# Remove an item from the backpack.
>>> my_backpack.take('pencils')
>>> my_backpack.contents
['notebook']
\end{lstlisting}

\begin{problem}
This exercise will expand the \li{Backpack} class.
Add attributes to the class called \li{name} and \li{max_size}.
The \li{name} attribute will be a label for the backpack.
The \li{max_size} attribute is the total capacity of the backpack.
Modify the constructor method to accept \li{name} and \li{max_size} in addition to \li{color}.
Set the default value of \li{name} to `backpack' and \li{max_size} to 5.

Once the new attributes are set, modify the \li{put} method.
Ensure that the backpack does not go over capacity.
If the user tries to put in more than max\_size items, print a message that says ``Backpack Full.''

Finally, add a new method to the backpack called \li{dump}.
This method removes all the contents of the backpack.
\end{problem}

\section*{Inheritance}

In object-oriented programming, inheritance is a tool for code reuse and organization.
To create a new class similar to one that already exists, it is sometimes easier to ``inherit'' the already existing methods.
This is done by including the existing class as an argument in the class definition.
This creates a \emph{class heierarchy}.
For example, a knapsack is a kind of backpack.
Thus, every knapsack is a backpack, but not all backpacks are knapsacks.
A class that inherits from another class is called a \emph{subclass}.
The class that a subclass inherits from is called a \emph{superclass}.
To inherit a \li{Knapsack} subclass from the \li{Backpack} class, use the following code:

\begin{lstlisting}
# First, import the Backpack class from Backpack.py
from Backpack import Backpack

# Inherit the Backpack class in the Knapsack definition
class Knapsack(Backpack):
	'''A class for creating knapsack objects.  Inherits from
	Backpack.  A knapsack is smaller than a backpack and
	is tied closed.
	'''
	def __init__(self, color='black', max_size=3):
		'''A knapsack only holds 3 objects by default instead of 5
		Use the Backpack constructor to initialize the name and 
		max_size attributes
		'''
		Backpack.__init__(self, color, max_size)

		# Add an attribute that keeps track of whether the knapsack is tied shut.
		self.closed = True
\end{lstlisting}

If methods in the new class need to be changed, they are overwritten as is the case of the constructor in the \li{Knapsack} class.
New methods are included normally.
For example, the \li{put} and \li{take} methods in \li{knapsack} are modified to check if the knapsack is shut. 
An \li{open} and \li{close} method are added as well.

\begin{lstlisting}
	def put(self, item):
		'''If the knapsack is untied, use the Backpack put() method.'''		
		if self.closed == True:
			print "Knapsack is closed!"
		else:
			Backpack.put(self, item)

	def take(self, item):
		'''If the knapsack is untied, use the Backpack take() method'''
		if self.closed == True:
			print "Knapsack is closed!"
		else:
			Backpack.take(self, item)

	def untie(self):
		'''Untie the backpack'''
		self.closed = False

	def tie(self):
		'''Tie the backpack'''
		self.closed = True

\end{lstlisting}

Since \li{Knapsack} inherits \li{Backpack}, a knapsack object is a backpack object.
All methods defined in the \li{Backpack} class are available to instances of the \li{Knapsack} class.
Thus, in this example, the \li{dump} method is available even though it is not defined explicitly in the \li{knapsack} class.

\begin{lstlisting}
# Import the Knapsack class from Knapsack.py
>>> from Knapsack import Knapsack
>>> my_knapsack = Knapsack()

# The put and take method now require the knapsack to be open.
>>> my_knapsack.put('compass')
Knapsack is closed!

# Open the knapsack and put in some items.
>>> my_knapsack.untie()
>>> my_knapsack.put('compass')
>>> my_knapsack.put('pocket knife')
>>> my_knapsack.contents
['compass', 'pocket knife']

# The dump method is inherited from the backpack class, and
# can be used even though it is not defined in the knapsack class.
>>> my_knapsack.dump()
>>> my_knapsack.contents
[]
\end{lstlisting}

\begin{problem}

Create a jetpack class that inherits the backpack class.
Overwrite the constructor so that the \li{max_size} defaults to 2 and \li{color} defaults to `silver.'
Also in the constructor initialize a \li{fuel} attribute that keeps track of how much fuel is left in the jetpack.
Set the default amount of fuel to be 10.
Finally, add a \li{fly} method that accepts an amount of fuel to be burned, and decrement the amount of fuel by that amount.
If the user tries to burn more fuel than remains, print ``Not enough fuel!"
Finally, overload the \li{dump} method so that both the contents and fuel are emptied.
\end{problem}

\section*{Magic Methods}

Python magic methods are a way to make objects behave like built-in data types.
For example, numbers can be added, subtracted, multiplied, divided, and compared.
Magic methods imitate this behavior for any object.
In the following class, adding two backpacks together combines their contents.

\begin{lstlisting}
class Backpack:
	
	def __init__(self, color = 'black'):
		self.color = color
		self.contents = []

	def put(self, item):
		self.contents.append(item)

	def take(self, item):
		self.contents.remove(item)

	def __add__(self, other):
		'''
		Add the contents of other to the contents of self
		Note that the contents of other are unchanged
		'''	
		self.contents = self.contents + other.contents
\end{lstlisting}

All magic methods begin and end with two underscores.
To demonstrate the addition method, create two backpacks and add them together.

\begin{lstlisting}
>>> from Backpack import Backpack
>>> backpack1 = Backpack()
>>> backpack2 = Backpack()

# Put some items in the backpacks
>>> backpack1.put('textbook')
>>> backpack2.put('water bottle')

# Now add the backpacks like numbers
>>> backpack1 + backpack2
>>> backpack1.contents
['textbook', 'water bottle']
>>> backpack2.contents
['water bottle']
\end{lstlisting}

Subtraction, division, or multiplication may be similary defined with the \li{__sub__}, \li{__div__}, and \li{__mul__} magic methods.
Magic methods also allow for comparisons.
The methods \li{__lt__}, \li{__le__}, \li{__gt__}, and \li{__ge__} methods correspond to $<$, $<=$, $>$, and $>=$ respectively.
For example, one backpack might be less than another if it has fewer items in contents.

\begin{lstlisting}
class Backpack:
	
	def __init__(self, color = 'black'):
		self.color = color
		self.contents = []

	def put(self, item)
		self.contents.append(item)

	def take(self, item):
		self.contents.remove(item)

	def __add__(self, other):
		self.contents = self.contents + other.contents

	def __lt__(self, other):
		'''
		Compare two backpacks.  If self has fewer contents than other,
		return True.  Otherwise, return False.

		Inputs:
			self (Backpack) - the backpack on the left-hand side of
			    the inequality.
			other (Backpack) - The backpack on the right-hand side of
			    the inequality.
		'''
		if len(self.contents) < len(other.contents):
			return True
		else:
			return False
\end{lstlisting}

To test this, create two backpacks and use the $<$ operator.
\begin{lstlisting}
>>> from Backpack import Backpack
>>> backpack1 = Backpack()
>>> backpack2 = Backpack()

>>> backpack1.put('book')
>>> backpack2.put('water bottle')
>>> backpack1 < backpack2
False

>>> backpack2.put('pencils')
>>> backpack1 < backpack2
True
\end{lstlisting}

A complete list of available magic methods can be found \href{https://docs.python.org/2/reference/datamodel.html#special-method-names}{here}.

\begin{problem}

The \li{__repr__} magic method is used to give the string representation of an object.
When the object is used with the \li{print} command, the \li{__repr__} method is called.
Add the \li{__repr__} method to Backpack.py so that \li{print backpack} yields
\begin{lstlisting}
 Color:		<color>
 Size:		<number of items in contents>
 Max Size:	<max_size>
 Contents:
 			<item1>
 			<item2> ...
\end{lstlisting}

If the backpack is empty, the contents line should read:
\begin{lstlisting}
Contents:	Empty
\end{lstlisting}
(Hint: Use the '\t' character to help align output.)

The \li{__eq__} magic method is used to determine if two objects are equal.
Add the \li{__eq__} magic method to Backpack.py so that two backpack objects are equal if and only if they have the same name, color, and contents.
Note that the contents do not need to be in the same order for the contents to be the same.
\end{problem}

\begin{problem}
Create a \li{ComplexNumber} class that supports the basic operations of complex numbers.
Implement the following magic methods.
\begin{enumerate}
\item \li{__add__(self, other)}
\item \li{__sub__(self, other)}
\item \li{__mul__(self, other)}
\item \li{__div__(self, other)} (Use the complex conjugate to implement division)
\end{enumerate}

Also implement a \li{conjugate} and \li{norm} method that return a \li{ComplexNumber} object that is the complex conjugate and the magnitude of the complex number, respectively.
\end{problem}
