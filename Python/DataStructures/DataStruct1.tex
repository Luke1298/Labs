\lab{Python}{Data Structures I}{Data Structures I}
\label{lab:Python_DataStructures}
\objective{Empower students with the basic knowledge regarding the fundamental canonical data structures in order to understand performance and runtime characteristics.}

At the core of every algorithm is how we store the data we are working with.
Storing and retrieving data takes time, and depending on the data structure used the time it takes grows with the size of the data set - whether that be logrithmic, linear, exponential, or factorial growth.
This means that to work with large data sets, we need to understand the characteristics of various data structures, so that we can choose the data structure that will help us efficiently access the data we need.

\section*{Abstract Data Types}

The most basic way to store data is via \emph{primitive data types}.
These data types are booleans, strings, floats, and integers.
Most information that we care to store is in one of these forms;
however, these primitive types can quickly become unwieldy for storing large amounts of information.
Imagine having a different name for every piece of data you use!
Luckily, we can create more natural ways to store data by using primitive data types, along with arrays, as our building blocks.
More complex data structures are called \emph{abstract data types}.
Python possesses a couple of the more common abstract data types, including dictionaries, sets, and lists.
All of the abstract data types slow down in performance as the size of the data structure increases. %

Most abstract data types use an object called a \emph{node} to store data.
A node essentially acts as a box in which we store an arbitrary piece of data.
Nodes can store absoluetly anything, and what they contain will depend on the data structure and what data is being stored.
A good way to understand an abstract data structure is to think of it like the postal delivery system.
We start by inserting an item into the postal system.  (Because the data stored in a node is arbitrary, our item can be anything.)
The first thing that the postal system will do with our item is put it into a box and attach delivery specific data that would allow it to be processed more effectively, 
like the return address or a little label indicating that the contents are ``fragile.''
The postal system now only has to efficiently handle boxes;
it doesn't have to worry about working with each item that goes through system.
Thus, the postal system has abstracted itself from the items it delivers.

If we didn't use nodes or ``data boxes'', we would have to build a new structure for each data type we wanted to store.
Thus, nodes allow us to abstract the data structure from the kind of data we stored inside.
Nodes allow us to generalize the functionality of a data structure.

\section*{Linked Lists}
\emph{Linked lists} are one of the most common and basic data structures.
A linked list is a list of nodes that are linked together.
There are several ways we can link nodes together.  
The three most common types of linked lists are singly-linked, doubly-linked, and circularly-linked.
Singly-linked nodes store only a single reference that points to the next node in the list.
Doubly-linked nodes have two references:
one that points to the previous node and one that points to the next node in the list.
This allows for a doubly-linked list to be traversed in both directions, whereas a singly-linked list can only be traversed in one direction.
A circularly linked list is a singly or doubly-linked list where the tail points to the head as its next node and the head points to the tail as its previous.

%TODO: pictures

Unlike arrays, which have random access, a
linked list has a reference to where the head is, but can only walk through each node to find later ones in the list.
This means that to insert or delete nodes in the middle of the list, we must first locate the correct node by traversing the list - starting at the head and visiting each node.
Thus nodes can be inserted or removed from either end of a linked list in constant time, independent of the size of the list, but inserting and removing in the middle takes longer as the list grows.

Linked Lists may also have a tail reference that points to the last node of the list and a counter to keep track of the size of the list.
Keeping a counter takes memory, but that way we don't need to recount the list everytime we want to know the size.


\subsection*{Inserting and Removing}

To insert into the end of a linked list, we need to make the old tail of the list reference the new node, increase the size counter, and update the reference to the tail of the list (if we have either).

To remove a node, since it is in the middle of the list, we need to update the reference in the node before, the reference in the node after (if it is a doubly-linked list), and decrease the size counter (if we have one).
If we are removing the last node in the list, we also need to update the reference to the tail, and if it is the only node in the list we need to set the reference to the head as \li{None}.


\begin{problem}
Implement a singly-linked list with methods for inserting, removing, and finding nodes. Insert nodes at the end of the list, but implement a method such that you can remove a node from anywhere in the list. The code below gives a structure for the class as well as methods for clearing the list and representing the list as a string.
\begin{lstlisting}
class Node():
    def __init__(self):
        self.right=None
        self.value=None
class Linkedlist():
    def __init__(self):
        self.head=None
    def insert(self,data):
        pass
    def remove(self,data):
        pass
    def find(self,data):
        pass
    def clear(self):
        self.head=None
    def __repr__(self):
        temp=self.head
        string='['
        while temp!=None:
            string+=str(temp.val)+','
            temp=temp.right
        string+=']'
        return string
\end{lstlisting}
\label{prob:LinkedLists}
\end{problem}

If you are familiar with other coding languages, you might be a bit worried about the \li{clear} method.
You would think that setting the reference to the head to \li{None} would cause us to lose all access to all the nodes in the list without being able to delete them!
In other languages, this would indeed cause serious memory leaks.
However, Python is able to keep track of which variables are being used, and if there is no access to them, it automatically deletes them.
This allows us to set the reference to the head to be None and let Python clean up the leftover nodes.

\section*{Stacks, Queues, and Deques}
\emph{Stacks}, \emph{queues} (pronounced `cue'), and \emph{deques} (pronouced `deck') can be thought of as types of linked lists where access is restricted. Rather than accessing data anywhere in the list, we only add and remove from certain sides.
This means that rather than searching for a certain piece of data to remove we use them to keep track of the order the data was added in and simply ask for what is on the end.

Stacks are built on the in `last in, first out' principle, meaning that the last item that was put in the stack will be the first one to leave. You can imagine this as a pile of plates in the kitchen cupboard. Rarely, if ever, do you take the plate on the bottom of the pile. Instead you take the one on the top. Also, when putting plates in the cupboard, you put them on top of the pile. This means that the last plate put in will be the first one taken out. To implement this, we would only add and remove from the same side of a linked list.

Queues use `first in, first out,' meaning that the first thing put in will be the first thing taken out. This is similar to people waiting in line. The first person in line will be the first served and taken out of the line. This is implemented as a linked list where we add on the right and remove on the left.

Deques are double ended queues, meaning that we can add and remove on either end. Really the only difference between this and a linked list is that we cannot add and remove from the middle.

\begin{problem}
Implement a queue and a stack by having them inherit from your linked list class.
Have the stack add and remove from the right.
Have the queue add on the right and remove from the left.
\label{prob:Stack}
\end{problem}

\section*{Hash Tables}
A \emph{hash table} is a very simple data structure that trades space for speed.
Most of the operations of a hash table execute in constant time, independent of the size of the data structure.
As such, hash tables have very fast lookup times;  they form the underlying data structure of Python's dictionaries.

In essence, a hash table is an array, but rather than walking through to look for our a piece of data, we have a \emph{hash function} that takes our data as an input and then outputs a positive integer used to index the array.
Since the hash function must be executed to perform any operation on the hash table, it is important that the hash function executes quickly.
Thus, the heart of a hash table is a good hash function.

What makes a good hash function?
Hash functions need to distribute items evenly throughout the hash table.
Since we usually mod by the table size at the end, we need to make sure the remainders mod $n$ are evenly distributed.
For example, if our hash function is the length of a string mod $n$, once the table gets large very few strings will be hashed to the end of the table.
We can fix this by multiplying by a large prime number.
A large number helps guarantee that the remainders will vary, and prime rids us of the chance that we are multiplying by a multiple of the table size (which would send us to 0 every time). 

Let's walk through a simple example where we use a hash table to store strings.
Our hash function will be the length of the string mod the size of the table.
Thus, if our table has size 4, `cat' would be stored in spot 3.
Similarly, `lion' would be sent to 4 \% 4 = 0.
\[
\begin{tabular}{|c|c|c|c|}
\hline
0 & 1 & 2 & 3\\
\hline
`lion' & & & `cat' \\
\hline
\end{tabular}
\]


\begin{problem}
Implement a hash table to store tuples of names and i.d.s with methods to add and remove items.
Implement it so that when a hash table object is created, you pass in a parameter to specify the size.
Have your hash funtion take the i.d., multiply by 7927 (a prime number), and mod by the size of the table.
\label{prob:Hash1}
\end{problem}

In our earlier example, our hash function is far from perfect.
If we try to store `leopard', our hash function sends it to 7 \% 4 = 3.
But `cat' is already stored in spot 3. 
This is called a \emph{hash collision}.
So how can we deal with collisions?

An ideal hash function will map unique inputs to unique outputs that are uniformly distributed over the hash space.
However, it is very difficult to create an ideal hash function. 
Most hash functions will experience hash collisions.
Fortunately, there are ways to handle hash collisions. 
The two methods that we will discuss are open addressing and chaining.

\subsection*{Open Addressing}
The term \emph{open addressing} indicates that the output of the hash function doesn't necessarily identify the location of some data.
One popular form of open addressing is \emph{probing}.
We will discuss linear probing and quadratic probing.

In the event of a hash collision, linear probing seeks to resolve the collision by looking for the next available location in the hash table.
We can do this by sequentially visiting each location and when we find an empty one, storing our data in that location.
The following is an example of a linear probing hash function where $n$ is the size of the hash table, $h(x)$ is the hash function, and $i$ is a constant:
\begin{equation*}
h(x, i) = h(x) + i \pmod{n}
\end{equation*}
We start with $i = 0$, and if the spot is full, we iterate over $i$ until we find an empty spot.
However, when we want to retrieve that information we will be directed to the wrong location when we use our hash function.
We then have to begin iterating through the table to look for our data.

If we go back to our earlier example, we can store `leopard' in our hash table using linear probing.
Our hash function sends us to spot 3, but it is already filled, so we start stepping through the table.
Since 3 is at the end of the table, we go to the begining, spot 0 is already filled, so we go to spot 1.
Spot 1 is empty, so we store `leopard' in spot 1.
\[
\begin{tabular}{|c|c|c|c|}
\hline
0 & 1 & 2 & 3\\
\hline
`lion' & `leopard' & & `cat' \\
\hline
\end{tabular}
\]
Now we want to find `leopard'.
Our hash function sends us to 3, and we find `cat', so we start stepping through the table.
Eventaully we find `leopard' in spot 1.

%TODO: picture

With linear probing, elements in the table will cluster together.
Also, if our hash table is densely populated or our hash function has a lot of collisions,
 we lose the efficiency of a hash table because we resort to a linear search.
Quadratic probing seeks to mitigate the issues of linear probing by spreading out hash collisions more evenly.
A quadratic probing hash function where $c_1$ and $c_2$ are constants is of the form
\begin{equation*}
h(x, i) = h(x) + c_1i + c_2i^2 \pmod{n}
\end{equation*}

\subsection*{Chaining}
Chaining is a form of closed addressing, meaning that the output of the hash function points to the location of the data.
With chaining, each location of the hash table references a list.
When one or more pieces of information hash to the same location, it is added to that location's list.
With a good hash function, the average size of hash location's list is relatively short,
so searching the list doesn't affect the overall performance of the hash table.

Let's add `leopard' to our earlier example, this time using chaining.
Our hash function sends us to 3, but `cat' is already there.
Thus we add `leopard' to the list in spot 3 that already contains `cat'.
\[
\begin{tabular}{|c|c|c|c|}
\hline
0 & 1 & 2 & 3\\
\hline
`lion' & & & `cat', `leopard' \\
\hline
\end{tabular}
\]

\begin{problem}
Improve your hash table from Problem \ref{prob:Hash1} to deal with hash collisions via chiaining.
\end{problem}

\subsection*{Hash Table Details}
As a hash table fills up, its performance degrades.
This is becuase more hash collisions occur, causing us to use linear probing or chaining and we end up searching through lists more often.
To combat this, a \emph{load factor} is often tracked.
This load factor reveals how `saturated' the table is by calculating the ratio of the number of empty locations to the size of the hash table.
When the load factor exceeds a certain threshold, we must allocate more space for the table.
However, since the hash function was dependant on the size of the table, we need to re-hash everything already in the table. 
This can sometimes be a very expensive operation.
It's important to avoid resizing the table whenever possible.

After adding `leopard' into our earlier example, our hash table is very full. Let's re-hash it into a table with size 8.
Our new hash function sends a string to it's size mod 8 (rather than mod 4 like before).
`cat' gets sent to 3, `lion' to 4, and `leopard' to 7. 
Now that we have no hash colisions there are no lists to iterate through and our hash table performs much better.

\[
\begin{tabular}{|c|c|c|c|c|c|c|c|}
\hline
0 & 1 & 2 & 3 & 4 & 5 & 6 & 7 \\
\hline
 & & & `cat' & `lion' & & &`leopard' \\
\hline
\end{tabular}
\]

\begin{problem}
Create methods for your hash table to resize.
When the load factor exceeds $.75$, resize the hash table so that the load factor will be below $.33$.
\label{prob:Hash3}
\end{problem}

\section*{Trees}
``Trees sprout up just about everywhere in computer science'' - Donald Knuth

Trees are similar to linked lists, but rather than just pointing to one node that comes after them, they can point to many.
If our data can be sorted, we can add them in such a way that we know how to traverse the tree to find them, which can substantially cut down search time.
This makes trees especially good if we need fast search times.

There are many types of trees, but we will only cover one of the most basic ones: the \emph{binary search tree} (BST).
The binary search tree will show us the basics elements of the tree data structure.
A binary tree is a tree that has a maximum of two children per node,
has no duplicate elements, and has an ordered structure.
Trees are commonly ordered such that the left half of the tree contains elements less than the root, and the right contain elements greater than the root.

Odering the tree this way means that the very first time we choose right or left, we cut out half of the tree!
Similarly, at each step we cut out half of the remaining nodes to search.
This makes searching a binary tree very fast.
Because the process of locating a node is so quick, the processes of inserting and removing nodes are as well.

A node in a binary search tree holds a value and references to the left and right subtrees.
A node's children are the nodes it references.
The root of the tree has no parents, and is where we always start searching.
A node with no children is called a leaf node.

\begin{lstlisting}
class Node(object):
    def __init__(self, data):
        self.data = data
        self.left = None
        self.right = None
\end{lstlisting}

BSTs are a recursive data structure because each of the left and right subtrees are themselves BST's.

Many of the algorithms for working with binary search trees are also recursive.
A recursive function is a function that calls itself until it reaches a base case.
In the search function below, the two base cases either establish that we have the correct
node, or that we don't have a node at all.
In both these cases, we simply return what we have.
If neither base case applies, we search either the left or the right subtree depending on if the data we are looking for is greater or less than the data in the node we are looking at.
While recursive functions are often useful and intuitive with trees, we don't have to use them if we don't want to.
In fact, for large trees, a recursive algorithm will consume a lot of resources.

\begin{lstlisting}
def search(data, node):
    if node is None or node.data == data:
        return node
    elif data < node.data:
        search(data, node.left)
    else:
        search(data, node.right)
\end{lstlisting}

Inserting a node into a binary search tree is a simple procedure.
Our goal is to insert into a leaf node, so
we recursively insert to the right if the element is greater than the current node
or insert to the left if the element is less than the current node.
Once we reach a node without a left or right subtree we add our new node as that subtree.
Thus, we always insert into leaf nodes.
We never insert a non-leaf node into a BST.

Let's walk through an example.
If we start with an empty tree, and add 23 and 17 then
23 becomes the root of the tree, and 17 the left child becuase $17 < 23$. 
Then if we add 476, it becomes the right child of 23 becuase $23 < 476$.
Next, if we add 28, we start at the root, then go to the right to 476 because 28 is greater than the root.
Next, 28 becomes the left child of 476.

TODO: pictures showing each step.

\begin{problem}
Implement a binary tree with methods for finding and inserting nodes.
Use the \li{Node} class given above for your node objects.
\label{prob:BST1}
\end{problem}

Removing nodes is only slightly more complicated.
We have three cases to consider:
\begin{enumerate} % TODO: need examples of all of these.
\item No children.  This is the simplest case.  If the node we want to delete has no children then we simply remove the node.
\item One child.  In this case, we have to worry about children.
If we were to just delete the node, we would lose the entire subtree.
The solution, however, is simple enough:
we promote the child to take the place of the node we are deleting.
\item Two children.  We need to be a little careful in this case.
We can't simply promote a child node.  Indeed, how would we decide which child to promote?
Fortunately, we can reduce this case down to one of the previous two cases.
Suppose we are removing node, $n$, which has two children.
We first locate either the smallest node of the right subtree or the greatest node of the left subtree.
Let's call that $c$.
We swap the data of nodes $c$ and $n$ (so node $c$ now has the data that $n$ had and vice versa).
Due to how we found $c$, we know that it either has one child or no children and we can now simply remove $c$ using one of the simpler cases.
\end{enumerate}

\begin{problem}
Give your binary tree class from Problem \ref{prob:BST1} a method for removing nodes. 
Make sure  to account for all the cases given above.
\label{prob:BST2}
\end{problem}

\subsection*{Balanced Trees}
Binary search trees often perform very well in searching for data.
However, the order of insertion into a binary search tree largely determines how well the tree performs.
Binary search trees work best when elements are added in a random order.
This keeps the tree relatively full - each level has the maximum number of nodes -  rather than having some subtrees being much larger than others.
If the elements are sorted before adding to a BST, the efficiency of a BST is completely mitigated.
The resulting degenerate tree is essentially a linked list.

One method for solving these problems is to keep the tree balanced.
On each insertion and removal, the tree is re-balanced to maintain optimal performance.
AVL Trees and Red-Black Trees are examples of balanced trees.

TODO: pictures of a full tree versus adding in already sorted data.

\section*{Heaps}

A \emph{heap} is another tree-based data structure; however, heaps are only partially ordered and are always full.
This means that parents are ordered with respect to children, but unlike a BST, siblings or cousins have no order. 
However, becuase heaps are only partially ordered, searching for nodes other than the root is not as efficient as a BST.
Heaps are commonly used for \emph{priority queues}, because the only element you need access to is the one with the greatest priority.
Max and min heaps are also common and are used when you need to find the maximum or minimum in constant time.
In a max heap, each parent is greater than its children.
This puts the maximum at the root of the tree.

%TODO: pictures of max and min heaps 

To add into a heap, we put the node into an empty spot on the lowest level of the tree.
Then we compare it to its parent and if it is greater than its parent, we switch the two nodes.
We continue this until the new node reaches a parent that is greater than it or it becomes the new root.

Since heaps are really only useful for keeping track of the maximum, removing the maximum is easy, but removing anything else is not.
To remove the maximum (the root), we compare the root's children and promote the greatest to be the new root.

%TODO: pictures of insert and removing from a max heap

\section*{Sparse Matrices}
Matrices are powerful tools, but they can take a lot of space in memory.
A $n \times n$ matrix takes $n^2$ places of storage!
If every entry in the matrix is useful information this is fine, but if the matrix is filled with a lot of zeros, that means we are using almost $n^2$ spots in memory to store useless zeros.
Not only does this unnessesarily take up space, it means that during computation we will spend a lot of time adding and multiplying zeros.
That can be a lot of useless computation. 
The solution to this is \emph{sparse matrices}.
They are other data structures used to represent a matrix by only storing non-zero entries.

One common way is a dictionary.
The index of the spot in the matrix is the key and the number in that spot is the value.
For example, 
\[
\begin{pmatrix}
0 & 0 & 0 & 0 & 0 \\
22 & 0 & 0 & 0 & 0 \\
0 & 0 & 0 & 0 & 0 \\
0 & 0 & 5 & 0 & 0 \\
0 & 0 & 0 & 0 & -4
\end{pmatrix}
\]
would be stored as 
\[
\begin{tabular}{|c|l|}
\hline
Key & Value \\
\hline
(1,0) & 22 \\
\hline
(3,2) & 5 \\
\hline
(4,4) & -4 \\
\hline
\end{tabular}
\].
 
In this case, we stored a 5 by 5 matrix in a dictionary with only 3 key-value pairs.
 
SciPy has a nice implementation of sparse matrices.
Not only does SciPy reduce the amount of space needed to store matrices, it has fast algorithms for matrix operations that takes advantage of the fact that there are so many zeros.
 
\begin{problem}
Use your time function from Section 1 and the code given below to time matrix multiplication using sparse matrices versus normal matrix multiplication for $ n = 100, 500, 1000, 1500,$ and $2000$. Which is faster? What happens as the matrices grow larger?
\begin{lstlisting}
from scipy import sparse
import numpy as np

#dimension of matrices
n = 100

# create 2 random sparse matrices with density 0.1
A = sparse.rand(n,n)
B = sparse.rand(n,n)

#Multiply using sparse matrix techniques
A.dot(B)

#Change to dense matrices
A_dense = A.todense()
B_dense = B.todense()

#Multiply with normal matrix multiplication
A_dense.dot(B_dense)

 \end{lstlisting}
 \label{prob:Sparse}
 \end{problem}

\begin{comment} %move to start of Kevin Bacon
\section*{Graphs}
We commonly use graphs in numerical computations.
There are two different data structures that can be used to represent graphs.
Each data structure has its own advantages and disadvantages.
Which one you use depends greatly on they type of graph problem you are solving.
The two ways to represent graphs are adjacency matrices and adjacency lists, which can both
 be used to represent directed or undirected graphs.

\subsection*{Adjacency Matrices}
An Adjacency matrix is two dimensional matrix used to represent a graph.
If a graph has $n$ nodes, then representing it as an adjacency matrix requires a $n \times n$ matrix.
The $ij$th entry of the adjacency matrix represents the presence of an edge between nodes $i$ and $j$.
If the entry is 0, then no edge exists between nodes $i$ and $j$, otherwise there is is an edge between nodes $i$ and $j$.
An adjacency matrix allows us to query the existence of an edge, or its edge weight, in constant time.

\subsection*{Adjacency Lists}
We can also represent graphs as a nested list of neighbors.
The $i$th index in the list contains a list of adjacent nodes to node $i$.
If node $j$ is in this list, then there exists an edge between $i$ and $j$.
We can query the neighbors of any node in the graph in constant time.
This makes algorithms that operate locally on the graph very efficient.

\begin{problem}
Implement methods that will return a list of adjacent nodes given either an adjacency list or adjacency matrix.
\end{problem}

\end{comment}


