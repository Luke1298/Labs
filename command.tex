%------------------------------------------------------------------------------%
% command.tex                                                                  %
% This file contains the various environments and other misc. commands         %
%------------------------------------------------------------------------------%

%BibTeX project-wide style
\bibliographystyle{alpha}

%counter for problems. reset each chapter
\newcounter{problemnum}[chapter]
\newtheoremup{problemnum}{Problem}

\newcommand{\objective}[1]{{\bf Lab Objective: } \emph{#1} \bigskip}
\renewcommand{\chaptername}{Lab}
\renewcommand{\bibname}{References}

\newcommand{\lab}[3]{\chapter[#3]{#1: #2}}

% Various commands that make life easier
\newcommand{\argmax}{\mbox{argmax}}
\newcommand{\indicator}[1]{\mathbbm{1}_{\left[#1\right] }}
\providecommand{\abs}[1]{\left\lvert#1\right\rvert}
\providecommand{\norm}[1]{\left\lVert#1\right\rVert}
\providecommand{\set}[1]{\lbrace#1\rbrace}
\providecommand{\setconstruct}[2]{\lbrace#1:#2\rbrace}
\DeclareMathOperator{\res}{res}           % Residue
\DeclareMathOperator{\Res}{Res}           % Residue

\newcommand{\li}[1]{\lstinline[prebreak=]!#1!}

\newcommand{\ipt}[2]{\langle #1,#2 \rangle}
\newcommand{\ip}{\int_{-\infty}^{+\infty}}

\renewcommand{\ker}[1]{\mathcal{N}(#1)}
\newcommand{\ran}[1]{\mathcal{R}(#1)}

% Full line comments in the Algorithmic environment.
\algnewcommand{\LineComment}[1]{\State \(\triangleright\) #1}

\newenvironment{amatrix}[1]{%
\left(\begin{array}{@{}*{#1}{c}|c@{}}
}{%
\end{array}\right)
}

\newenvironment{dmatrix}[2]{%
\left(\begin{array}{@{}*{#1}{c}|*{#2}{c}@{}}
}{%
\end{array}\right)
}


%%Frame environments
\definecolor{shadecolor}{gray}{0.90}
\mdfdefinestyle{problem}{backgroundcolor=shadecolor,
                         hidealllines=true,
                         skipabove=10pt,
                         skipbelow=10pt,
                         innertopmargin=15pt,
                         innerbottommargin=15pt,
                         innerleftmargin=15pt,
                         innerrightmargin=15pt}

\definecolor{warning}{RGB}{255,231,231}
\definecolor{warnline}{RGB}{255,15, 15}
\newmdenv[
  roundcorner=10pt,
  skipabove=10pt
  skipbelow=10pt
  leftmargin=20pt,
  rightmargin=20pt,
  backgroundcolor=warning,
  innertopmargin=10pt,
  innerbottommargin=10pt,
  innerleftmargin=10pt,
  middlelinewidth=0pt,
  everyline=true,
  linecolor=warnline,
  linewidth=1pt,
  font=\normalfont\normalsize,
  frametitlefont=\large\bfseries,
  frametitleaboveskip=1em,
  frametitlerule=true,
  frametitle={\sc Warning}
]{warn}

\definecolor{information}{RGB}{231,231,255}
\definecolor{infoline}{RGB}{15,15, 255}
\newmdenv[
  roundcorner=10pt,
  skipabove=10pt
  skipbelow=10pt
  leftmargin=20pt,
  rightmargin=20pt,
  backgroundcolor=information,
  innertopmargin=10pt,
  innerbottommargin=10pt,
  innerleftmargin=10pt,
  middlelinewidth=0pt,
  everyline=true,
  linecolor=infoline,
  linewidth=1pt,
  font=\normalfont\normalsize,
  frametitlefont=\large\bfseries,
  frametitleaboveskip=1em,
  frametitlerule=true,
  frametitle={\sc Note}
]{info}

\newenvironment{problem}{\begin{mdframed}[style=problem]\begin{problemnum}}{\end{problemnum}\end{mdframed}}

